\section{P and NP}

%%%%%%%%%%
\begin{frame}{Computability theory first}
  \begin{theorem}{}
    Halting problem is undecidable.
  \end{theorem}

  \begin{proof}
    \fignocaption{width = 0.15\textwidth}{fig/turing.jpg}
  \end{proof}
\end{frame}
%%%%%%%%%%
\begin{frame}{Complexity theory to follow}
  \begin{columns}[t]
    \column{0.50\textwidth}
      Is a given Sodoku configuration solvable?

      \fignocaption{width = 0.40\textwidth}{fig/sudoku.png}
    \column{0.50\textwidth}
      Is $3 \times 3 \times 3$ Rubik's Cube solvable in 20 moves?

      \fignocaption{width = 0.45\textwidth}{fig/rubik.jpg}
  \end{columns}
\end{frame}
%%%%%%%%%%
\begin{frame}{Decision problems}
  \begin{definition}[Decision problems.]
    Decision problems are problems whose solution is ``Yes/No''.
  \end{definition}

  \begin{alertblock}{Remarks.}
    \begin{itemize}
      \item decision problem vs. optimization problem
      \item decision problem is ``hard'' $\Rightarrow$ its optimization problem is ``hard''
    \end{itemize}
  \end{alertblock}
\end{frame}
%%%%%%%%%%
\begin{frame}{Decision problems vs. optimization problem}
  \begin{exampleblock}{INDEPENDENT SET}
    \begin{block}{Optimization problem.}
      \begin{description}
	\item[Instance:] Undirected graph $G = (V, E)$.
	\item[Question:] Find the maximal independent set in $G$.
      \end{description}
    \end{block}

    \vspace{-0.30cm}
    \fignocaption{width = 0.20\textwidth}{fig/independent-set.png}
    \vspace{-0.50cm}

    \begin{block}{Decision problem.}
      \begin{description}
	\item[Instance:] Undirected graph $G = (V, E)$ and an integer $k$.
	\item[Question:] Does $G$ has an independent set of size (at least) $k$?
      \end{description}
    \end{block}
  \end{exampleblock}
\end{frame}
%%%%%%%%%%
\begin{frame}{The class P}
  \begin{definition}[The class P]
    P is the class of decision problems that are solvable in Polynomial time.
  \end{definition}

  \begin{exampleblock}{Examples for the class P}
    \begin{columns}[t]
      \column{0.30\textwidth}
	\begin{center}
	  Euler path.
	\end{center}
	\fignocaption{width = 0.75\textwidth}{fig/euler-bridge-graph.png}
      \column{0.30\textwidth}
	\begin{center}
	  Maze problem.
	\end{center}
	\fignocaption{width = 0.80\textwidth}{fig/maze.jpg}
      \column{0.30\textwidth}
	\begin{center}
	  Primality testing.
	\end{center}
	\fignocaption{width = 0.90\textwidth}{fig/primes.png}
    \end{columns}
  \end{exampleblock}
\end{frame}
%%%%%%%%%%
\begin{frame}{The class NP}
  \begin{definition}[The class NP]
    NP is the class of decision problems that are solvable in Polynomial time by \textcolor{blue}{Non-deterministic} algorithm.
    \[
      \text{NP} \neq \textrm{\textcolor{red}{N}on-\textcolor{red}{P}olynomial}
    \]
    \[
      \text{NP} \neq \textrm{\textcolor{red}{N}o \textcolor{red}{P}roblem}
    \]
  \end{definition}

  \pause
  \begin{definition}[Non-deterministic polynomial algorithm.]
    Given an instance $\mathcal{I}$ of a decision problem:
    \begin{description}
      \item[Guessing:] generate a certificate $c$ for $\mathcal{I}$
      \item[Verifying:] $V(\mathcal{I},c)$
    \end{description}

    \[
     O(\text{Guessing}) + O(\text{Verifying}) = O(n^{c})
    \]
  \end{definition}

\end{frame}
%%%%%%%%%%
\begin{frame}{Proof of being in NP}
  \begin{theorem}{}
    INDEPENDENT SET $\in$ NP.
  \end{theorem}

  \begin{proof}
    Given $G = (V,E)$ and $k$:
    \begin{description}
      \item[Guessing:] Nondeterministically select a subset $c$ of $k$ vertices of $G$. 
      \item[Verifying:] Test whether $G$ contains no edges for all vertices pairs in $c$.
      \item[Output:] If the test passes, ouput ``yes''; otherwise, output ``no''.
    \end{description}
    The complexity is $O(n^{2})$.
  \end{proof}
\end{frame}
%%%%%%%%%%
\begin{frame}{The class NP}
  \begin{exampleblock}{Examples for the class NP}
    \begin{columns}[t]
      \column{0.30\textwidth}
	\begin{center}
	  Hamiltonian path.
	\end{center}
	\fignocaption{width = 0.75\textwidth}{fig/hamiltonian.png}
      \column{0.30\textwidth}
	\begin{center}
	  Clique problem.
	\end{center}
	\fignocaption{width = 0.80\textwidth}{fig/clique.jpg}
      \column{0.30\textwidth}
	\begin{center}
	  Knapsack problem.
	\end{center}
	\fignocaption{width = 0.70\textwidth}{fig/knapsack.png}
    \end{columns}
  \end{exampleblock}
\end{frame}
%%%%%%%%%%
\begin{frame}{P vs. NP}
  \begin{block}{P vs. NP}
    \[
      \text{P: polynomially solvable}
    \]

    \[
      \text{NP: polynomially verifiable}
    \]
  \end{block}

  \fignocaption{width = 0.50\textwidth}{fig/p-vs-np-dollars.jpg}
\end{frame}
%%%%%%%%%%
\begin{frame}{P vs. NP}
  \begin{theorem}{}
    \[
      \text{P} \subseteq \text{NP}.
    \]
  \end{theorem}

  \begin{proof}
    To design a non-deterministic polynomial algorithm given a deterministic polynomial algorithm (PA).
    \begin{description}
      \item[Guessing:] generate a certificate $c$ for instance $\mathcal{I}$.
      \item[Verifying:] ignore $c$; output $\text{PA}(\mathcal{I})$.
    \end{description}
  \end{proof}
\end{frame}
%%%%%%%%%%
