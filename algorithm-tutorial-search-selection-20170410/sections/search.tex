\section{Searching}

%%%%%%%%%%%%%%%%%%%%
\begin{frame}{$\max/\min$ differences (Problem 4.5)}
  \begin{enumerate}[(a)]
	\item unsorted; $\max |x - y|$; $O(n)$
	\item sorted; $\max |x - y|$; $O(1)$
	\item unsorted; $\min |x - y|$; $O(n \log n)$
	\item sorted; $\min |x - y|$; $O(n)$
  \end{enumerate}
\end{frame}
%%%%%%%%%%%%%%%%%%%%
\begin{frame}{Searching in matrix (Problem 4.6)}
  \begin{itemize}
	\item $M$: matrix $m \times n$
	\item row: increasing from left to right
	\item col: increasing from top to down
	\item Is $x \in M$?
  \end{itemize}

  \pause
  \centerline{Divide and conquer.}

  \pause
  \[
	T(m,n) = 3T(\frac{m}{2}, \frac{n}{2}) + 1 
  \]

  \pause
  \[
	m = n \implies T(n) = 3T(\frac{n}{2}) + 1 \pause \implies T(n) = \Theta(n^{\log_{2}3})
  \]

  \pause
  \centerline{Always checking the lower left corner.}

  \pause
  \[
	T(m, n) = m + n - 1
  \]
\end{frame}
%%%%%%%%%%%%%%%%%%%%
\begin{frame}{Searching in matrix (Problem 4.6)}
  \[
	\text{Assume } M: n \times n 
  \]

  \[
	W(n) \le 2n - 1
  \]

  \pause
  \vspace{0.30cm}
  \centerline{$W(n) \ge 2n - 1$ by adversary argument!}

  \pause
  \begin{align*}
	i + j \le n - 1 &\implies x > M_{ij} \\
	i + j > n - 1 &\implies x < M_{ij} \\
  \end{align*}
\end{frame}
%%%%%%%%%%%%%%%%%%%%
\begin{frame}{Local minimum (Problem 4.11)}
  \begin{itemize}
	\item Array $A[0 \ldots n]$
	\item Boundary conditions:
	  \begin{align*}
		A[0] &\ge A[1] \\
		A[n-2] &\le A[n-1]
	  \end{align*}
	\item Local minimum $A[i]$:
	  \[
		A[i-1] \ge A[i] \le A[i+1]
	  \]
	\item Goal: Find \emph{any} local minimum.
  \end{itemize}
\end{frame}
%%%%%%%%%%%%%%%%%%%%
\begin{frame}{Local minimum (Problem 4.11)}
  \begin{enumerate}
	\item Checking each element:
	  \[
		T(n) = O(n)
	  \]
	\item $\min A$:
	  \[
		T(n) = O(n)
	  \]
	\pause
	\item Required:
	  \[
		T(n) = O(n \log n)
	  \]

	  \pause
	  \[
		T(n) = T(\frac{n}{2}) + 1
	  \]
  \end{enumerate}
\end{frame}
%%%%%%%%%%%%%%%%%%%%
\begin{frame}{Local minimum (Problem 4.11)}
  2D local minimum:
  \begin{itemize}
	\item Matrix $M: n \times n$
	\item Boundary conditions:
	  \[
		\infty
	  \]
	\item Local minimum $A[i,j]$:
	  \begin{align*}
		A[i,j-1] &\ge A[i,j] \le A[i,j+1] \\
		A[i-1,j] &\ge A[i,j] \le A[i+1,j]
	  \end{align*}
	\item Goal: Find \emph{any} local minimum.
  \end{itemize}

  \pause
  \[
	O(n^2) \pause \implies O(n \log n) \pause \implies O(n)
  \]
\end{frame}
%%%%%%%%%%%%%%%%%%%%
\begin{frame}{$a_i = i$ (Problem 4.2)}
  \begin{itemize}
	\item Sorted integer sequence $\set{a_1, a_2, \ldots, a_n}$:
	  \[
		\forall i \neq j: a_i \neq a_j
	  \]
	\item Goal: 
	  \[
		\exists? i: a_i = i
	  \]
  \end{itemize}

  \pause
  \[
	T(n) = O(n)
  \]

  \pause
  \[
	T(n) = T(\frac{n}{2}) + 1 = O(\log n)
  \]
\end{frame}
%%%%%%%%%%%%%%%%%%%%
\begin{frame}{Smallest missing positive integer (Problem 4.3)}
  \begin{itemize}
	\item Sorted array $A[1 \ldots n]$:
	  \begin{gather*}
		a_i \in \mathbb{Z}^{+} \\
		\forall i \neq j: a_i \neq a_j
	  \end{gather*}
	\item Goal: Find the smallest missing positive integer.
  \end{itemize}

  \pause
  \[
	T(n) = O(n)
  \]

  \pause
  \[
	T(n) = T(\frac{n}{2}) + 1 = O(\log n)
  \]
\end{frame}
%%%%%%%%%%%%%%%%%%%%
\begin{frame}{Computing $\lceil \sqrt{N} \rceil$ (Problem 4.12)}
  \begin{itemize}
	\item Given an $n$-bit natural number $N$ ($0 \le N < 2^{n} - 1$)
	\item Goal: Compute $\lceil \sqrt{N} \rceil$ using $O(n)$ additions and shifts.
  \end{itemize}
     
  \vspace{0.30cm}
  \begin{columns}
    \column{0.55\textwidth}
	  \pause
	  \begin{block}{Elementrary operations:}
	    \begin{itemize}
	      \item $n$-bit + $n$-bit: $O(1)$
	      \item $n$-bit shifted by $1$-bit: $O(1)$
	      \item $\Rightarrow x^2: O(n)$
	    \end{itemize}
	  \end{block}
	\column{0.45\textwidth}
	  \pause
	  \begin{enumerate}[<+->]
	    \item Na{\"i}ve search: $O(2^{n} \cdot n)$
	    \item Binary search: $O(n \cdot n)$
	    \item Binary search in range:
		  \[
			2^{\lfloor \frac{n-1}{2} \rfloor} \le \lceil \sqrt{N} \rceil \le 2^{\lceil \frac{n}{2} \rceil}
		  \]
		  \[
			\lg{(2^{\lceil \frac{n}{2} \rceil} - 2^{\lfloor \frac{n-1}{2} \rfloor})} = O(n)
		  \]
		  \[ O(n \cdot n) \]
	  \end{enumerate}
  \end{columns}
\end{frame}
%%%%%%%%%%%%%%%%%%%%
\begin{frame}{Computing $\lceil \sqrt{N} \rceil$ (Problem 4.12)}
  \begin{block}{A Little History:}
    \begin{description}
	  \item[$2007$:] Mid-term problem \\
		\centerline{$O(n)$ required; NO $O(n)$ solutions, however}
	  \item[$\sim 2013$:] $O(n^2)$
	  \pause
	  \item[$2014$:] $O(n)$
		\fignocaption{width = 0.80\textwidth}{figs/square-root-cs-se.png}
    \end{description}
  \end{block}

  \pause
  \[
	x^2: O(n) \to O(1)
  \]
\end{frame}
%%%%%%%%%%%%%%%%%%%%
\begin{frame}{Computing $\lceil \sqrt{N} \rceil$ (Problem 4.12)}
  Given
  \[ M = \lfloor N/4 \rfloor, x = \lceil \sqrt{M} \rceil,
    \textrm{ and } (x, x^2),
  \]
  what is
  \[ y = \lceil \sqrt{N} \rceil \textrm{ and } (y, y^2)? \]

  \pause
  \begin{exampleblock}{An Example:}
  \[
	\begin{array}{lll}
	  N = 280 & y = \lceil \sqrt{280} \rceil = 17 & y^2 = 289 \\ \pause
	  M = \lfloor 280/4 \rfloor = 70 & x = \lceil \sqrt{70} \rceil = 9 & x^2 = 81 \\ \pause
	  M = \lfloor 70/4 \rfloor = 17 & x = \lceil \sqrt{17} \rceil = 5 & x^2 = 25 \\
	  M = \lfloor 17/4 \rfloor = 4 & x = \lceil \sqrt{4} \rceil = 2 & x^2 = 4 \\
	  M = \lfloor 4/4 \rfloor = 1 & x = \lceil \sqrt{1} \rceil = 1 & x^2 = 1
	\end{array}
  \]
  \end{exampleblock}
\end{frame}
%%%%%%%%%%%%%%%%%%%%
\begin{frame}{Computing $\lceil \sqrt{N} \rceil$ (Problem 4.12)}
  \begin{algorithm}[H]
    \caption{Computing $\lceil \sqrt{N} \rceil$.}
    \begin{algorithmic}[]
      \Procedure{Sqrt-Root}{$N$}
	    \If{$N < 3$}
	      \State \Return $1 \Rightarrow (1,1); 2 \Rightarrow (2,4); 3 \Rightarrow (2,4)$
	    \EndIf
	    \State $M \gets \lfloor N/4 \rfloor$
	    \State $(x, x^2) \gets \textsc{Sqrt-Root}{(M)}$
		\State \Return the $(y, y^2)$ with $y^2 \sim N$:
           \[
			(y, y^2) = \left\{ \begin{array}{ll}
			y = 2x & y^2 = 4x^2 \\
			y = 2x + 1 & y^2 = 4x^2 + 4x + 1 \\
			y = 2x - 1 & y^2 = 4x^2 - 4x + 1
			\end{array} \right.
	      \]
	  \EndProcedure
    \end{algorithmic}
  \end{algorithm}
\end{frame}
%%%%%%%%%%%%%%%%%%%%
\begin{frame}{Computing $\lceil \sqrt{N} \rceil$}
  \[ \xout{T(n) = T(n/4) + O(1) = \Theta(\lg n)} \]

  \[ T(n) = T(n - 2) + O(1) = \Theta(n) \]
\end{frame}
%%%%%%%%%%%%%%%%%%%%
