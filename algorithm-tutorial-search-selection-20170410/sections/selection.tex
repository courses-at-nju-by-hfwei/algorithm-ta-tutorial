\section{Selection}

%%%%%%%%%%%%%%%%%%%%
\begin{frame}{The $3$rd largest element (Problem 3.1)}
  \[
	V_k(n): \text{}
  \]

  \pause
  \begin{align*}
	V_1(n) &= n - 1 \\
	V_2(n) &= (n - 1) + (\lceil \log n \rceil - 1) \\
  \end{align*}

  \pause
  \[
	V_3(n) = ?
  \]

  \pause
  \[
	V_3(n) \le (n - 1) + (\lceil \log n \rceil - 1) + \textcolor{red}{(n - 3)}
  \]

  \pause
  \[
	V_3(n) \le (n - 1) + (\lceil \log n \rceil - 1) + \textcolor{red}{(\lceil \rceil)} 
  \]
\end{frame}
%%%%%%%%%%%%%%%%%%%%
\begin{frame}{The 3rd largest element (Problem 3.1)}
  \begin{center}
	``What is the exact value of $V_3(n)$?'' \\[0.30cm] \pause

	``I don't know!''  
  \end{center}

  \pause
  \begin{alertblock}{Reference}
	``The Art of Computer Programming, Vol 3: Sorting and Searching'' by Donald E. Knuth.
  \end{alertblock}
\end{frame}
%%%%%%%%%%%%%%%%%%%%
\begin{frame}{The 3rd largest element (Problem 3.1)}
  \begin{center}
	``Does your algorithm need to find the $1$st and the $2$nd elements?'' \\[0.30cm] \pause

	``YES!''
  \end{center}

  \pause
  \begin{center}
	``Do all algorithms have to find the $1$st and the $2$nd elements?'' \\[0.30cm] \pause

	``NO!''
  \end{center}

  \pause
  \begin{alertblock}{References}
	``Selecting the Top Three Elements'' by Aigner, 1982.
  \end{alertblock}

  % \pause
  % \begin{description}
  %   \item[$V_t(n)$:] 
  %   \item[$W_t(n)$:] 
  %   \item[$U_t(n)$:] 
  % \end{description}

  % \pause
  % \begin{align*}
  %   U_1(n) &= V_1(n) = W_1(n) = n - 1 \\
  %   W_2(n) &= V_2(n) = n + \lceil \log n \rceil - 2
  % \end{align*}
\end{frame}
%%%%%%%%%%%%%%%%%%%%
\begin{frame}{The largest $k$ elements (Problem 3.5)}

\end{frame}
%%%%%%%%%%%%%%%%%%%%
\begin{frame}{Close to median (Problem 3.6)}
\end{frame}
%%%%%%%%%%%%%%%%%%%%
\begin{frame}{Medians of sorted arrays (Problem 3.7)}

\end{frame}
%%%%%%%%%%%%%%%%%%%%
\begin{frame}{Weighted median (Problem 3.9)}
\end{frame}
%%%%%%%%%%%%%%%%%%%%
%%%%%%%%%%%%%%%%%%%%

