\section{Amortized Analysis}

%%%%%%%%%%%%%%%%%%%%
\begin{frame}{Amortized analysis}
  \begin{quote}
	Amortized analysis is \\
	\textcolor{purple}{an algorithm analysis technique} for \\ 
    \textcolor{red}{analyzing a sequence of operations} \\
	\textcolor{blue}{irrespective of the input} to show that \\
    \textcolor{cyan}{the average cost per operation} is small, even though \\ 
    \textcolor{teal}{a single operation within the sequence might be expensive}.
  \end{quote}
\end{frame}
%%%%%%%%%%%%%%%%%%%%
\begin{frame}{Methods for amortized analysis: the summation method}
  \begin{gather*}
	o_1, o_2, \ldots, o_n \\[5pt]
	c_1, c_2, \ldots, c_n
  \end{gather*}

  \pause
  \[
	(\sum_{i = 1}^{n} c_i) / n
  \]
\end{frame}
%%%%%%%%%%%%%%%%%%%%
\begin{frame}{Summation method: array doubling revisited}
  \centerline{On any sequence of $n$ \textsc{Insert} ops on an initially empty array.}

  \pause
  \[
    \begin{array}{ccccccccccc}
      o_i: 	& 1 & 2 & 3 & 4 & 5 & 6 & 7 & 8 & 9 & 10 \\
      c_i:  & 1 & 2 & 3 & 1 & 5 & 1 & 1 & 1 & 8 & 1  \\
    \end{array}
  \]

  \pause
  \begin{displaymath}
	c_i = \left\{ \begin{array}{ll}
	  (i-1)+1 = i & \textrm{if $i - 1$ is an exact power of 2}\\
	  1 & \textrm{o.w.}
	\end{array} \right.
  \end{displaymath}

  \pause
  \[
	\sum_{i=1}^{n} c_i \le n + \sum_{j=0}^{\lceil \lg n \rceil - 1} 2^{j} = n +
	(2^{\lceil \lg n \rceil} - 1) \le n + 2n = 3n
  \]

  \pause
  \[
	\forall i, \hat{c_i} = 3
  \]
\end{frame}
%%%%%%%%%%%%%%%%%%%%
\begin{frame}{Methods for amortized analysis: the accounting method}
  \begin{gather*}
	o_1, o_2, \ldots, o_n \\[5pt]
	c_1, c_2, \ldots, c_n \\[5pt]
	a_1, a_2, \ldots, a_n
  \end{gather*}

  \pause
  \[
    \hat{c_i} = c_i + a_i, a_i >=< 0. 
  \]

  \pause
  \[
	\forall n, \sum_{i=1}^{n} c_i \le \sum_{i=1}^{n} \hat{c_i} \pause \implies \forall n, \sum_{i=1}^{n} a_i \geq 0
  \]

  \pause
  \begin{alertblock}{Key way of thinking:}
	Put the accounting cost on specific objects.
  \end{alertblock}
\end{frame}
%%%%%%%%%%%%%%%%%%%%
\begin{frame}{Accounting method: array doubling revisited}
  \[
    \hat{c_i} = 3 \text{\emph{ vs. }} \hat{c_i} = 2
  \]

  \pause
  \[
	\hat{c_i} = 3 = \underbrace{1}_{\textrm{insert}} +
	\underbrace{1}_{\textrm{move itself}} + \underbrace{1}_{\textrm{help move another}}
  \]

  \pause
  \begin{table}
    \begin{tabular}{c|ccc}
	  & $\hat{c_i}$ & $c_i$ (actual cost) & $a_i$ (accounting cost)
	  \\ \hline
	  \textsc{Insert} (normal) & $3$ & $1$ & $2$\\
	  \textsc{Insert} (expansion) & $3$ & $1 + t$ & $-t + 2$
    \end{tabular}
  \end{table}
\end{frame}
%%%%%%%%%%%%%%%%%%%%
\begin{frame}{Array merging (Problem 4.13): the summation method}
  \centerline{\textsc{Create} ($1$); \textsc{Merge} ($2m$)}

  \begin{columns}
	\column{0.40\textwidth}
	  \pause
	  \begin{align*}
		i & \quad c_i \\
		1 & \quad 1  \\
		2 & \quad 1 + 2 \\
		3 & \quad 1 \\
		4 & \quad 1 + 2 + 4 \\
		5 & \quad 1 \\
		6 & \quad 1 + 2 \\
		7 & \quad 1 \\
		8 & \quad 1 + 2 + 4 \\
		\vdots & \quad \cdots
	  \end{align*}
	\column{0.55\textwidth}
	  \pause
	  \[
		\sum_{i=1}^{n} c_i = \sum_{j=0}^{\lfloor \log n \rfloor} \lfloor \frac{n}{2^j} \rfloor 2^j \le n (\lfloor \log n \rfloor + 1)
	  \]

	  \pause
	  \[
		\forall i, \hat{c_i} = 1 + \lfloor \log n \rfloor
	  \]
  \end{columns}
\end{frame}
%%%%%%%%%%%%%%%%%%%%
\begin{frame}{Array merging (Problem 4.13): the accounting method}
  \[
	\hat{c_i} = 1 + \lfloor \log n \rfloor
  \]

  \pause
  \vspace{0.60cm}
  \centerline{What does it mean?}

  \pause
  \vspace{0.80cm}
  \[
	\forall n, \sum_{i=1}^{n} a_i \geq 0
  \]
\end{frame}
%%%%%%%%%%%%%%%%%%%%
\begin{frame}{Two stacks, one queue (Problem 4.14)}
  \begin{algorithm}[H]
    \caption{Simulating a queue using two stacks $S_1, S_2$.}
    \begin{algorithmic}[]
      \Procedure{Enq}{$x$}
		\State \textsl{Push}($S_1, x$)
      \EndProcedure

	  \Statex

      \Procedure{Deq}{\null}
        \If{$S_2 = \emptyset$}
          \While{$S_1 \neq \emptyset$}
            \State \textsl{Push}($S_2$, \textsl{Pop}($S_1$))
          \EndWhile
        \EndIf
        \textsl{Pop}($S_2$)
      \EndProcedure
    \end{algorithmic}
  \end{algorithm}
\end{frame}
%%%%%%%%%%%%%
\begin{frame}{Two stacks, one queue: the summation method}
  \[
	(\sum_{i = 1}^{n} c_i) / n
  \]

  \pause
  \centerline{The operation sequence is NOT known.}
\end{frame}
%%%%%%%%%%%%%
\begin{frame}{Two stacks, one queue: the accounting method}
  \begin{table}
    \begin{tabular}{ccccc}
	  item: & Push into $S_1$ & Pop from $S_1$ & Push into $S_2$ & Pop from $S_2$ \\
	  & $1$ & $1$ & $1$ & $1$
    \end{tabular}
  \end{table}

  % $\#S_1 = t:$
  % \begin{table}
  %   \begin{tabular}{c|ccc}
  %     & $\hat{c_i}$ & $c_i$ (actual cost) & $a_i$ (accounting cost)
  %     \\
  %     \textsc{Enqueue} & 3 & 1 & 2 \\
  %     \textsc{Dequeue} ($S_2 = \emptyset$) & 1 & 1 & 0 \\
  %     \textsc{Dequeue} ($S_2 \neq \emptyset$) & 1 & 1+2t & -2t
  %   \end{tabular}
  % \end{table}

  \pause
  \begin{align*}
	\hat{c}_{\textsc{Enq}} &= 3 \\
	\hat{c}_{\textsc{Deq}} &= 1 \\
  \end{align*}

  \pause
  \[ 
    \sum_{i=1}^{n} a_i \ge 0 \pause \Longleftarrow \sum_{i=1}^{n} a_i = \#S_1 \times 2
  \]
\end{frame}
%%%%%%%%%%%%%%%%%%%%
\begin{frame}[noframenumbering]
  \fignocaption{width = 0.50\textwidth}{figs/thankyou.jpg}
\end{frame}
%%%%%%%%%%%%%%%%%%%%
\begin{frame}[noframenumbering]
  \begin{center}
  {\Large 写清楚算法原理,不要只写代码} \\[15pt]
  {\Large 写清楚算法原理,不要只写代码} \\[15pt]
  {\Large 写清楚算法原理,不要只写代码}
  \end{center}
\end{frame}
%%%%%%%%%%%%%%%%%%%%
