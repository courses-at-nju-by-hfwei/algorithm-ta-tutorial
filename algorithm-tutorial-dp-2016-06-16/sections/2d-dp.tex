\section{2-D DP}

\subsection{2-D DP (part 1)}
%%%%%%%%%%
\begin{frame}{2-D DP (part 1)}
  \begin{exampleblock}{LCS: longest common subsequence \pno{2.2.7}}
    \begin{itemize}
      \item $X = X_{1} \cdots X_{m}; Y = Y_{1} \cdots Y_{n}$
      \item find (the length of) an LCS of $X$ and $Y$
    \end{itemize}
    \begin{align*}
      X &= \langle A,\textcolor{blue}{B},\textcolor{blue}{C},\textcolor{blue}{B},D,\textcolor{blue}{A},B \rangle  \\
      Y &= \langle \textcolor{blue}{B},D,\textcolor{blue}{C},A,\textcolor{blue}{B},\textcolor{blue}{A} \rangle \\
      Z &= \langle B,C,B,A \rangle
    \end{align*}
  \end{exampleblock}

  \begin{block}{Solution.}
    \begin{itemize}
      \item subproblem: $L[i,j]$: the length of an LCS of $X[1 \cdots i]$ and $Y[1 \cdots j]$
      \item goal: $L[m,n]$
    \end{itemize}
  \end{block}
\end{frame}
%%%%%%%%%%
\begin{frame}{2-D DP (part 1)}
  \begin{block}{Solution.}
    \begin{itemize}
      \item question: Is $X_{i} = Y_{j}$?
      \item recurrence:
	\begin{displaymath}
	  L[i,j] = \left\{ \begin{array}{ll}
	    L[i-1, j-1] + 1 & \textrm{if $X_{i} = Y_{j}$}\\
	    \max \set{L[i-1,j], L[i,j-1]} & \textrm{if $X_{i} \neq Y_{j}$}
	  \end{array} \right.
	\end{displaymath}
      \item<2-> initialization:
	\begin{align*}
	  L[i,0] &= 0, 0 \le i \le m \\
	  L[0,j] &= 0, 0 \le j \le n
	\end{align*}
    \end{itemize}
  \end{block}

  \uncover<3->{
  \begin{center}
    \textcolor{red}{It {\it may be} correct. But I feel quite uncomfortable without a proof.}
  \end{center}
  }
\end{frame}
%%%%%%%%%%
\begin{frame}{2-D DP (part 1)}
  \begin{alertblock}{Counterexample?}
    \[
      L[i,j] = L[i-1,j-1] + 1 \text{ if } X_{i} = Y_{j}
    \]

    \begin{columns}
      \column{0.50\textwidth}
	\begin{align*}
	  X &= \textcolor{blue}{a},\textcolor{blue}{b},c,c,\textcolor{red}{c} \\
	  Y &= \textcolor{blue}{a},\textcolor{blue}{b},\textcolor{red}{c}  \\
	  Z &= \textcolor{blue}{a},\textcolor{blue}{b},\textcolor{red}{c}
	\end{align*}
      \column{0.50\textwidth}
	\begin{align*}
	  X &= \textcolor{blue}{a},\textcolor{blue}{b},\textcolor{blue}{c},c,c \\
	  Y &= \textcolor{blue}{a},\textcolor{blue}{b},\textcolor{blue}{c}  \\
	  Z &= \textcolor{blue}{a},\textcolor{blue}{b},\textcolor{blue}{c}
	\end{align*}
    \end{columns}
  \end{alertblock}
\end{frame}
%%%%%%%%%%
\begin{frame}{2-D DP (part 1)}
  \begin{block}{Correctness proof (I).}
    \begin{theorem}{}
      $L[i,j] = L[i-1,j-1] + 1$ if $X_{i} = Y_{j}$.
    \end{theorem}

    \begin{theorem}{}
      $Z[1 \cdots k]$ with $\textcolor{blue}{Z_{k} \equiv X_{i} \land Z_{k} \equiv Y_{j}}$ \emph{is} an LCS of $X[1 \cdots i]$ and $Y[1 \cdots j]$.
    \end{theorem}
    \begin{proof}
      \begin{enumerate}
	\item $Z_{k} = X_{i} = Y_{j}$ (by contradiction)
	% \item But, $Z_{k} = X_{i} \nRightarrow Z_{k} \equiv X_{i}; Z_{k} = Y_{i} \nRightarrow Z_{k} \equiv Y_{i}$ 
	\item $Z_{k} = X_{i} = Y_{j} \Rightarrow \text{ either } Z_{k} \equiv X_{i} \text{ or } Z_{k} \equiv Y_{j}$ (by contradiction)
	  \begin{enumerate}
	    \item $Z_{k} \equiv X_{i} \land Z_{k} \equiv Y_{j}$
	    \item $Z_{k} \not\equiv X_{i} \land Z_{k} \equiv Y_{j}$
	    \item $Z_{k} \equiv X_{i} \land Z_{k} \not\equiv Y_{j}$
	  \end{enumerate}
      \end{enumerate}
    \end{proof}
  \end{block}
\end{frame}
%%%%%%%%%%
\begin{frame}{2-D DP (part 1)}
  \begin{block}{Correctness proof (II).}
    \begin{theorem}{}
      $L[i,j] = \max \set{L[i-1,j], L[i,j-1]} \text{ if } X_{i} \neq Y_{j}$
    \end{theorem}

    \begin{theorem}{}
      If $X_{i} \neq Y_{j}$, then either $X_{i} \notin \text{LCS}[i,j]$ or $Y_{j} \notin \text{LCS}[i,j]$.
    \end{theorem}
    \begin{proof}
      By contradiction.
    \end{proof}
  \end{block}
\end{frame}
%%%%%%%%%%
\begin{frame}<beamer:0>{2-D DP (part 1)}
  \begin{exampleblock}{LCS with repetition of $X_{i}$ \pno{2.2.8}}
    \begin{enumerate}
      \item repetition of $X_{i}$
      \item $k$-bounded repetition of $X_{i}$
    \end{enumerate}
  \end{exampleblock}

  \begin{block}{Solution.}
    \begin{enumerate}
      \item repetition of $x_{i}$:
	\begin{displaymath}
	  L[i,j] = \left\{ \begin{array}{ll}
	    L[\textcolor{red}{i}, j-1] + 1 & \textrm{if $X_{i} = Y_{j}$}\\
	    \max \set{L[i-1,j], L[i,j-1]} & \textrm{if $X_{i} \neq Y_{j}$}
	  \end{array} \right.
	\end{displaymath}
      \item $k$-bounded repetition of $X_{i}$:

	$X^{(k)} = X_{1}^{(k)} \cdots X_{m}^{(k)}$
    \end{enumerate}
  \end{block}
\end{frame}
%%%%%%%%%%
\begin{frame}<beamer:0>{2-D DP (part 1)}
  \begin{exampleblock}{Shortest common supersequence \pno{2.2.10}}
    \begin{itemize}
      \item $X = \set{x_{1} \cdots x_{m}}; Y = \set{y_{1} \cdots y_{n}}$
      \item to find (the length of) a SCS of $X$ and $Y$
    \end{itemize}
  \end{exampleblock}

  \begin{block}{Solution.}
    \begin{itemize}
      \item subproblem $L[i,j]$: the length of an SCS of $X[1 \cdots i]$ and $Y[1 \cdots j]$
      \item goal: $L[m,n]$
      \item question: is $X_{i} = Y_{j}$
      \item recurrence:
	\begin{displaymath}
	  L[i,j] = \left\{ \begin{array}{ll}
	    L[i-1, j-1] + 1 & \textrm{if $X_{i} = Y_{j}$}\\
	    \max \set{L[i-1,j] + 1, L[i,j-1] + 1} & \textrm{if $X_{i} \neq Y_{j}$}
	  \end{array} \right.
	\end{displaymath}
    \end{itemize}
  \end{block}

  \begin{alertblock}{Remark.}
    $\max(m,n) \le L(m,n) \le m+n$
  \end{alertblock}
\end{frame}
%%%%%%%%%%
\begin{frame}{2-D DP (part 1)}
  \begin{exampleblock}{Edit distance revisited}
    \begin{displaymath}
      \text{ED}[i,j] = \min \left\{ \begin{array}{ll}
	\text{ED}[i-1,j] + 1 &  \\
	\text{ED}[i,j-1] + 1 & \\
	\text{ED}[i-1,j-1] + \text{I}\set{X_{i} = Y_{j}} &
      \end{array} \right.
    \end{displaymath}
  \end{exampleblock}
    
  \uncover<2->{
  \begin{exampleblock}{}
    \begin{displaymath}
      \text{ED}[i,j] = \left\{ \begin{array}{ll}
        \text{ED}[i-1,j-1] & \text{if } X_{i} = Y_{j}  \\
        \min \left\{ \begin{array}{ll}
          \text{ED}[i-1,j] + 1 &  \\
          \text{ED}[i,j-1] + 1 &  \\
          \text{ED}[i-1,j-1] + 1 \\
	\end{array} \right. & \text{if } X_{i} \neq Y_{j}
      \end{array} \right.
    \end{displaymath}
  \end{exampleblock}
  }

  \uncover<3->{
  \begin{theorem}
    If $X_{i} = Y_{j}$, then $\text{ED}[i-1,j-1] \le \text{ED}[i-1,j] + 1$.
  \end{theorem}
  }
\end{frame}
%%%%%%%%%%
\begin{frame}<beamer:0>{2-D DP (part 1)}
  \begin{exampleblock}{Shuffle of strings \pno{2.2.12}}
    \begin{itemize}
      \item $X[1 \cdots m]; Y[1 \cdots n], Z[1 \cdots m+n]$
      \item is $Z$ a shuffle of $X$ and $Y$?
    \end{itemize}
  \end{exampleblock}

  \begin{block}{Solution.}
    \begin{itemize}
      \item $S[i,j]$: Is $Z[1 \cdots i+j]$ a shuffle of $X[1 \cdots i]$ and $Y[1 \cdots j]$?
      \item goal: $S[m,n]$
      \item question: what is the relation among $X_{i}, Y_{j}, \text{ and } Z_{i+j}$?
    \end{itemize}
  \end{block}
\end{frame}
%%%%%%%%%%
\begin{frame}<beamer:0>{2-D DP (part 1)}
  \begin{block}{Solution.}
    \begin{itemize}
      \item recurrence:
	\begin{displaymath}
	  S[i,j] = \left\{ \begin{array}{ll}
	    \text{false} & \textrm{if $Z_{i+j} \neq X_{i} \land Z_{i+j} \neq Y_{j}$}\\
	    S[i-1,j] & \textrm{if $Z_{i+j} = X_{i} \land Z_{i+j} \neq Y_{j}$}\\
	    S[i,j-1] & \textrm{if $Z_{i+j} \neq X_{i} \land Z_{i+j} = Y_{j}$}\\
	    S[i-1,j] \lor S[i,j-1] & \textrm{if $Z_{i+j} = X_{i} = Y_{j}$}
	  \end{array} \right.
	\end{displaymath}
      \item initialization: 
	\begin{align*}
	  S[0,0] &= \text{true} \\
	  S[0,j] &= \left\{ \begin{array}{ll}
	    \text{true} & \text{if } Y = Z  \\
	    \text{false} & \text{if } Y \neq Z
	  \end{array} \right. \\
	  S[i,0] &= \left\{ \begin{array}{ll}
	    \text{true} & \text{if } X = Z  \\
	    \text{false} & \text{if } X \neq Z
	  \end{array} \right.
	\end{align*}
    \end{itemize}
  \end{block}
\end{frame}
%%%%%%%%%%
\subsection{2-D DP (part 2)}

%%%%%%%%%%
\begin{frame}{2-D DP (part 2)}
  \begin{exampleblock}{Longest contiguous substring both forward and backward \pno{2.2.9}}
    \begin{itemize}
      \item string $T[1 \cdots n]$
      \item to find LCS both forward and backward
    \end{itemize}

    \begin{center}
      d\textcolor{blue}{ynam}icprogramming\textcolor{blue}{many}times
    \end{center}
  \end{exampleblock}

  \begin{alertblock}{Trial and error.}
    \begin{itemize}
      \item try subproblem $L[i]$: the length of an LCS in $T[1 \cdots i]$
      \item try subproblem $L[i,j]$: the length of an LCS in $T[i \cdots j]$
    \end{itemize}
  \end{alertblock}
\end{frame}
%%%%%%%%%%
\begin{frame}{2-D DP (part 2)}
  \begin{block}{Solution.}
    \begin{itemize}
      \item $L[i,j]$: the length of an LCS \textcolor{blue}{starting with $T_{i}$ and ending with $T_{j}$}
      \item goal: $\max_{1 \le i \le j \le n} L[i,j]$ (simply $O(n^3)$)
      \item<2-> question: Is $T_{i} = T_{j}$?
      \item<2-> recurrence: 
	\begin{displaymath}
	  L[i,j] = \left\{ \begin{array}{ll}
	    0 & \textrm{if $T_{i} \neq T_{j}$}  \\
	    L[i+1,j-1] + 1 & \textrm{if $T_{i} = T_{j}$}
	  \end{array} \right.
	\end{displaymath}
      \item<3-> initialization: 
	\begin{align*}
	  L[i,i] &= 0, 0 \le i \le n  \\
	  L[i,i+1] &= \left\{ \begin{array}{ll}
	    1 & \text{if } T_{i} = T_{i+1}  \\
	    0 & \text{if } T_{i} \neq T_{i+1}
	    \end{array} \right.
	\end{align*}
    \end{itemize}
  \end{block}
\end{frame}
%%%%%%%%%%
\begin{frame}[fragile]{2-D DP (part 2)}
  \begin{block}{Code: three ways of filling the table.}
    \fignocaption{width = 0.50\textwidth}{fig/three-ways-filling-table.png}

    \begin{center}
        \begin{verbatim}
          for d = 2 to n-1
            for i = 1 to n-d
              j = i + d
              ...
          return max_{1 <= i <= j <= n} L[i,j]
       \end{verbatim}
    \end{center}
  \end{block}
\end{frame}
%%%%%%%%%%
\begin{frame}[fragile]{2-D DP (part 2)}
  \begin{block}{Code: three ways of filling the table.}
    \begin{columns}[t]
      \column{0.50\textwidth}
        \begin{verbatim}
          for i = n-2 to 1
            for j = i+2 to n
              ...
          return ...
        \end{verbatim}
      \column{0.50\textwidth}
        \begin{verbatim}
          for j = 3 to n
            for i = j-2 to 1
              ...
          return ...
       \end{verbatim}
    \end{columns}
  \end{block}
\end{frame}
%%%%%%%%%%
\begin{frame}<beamer:0>{2-D DP (part 2)}
  \begin{exampleblock}{Longest subsequence palindrome \pno{2.2.15 (a)}}
    \begin{itemize}
      \item string $S[1 \cdots n]$
      \item to find (the length of) a longest subsequence palindrome
    \end{itemize}
  \end{exampleblock}

  \pause
  \begin{block}{Solution.}
    \begin{itemize}
      \item subproblem $L[i,j]$: the length of the LSP in $S[i \cdots j]$
      \item goal: $L[1,n]$
      \item question: is $S[i] = S[j]$?
      \item recurrence
    \end{itemize}
  \end{block}
\end{frame}
%%%%%%%%%%
\begin{frame}<beamer:0>{2-D DP (part 2)}
  \begin{exampleblock}{Longest subsequence palindrome \pno{2.2.15 (b)}}
    \begin{itemize}
      \item string $S[1 \cdots n]$
      \item decompose into a sequence of palindromes
    \end{itemize}
  \end{exampleblock}

  \begin{block}{Solution.}
    \begin{itemize}
      \item $\text{Num}[i,j]$: the minimum number of palindromes obtained from $S[i \cdots j]$
      \item subproblem $\text{MinPals}[i]$: the minimum number of palindromes obtained from $S[1 \cdots i]$
      \item goal: $\text{MinPals}[n]$
      \item question: what is the start index of the last palindrome?
      \item recurrence:
    \end{itemize}
  \end{block}
\end{frame}
%%%%%%%%%%
\begin{frame}{2-D DP (part 2)}
  \begin{exampleblock}{String split problem \pno{2.2.16}}
    \begin{itemize}
      \item split a string $S$ into many pieces
      \item cost $|S| = n \Rightarrow n$
      \item given locations of $m$ cuts: $\textcolor{gray}{C_{0}}, C_{1}, \cdots, C_{m}, \textcolor{gray}{C_{m+1}}$
      \item to find the MinCost of splitting $S$ into $m+1$ pieces $S_{0} \cdots S_{m}$
    \end{itemize}
  \end{exampleblock}

  \begin{block}{Solution.}
    \begin{itemize}
      \item subproblem: $\text{MinCost}[i,j]$: the minimum cost of splitting substring $S_{i} \cdots S_{j-1}$ using cuts $C_{i+1} \cdots C_{j-1}$
      \item goal: $\text{MinCost}[0,m+1]$
    \end{itemize}
  \end{block}
\end{frame}
%%%%%%%%%%
\begin{frame}{2-D DP (part 2)}
  \begin{block}{Solution.}
    \begin{itemize}
      \item question: what is the first cut in $C_{i+1} \cdots C_{j-1}$?
      \item recurrence:
	\[
	  \text{MinCost}[i,j] = \min_{i < k < j} \left( \text{MinCost}[i,k] + \text{MinCost}[k,j] + l(S_{i} \cdots S_{j-1}) \right)
	\]
      \item initialization:
	\[
	  \text{MinCost}[i, i+1] = 0
	\]
    \end{itemize}
  \end{block}
\end{frame}
%%%%%%%%%%
