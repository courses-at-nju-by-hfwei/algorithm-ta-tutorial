% file: sections/scc.tex

%%%%%%%%%%%%%%%%%%%%
\begin{frame}{}
  \begin{theorem}[Digraph as DAG (Problem $4.6$)]
    Every digraph is a dag of its SCCs.
  \end{theorem}

  \pause
  \vspace{0.60cm}
  \begin{center}
    \red{\large Two tiered structure of digraphs:} \\[10pt] 

    digraph $\equiv$ a \teal{dag} of SCCs \\[6pt]

    SCC: equivalence class over \teal{reachability}
  \end{center}
\end{frame}
%%%%%%%%%%%%%%%%%%%%
\begin{frame}{}
  \centerline{\red{\fbox{\large \text{digraph} $\equiv$ a \teal{dag} of SCCs}}}

  \vspace{0.50cm}
  \begin{exampleblock}{Kosaraju SCC algorithm, $1978$}
    \begin{quote}
      \begin{center}
	``SCCs can be topo-sorted \\ in \red{decreasing} order of their highest \red{finish} time.''
      \end{center}
    \end{quote}

    % \pause
    % \centerline{The vertice with the highest finish time is in a source SCC.}
  \end{exampleblock}

  \pause
  \vspace{0.60cm}
  \begin{enumerate}[(I)]
    \centering
    \item DFS on $G$; \; DFS/BFS on $G^{T}$
    \item DFS on $G^{T}$;\; DFS/BFS on $G$
  \end{enumerate}
\end{frame}
%%%%%%%%%%%%%%%%%%%%
\begin{frame}{}
  \begin{exampleblock}{Kosaraju SCC algorithm, $1978$ (Problem $4.7$)}
    \begin{center}
      $1$st DFS $\xLongrightarrow{?}$ BFS \\[8pt]
      $2$nd DFS $\xLongrightarrow{?}$ BFS
    \end{center}
  \end{exampleblock}

  \pause
  \vspace{0.50cm}
  \begin{description}
    \centering
    \item[$1$st DFS:] \brown{toposort} between SCCs
    \item[$2$nd DFS:] \brown{reachability} within an SCC
  \end{description}

  \pause
  \vspace{0.50cm}
  \centerline{\red{\fbox{\large digraph $\equiv$ a \teal{dag} of SCCs}}}
\end{frame}
%%%%%%%%%%%%%%%%%%%%
\begin{frame}{}
  \begin{exampleblock}{One-to-all reachability in a digraph (Problem 5.12)}
    \[
      v: v \leadsto^{?} \forall u
    \]
    \[
      \exists?\; v: v \leadsto \forall u
    \]
  \end{exampleblock}

  \pause
  \vspace{0.60cm}
  \centerline{\red{SCC}}

  \[
    \exists!\; \text{source vertex } v \iff v \leadsto \forall u
  \]

  \pause
  \[
    \impliedby: \red{\exists!}\; \blue{\text{source}}
  \]

  \pause
  \centerline{$\implies$: By contradiction.}
  \[
    \exists u: v \not\leadsto u \land \text{in}[u] > 0 \implies \exists \text{ cycle}
  \]
\end{frame}
%%%%%%%%%%%%%%%%%%%%
\begin{frame}{}
  \begin{exampleblock}{Impacts of vertices in a digraph (Problem $4.18$)}
    \[
      \text{impact}(v) = |\set{w \neq v: v \leadsto w}|
    \]

    \begin{itemize}
      \centering
      \item $\argmin_{v} \text{impact}(v)$
      \item $\argmax_{v} \text{impact}(v)$
    \end{itemize}
  \end{exampleblock}

  \pause
  \[
    \argmin_{v} \text{impact}(v) \in \text{\teal{sink} SCC of smallest cardinality}
  \]
  \pause
  \[
    \argmin_{v} \text{impact}(v) \in \text{\teal{source} SCC}
  \]

  \pause
  \vspace{0.30cm}
  \centerline{\red{$Q:$} $\forall v:$ computing $\text{impact}(v)$}
\end{frame}
%%%%%%%%%%%%%%%%%%%%
% \begin{frame}{}
%   \begin{exampleblock}{One-way streets (Problem $4.21$)}
% 	Digraph $G$ for city:
%     \begin{enumerate}
%       \item $\forall u,v: u \leftrightsquigarrow v$
%       \item $s: s \leadsto v \leadsto s$
%     \end{enumerate}
%   \end{exampleblock}
% 
%   \pause
%   \[
% 	(2)\; \set{v \mid s \leadsto v} \text{ is an SCC}
%   \]
% \end{frame}
%%%%%%%%%%%%%%%%%%%%
% \begin{frame}{}
%   \begin{exampleblock}{Connectivity (Problem $4.11$)}
% 	\begin{description}[Example:]
% 	  \item[Prove:] connected undirected graph $G$: 
% 		\[
% 		  \exists v: G \setminus v \text{ is still connected}
% 		\]
% 	  \item[Example:] strongly connected digraph $G$:
% 		\[
% 		  \exists v: G \setminus v \text{ is not strongly connected}
% 		\]
% 	  \item[Example:] digraph $G$ with 2 SCCs:
% 		\[
% 		  (G + e) \text{ is not strongly connected}
% 		\]
% 	\end{description}
%   \end{exampleblock}
% \end{frame}
%%%%%%%%%%%%%%%%%%%%
\begin{frame}{}
  \begin{exampleblock}{2SAT (Problem $4.23$)}
    \[
      I: (x_1 \lor \overline{x_2}) \land (\overline{x_1} \lor \overline{x_3}) \land (x_1 \lor x_2) \land (\overline{x_3} \lor x_4) \land (\overline{x_1} \lor x_4)
    \]
  \end{exampleblock}

  \pause
  \[
    \alpha \lor \beta \equiv \overline{\alpha} \to \beta \equiv \overline{\beta} \to \alpha
  \]

  \pause
  \begin{center}
    Implication graph $G_I$.
  \end{center}

  \pause
  \begin{theorem}[2SAT]
    \[
      \exists \text{ SCC } \exists x: v_x \in \text{SCC} \land v_{\overline{x}} \in \text{SCC} \iff I \text{ is not satisfiable}.
    \]
  \end{theorem}

  \pause
  \begin{alertblock}{Reference:}
    \begin{itemize}
      \item {\it ``A Linear-time Algorithm for Testing the Truth of Certain Quantified Boolean Formulas''} 
	by Bengt Aspvall, Michael Plass, and Robert Tarjan, $1979$.
    \end{itemize}
  \end{alertblock}
\end{frame}
%%%%%%%%%%%%%%%%%%%%
\begin{frame}{}
  % \begin{gather*}
  %   (x_0\lor x_2)\land(x_0\lor\lnot x_3)\land(x_1\lor\lnot x_3)\land(x_1\lor\lnot x_4)\land\\
  %   (x_2\lor\lnot x_4)\land (x_0\lor\lnot x_5)\land (x_1\lor\lnot x_5)\land (x_2\lor\lnot x_5)\land\\
  %   (x_3\lor x_6)\land (x_4\lor x_6)\land (x_5\lor x_6)
  % \end{gather*}
  \begin{columns}
    \column{0.50\textwidth}
      % \fignocaption{width = 0.80\textwidth}{figs/2sat-implication-graph-wiki.png}
      \fignocaption{width = 0.80\textwidth}{figs/2sat-implication-graph.pdf}
    \column{0.50\textwidth}
      \pause
      \fignocaption{width = 0.80\textwidth}{figs/2sat-implication-graph.png}
  \end{columns}
\end{frame}
%%%%%%%%%%%%%%%%%%%%
% \begin{frame}{Odd cycle in digraph}
%   \begin{exampleblock}{Odd cycle in digraph (Additional Problem)}
%     Find an odd cycle in a digraph $G$.
%   \end{exampleblock}
% 
%   \pause
%   \begin{lemma}
%     A digraph $G$ has an odd directed cycle $\iff$ $\exists \text{scc}: $ scc is non-bipartite (when treated undirected).
%   \end{lemma}
% 
%   \pause
%   \begin{alertblock}{Question}
% 	To prove the lemma and design an algorithm.
%   \end{alertblock}
%   % \begin{proof}
%   %   $\Longleftarrow$: undirected $C$; oriented
%   %     \begin{itemize}
%   %   	\item odd directed cycle
%   %   	\item choose a direction $\forall u \to v$: $\text{Len}(v \leadsto u)$ 
%   %     \end{itemize}
%   % \end{proof}
% \end{frame}
%%%%%%%%%%%%%%%%%%%%
