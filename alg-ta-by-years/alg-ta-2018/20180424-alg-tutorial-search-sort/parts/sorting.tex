% file: parts/sorting.tex

%%%%%%%%%%%%%%%
\begin{frame}{}
  \begin{definition}[$K$-sorting (Problem $6.8$)]
    An array $A[1 \cdots n]$ is \red{\it $k$-sorted} if it can be divided into $k$ blocks, 
    each of size $n/k$ (we assume that $n/k \in \mathbb{N}$), 
    such that the elements in each block are larger than the elements 
    in earlier blocks and smaller than elements in later blocks. 
    The elements within each block need \red{\it not} be sorted.
  \end{definition}

  \[
    n = 16,\; k = 4,\; \frac{n}{k} = 4
  \]

  \[
    1,\;2,\;4,\;3;\quad 7,\;6,\;8,\;5;\quad 10,\;11,\;9,\;12;\quad 15,\;13,\;16,\;14
  \]
\end{frame}
%%%%%%%%%%%%%%%

%%%%%%%%%%%%%%%
\begin{frame}{}
  \centerline{\teal{\large $k$-sorted}}

  \pause
  \[
    1\text{-sorted} \pause \to 2\text{-sorted} \pause \to 4\text{-sorted} \pause \to \cdots \to n\text{-sorted}
  \]

  \pause
  \centerline{Quicksort \red{(with median as pivot)} stops after the \red{$\log k$} recursions.}

  \pause
  \vspace{0.50cm}
  \[
    \boxed{\teal{\Theta(n \log k)}}
  \]
\end{frame}
%%%%%%%%%%%%%%%

%%%%%%%%%%%%%%%
\begin{frame}{}
  \[
    \red{\Omega(n \log k)}
  \]

  \pause
  \[
    L = \pause\binom{n}{n/k, \ldots, n/k} \pause = \frac{n!}{\left( (\frac{n}{k})! \right)^{k}}
  \]

  \pause
  \[
    H \ge \pause \log \left(\frac{n!}{\left( (\frac{n}{k})! \right)^{k}} \right)
  \]

  \pause
  \[
    n! \;\red{\sim}\; \sqrt{2 \pi n} \Big(\frac{n}{e}\Big)^{n} \implies \log n! \;\red{\sim}\; n \log n
  \]
\end{frame}
%%%%%%%%%%%%%%%

%%%%%%%%%%%%%%%
% \begin{frame}{}
%   \centerline{Sorting the $k$-sorted array.}
% 
%   \pause
%   \[
% 	O(n \log \frac{n}{k})
%   \]
% 
%   \pause
%   \[
% 	L = ((\frac{n}{k})!)^k
%   \]
% 
%   \pause
%   \[
% 	H \ge \log ((\frac{n}{k})!)^k = \Omega(n \log \frac{n}{k})
%   \]
% \end{frame}
%%%%%%%%%%%%%%%

%%%%%%%%%%%%%%%
\begin{frame}{}
  \centerline{\teal{\large Bolts and Nuts {\small (Problem $6.9$)}}}
  \fignocaption{width = 0.38\textwidth}{figs/bolts-nuts}

  \pause
  \centerline{\blue{\large Quicksort}}

  \pause
  \[
    \red{A(n)} = O(n \log n)
  \]

  \pause
  \begin{alertblock}{In the worst case:}
    \begin{itemize}
      \item ``Matching Nuts and Bolts'' by Alon \emph{et al.}, $\Theta(n \log^4 n)$
      \item ``Matching Nuts and Bolts Optimality'' by Bradford, 1995, $\Theta(n \log n)$ 
    \end{itemize}
  \end{alertblock}
\end{frame}
%%%%%%%%%%%%%%%

%%%%%%%%%%%%%%%
\begin{frame}{}
  \fignocaption{width = 0.38\textwidth}{figs/bolts-nuts}

  \[
    \red{\Omega(n \log n)}
  \]

  \pause
  \[
    \red{3^{H}} \ge L \;\red{\ge n!}\; \pause \implies H \ge \log n! \implies H = \Omega(n \log n)
  \]
\end{frame}
%%%%%%%%%%%%%%%

%%%%%%%%%%%%%%%
\begin{frame}{Repeated elements (Problem 2.12)}
  \[
    R[1 \dots n]
  \]
  \[
    \text{check}(R[i], R[j])
  \]
  \[
    \# > \frac{n}{13}
  \]

  \pause
  \vspace{0.30cm}
  \[
    \# > \frac{n}{k}
  \]

  \begin{center}
    an $O(n \log k)$ algorithm \\
    the lower bound $\Omega(n \log k)$
  \end{center}

  \begin{quote}
    \centering
    {\it ``Finding Repeated Elements''} by Misra \& Gries, 1982
  \end{quote}
\end{frame}
%%%%%%%%%%%%%
