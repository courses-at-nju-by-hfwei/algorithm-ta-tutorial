% file: parts/mathematical-induction.tex

%%%%%%%%%%%%%%%
\begin{frame}{}
  \centerline{\Large \teal{Mathematical Induction}}

  \fignocaption{width = 0.45\textwidth}{figs/domino}
\end{frame}
%%%%%%%%%%%%%%%

%%%%%%%%%%%%%%%
\begin{frame}{}
  \begin{exampleblock}{Horner's rule (Problem $1.5$)}
    \[
      P(x) = a_0 + a_1 x + a_2 x^2 + \cdots + a_{n-1} x^{n-1} + a_n x^n
    \]

    % file: algs/horner-rule.tex

\begin{algorithm}[H]
  % \caption{Horner rule for polynomial evaluation.}
  \begin{algorithmic}[1]
    \Procedure{Horner}{$A[0 \dots n], x$} \Comment{$A: \set{a_0 \dots a_n}$}
      \State $p \gets A[n]$

      \hStatex
      \For{$i \gets n-1 \;\text{\bf downto } 0$}
	\State $p \gets px + A[i]$
      \EndFor

      \hStatex
      \State \Return $p$
    \EndProcedure
  \end{algorithmic}
\end{algorithm}

  \end{exampleblock}

  \pause
  \vspace{0.50cm}
  \centerline{\red{Loop invariant} (after the $k$-th loop):}
  \pause
  \[
    \boxed{\red{\mathcal{I}:}\;\; \teal{p = \sum_{i = n}^{i = n-k} a_i x^{k-(n-i)}}}
  \]
\end{frame}
%%%%%%%%%%%%%%%

%%%%%%%%%%%%%%%
\begin{frame}{}
  \[
    \boxed{\red{\mathcal{I}:}\;\; \teal{p = \sum_{i = n}^{i = n-k} a_i x^{k-(n-i)}}}
  \]

  \vspace{0.50cm}
  \begin{columns}
    \column{0.50\textwidth}
      \fignocaption{width = 0.70\textwidth}{figs/8020-rule}
    \column{0.50\textwidth}
      \pause
      \qquad When you are in an exam:\\[5pt]
      \begin{description}
	\item[$20\%:$] Finding $\mathcal{I}$
	\item[$80\%:$] Proving $\mathcal{I}$ by \red{PMI}
      \end{description}
  \end{columns}
\end{frame}
%%%%%%%%%%%%%%%

%%%%%%%%%%%%%%%
\begin{frame}{}
  \centerline{\red{\large Prove by mathematical induction on the number $k$ of loops.}}

  \pause
  \vspace{0.30cm}
  \begin{description}[<+->]
    \setlength{\itemsep}{6pt}
    \item[Base Case:] $k = 0$.
    \item[Inductive Hypothesis:] $\mathcal{I}$ is valid after the $k$-th ($k \ge 0$) loop.
    \item[Inductive Step:] $\mathcal{I}$ maintains for the $(k+1)$-th loop:
  \end{description}

    \pause
    \vspace{-0.50cm}
    \[
      \boxed{\teal{\left(\sum_{i = n}^{i = n-k} a_i x^{k-(n-i)}\right) \cdot x + A[n-k-1] = \sum_{i = n}^{i = n-(k+1)} a_i x^{(k+1)-(n-i)}}}
    \]

  \pause
  \vspace{0.30cm}
  \centerline{\red{\large Termination}}
  \pause
  \begin{enumerate}[(a)]
    \centering
    \item $i \gets n-1 \;\text{\bf downto } 0$
    \item $k = n \implies p = \sum\limits_{i=0}^{i=n} a_i x^i$
  \end{enumerate}
\end{frame}
%%%%%%%%%%%%%%%

%%%%%%%%%%%%%%%
\begin{frame}{}
  \begin{exampleblock}{Integer Multiplication (Problem $1.6$)}
    \vspace{-0.30cm}
    % file: algs/integer-mult.tex

\begin{algorithm}[H]
  % \caption{Integer Multiplication.}
  \begin{algorithmic}[1]
    \Procedure{Int-Mult}{$y, z$}
      \If{$z = 0$}
        \State \Return 0
      \EndIf

      \State \Return $\Call{Int-Mult}{cy, \lfloor \frac{z}{c} \rfloor} + y (z \mod c)$
    \EndProcedure
  \end{algorithmic}
\end{algorithm}

  \end{exampleblock}
  
  \pause
  \vspace{0.30cm}
  \centerline{\red{\large Prove by mathematical induction on \pause{the non-negative integer $z$.}}}

  \pause
  \vspace{0.20cm}
  \begin{description}[<+->]
    \item[BC:] $z = 0: \quad \textsc{Int-Mult}(y,0) = 0 = y \cdot 0$.
    \item[I.H.:] $\red{< z}\; (z > 0): \quad \textsc{Int-Mult}(y,z) = yz$.
    \item[I.S.:] $=z$.
  \end{description}

  \pause
  \vspace{-0.80cm}
  \begin{align*}
    \textsc{Int-Mult}(y,z) &= \textsc{Int-Mult}(cy, \lfloor \frac{z}{c} \rfloor) + y (z \bmod c) \\
			   &= cy \cdot \lfloor \frac{z}{c} \rfloor + y (z \bmod c) = yz
  \end{align*}
\end{frame}
%%%%%%%%%%%%%%%

%%%%%%%%%%%%%%%
\begin{frame}{}
  \begin{exampleblock}{$2$-tree; Full Binary Tree (Problem $2.5$)}
    \fignocaption{width = 0.40\textwidth}{figs/full-binary-tree}
    \[
      \teal{n_0} = \red{n_2} + 1
    \]
  \end{exampleblock}

  \vspace{0.50cm}
  \centerline{\red{\large Prove by mathematical induction on \pause {\it the size of binary tree.}}}
\end{frame}
%%%%%%%%%%%%%%%

%%%%%%%%%%%%%%%
\begin{frame}{}
  \begin{exampleblock}{$2$-tree; Full Binary Tree (Problem $2.5$)}
    \fignocaption{width = 0.40\textwidth}{figs/full-binary-tree}
  \end{exampleblock}
  
  \pause
  \[
    \teal{n_{0L}} = \red{n_{2L}} + 1
  \]
  \[
    \teal{n_{0R}} = \red{n_{2R}} + 1
  \]

  \pause
  \[
    \teal{n_{0L}} + \teal{n_{0R}} \;\text{\it vs. } \red{n_{2L}} + \red{n_{2R}} + 1
  \]
\end{frame}
%%%%%%%%%%%%%%%
