% parts/analysis.tex

%%%%%%%%%%%%%%%
\begin{frame}{}
  \fignocaption{width = 0.35\textwidth}{figs/AoA}
\end{frame}
%%%%%%%%%%%%%%%

%%%%%%%%%%%%%%%
\begin{frame}{}
  \begin{center}
    {\large \teal{Problem $P$ \qquad Algorithm $A$}} \\[8pt] \pause
    {\large \red{Inputs: $\mathcal{X}_{n}$ of size $n$}}
  \end{center}

  \pause
  \[
    W(n) = \max_{X \in \mathcal{X}_{n}} T(X)
  \]

  \pause
  \[
    B(n) = \min_{X \in \mathcal{X}_{n}} T(X)
  \]

  \pause
  \[
    A(n) = \boxed{\red{\sum_{X \in \mathcal{X}_{n}} T(X) \cdot P(X)}} \pause = \underset{X \in \mathcal{X}_{n}}{\mathbb{E}} [T(X)]
  \]
\end{frame}
%%%%%%%%%%%%%%%

%%%%%%%%%%%%%%%
\begin{frame}{}
  \begin{exampleblock}{Average-case Time Complexity (Problem $1.8$)}
    \[
      r \in [1,n], \; r \in \mathbb{Z}^{+}
    \]
    \vspace{-0.80cm}

    \begin{columns}
      \column{0.40\textwidth}
	\[
	  P\set{r = i} = \left\{\begin{array}{lr}
	    \frac{1}{n}, & 1 \le i \le \frac{n}{4} \\[6pt]
	    \frac{2}{n}, & \frac{n}{4} < i \le \frac{n}{2} \\[6pt]
	    \frac{1}{2n}, & \frac{n}{2} < i \le n
	  \end{array}\right.
	\]
      \column{0.50\textwidth}
	\[
	  T(r) = \left\{\begin{array}{lr}
	    10, & r \le \frac{n}{4} \\[6pt]
	    20, & \frac{n}{4} < r \le \frac{n}{2} \\[6pt]
	    30, & \frac{n}{2} < r \le \frac{3n}{4} \\[6pt]
	    n, & \frac{3n}{4} < r \le n \\[6pt]
	  \end{array}\right.
	\]
	% % file: algs/op-perform.tex

\begin{algorithm}[H]
  % \caption{Performing Operations.}
  \begin{algorithmic}[1]
    \Procedure{Perform}{$r$} \Comment{$r \in [1,n], \; r \in \mathbb{Z}^{+}$}
      \If{$r \le \frac{n}{4}$}
	\State perform $10$ operations
      \ElsIf{$\frac{n}{4} < r \le \frac{n}{2}$}
	\State perform $20$ operations
      \ElsIf{$\frac{n}{2} < r \le \frac{3n}{4}$}
	\State perform $30$ operations
      \Else
	\State perform $n$ operations
      \EndIf
    \EndProcedure
  \end{algorithmic}
\end{algorithm}

    \end{columns}
  \end{exampleblock}

  \pause
  \vspace{-0.30cm}
  \begin{align*}
    A &= \sum_{X \in \mathcal{X}} T(X) \cdot P(X) \\
      &= T(1) P(1) + T(2) P(2) + \cdots + T(n) P(n) \\
      &= \frac{n}{4} \times 10 \times \frac{1}{n} \red{+} \frac{n}{4} \times 20 \times \frac{2}{n} \red{+} 
	 \frac{n}{4} \times 30 \times \frac{1}{2n} \red{+} \frac{n}{4} \times n \times \frac{1}{2n} \\
      % &= \frac{1}{8} n + \frac{65}{4}
      &= \cdots
  \end{align*}
\end{frame}
%%%%%%%%%%%%%%%

%%%%%%%%%%%%%%%
% \begin{frame}{Average-case Analysis of Quicksort}
%   \[
%     A(n) = n-1 + \frac{1}{n} \sum_{i=0}^{i=n-1} (A(i) + A(n-i-1))
%   \]
% 
%   \[
%     A(n) = \underset{X \in \mathcal{X}_{n}}{\mathbb{E}} [T(X)] = \sum_{X \in \mathcal{X}_{n}} T(X) \cdot P(X)
%   \]
% 
%   \pause
%   \begin{align*}
%     A(n) &= \mathbb{E}[T(X)] \\
% 	 &\;\red{= \mathbb{E}[\mathbb{E}[T(X)|I]]} \\
% 	 &= \sum_{i=0}^{i=n-1} P(I = i)\; \mathbb{E}[T(X) \mid I = i] \\
% 	 &= \sum_{i=0}^{i=n-1} \frac{1}{n} [n-1 + A(i) + A(n-i-1)]
%   \end{align*}
% \end{frame}
%%%%%%%%%%%%%%%

%%%%%%%%%%%%%%%
\begin{frame}{}
  \begin{exampleblock}{$3$-element Sorting (Problem $1.1$)}
    \begin{enumerate}[(1)]
      \item Design an algorithm for \red{sorting} $3$ distinct elements.
      \item Worst-case and average-case time complexity.
      \item Worst-case lower bound.
    \end{enumerate}
  \end{exampleblock}

  \pause
  \fignocaption{width = 0.65\textwidth}{figs/3-sorting}

  \pause
  \vspace{-0.30cm}
  \[
    W(3) =\pause 3 \pause\qquad B(3) =\pause 2 \pause\qquad \red{A(3)} = \pause \frac{1}{6} (3+3+2+3+3+2) = \frac{8}{3}
  \]

  \pause
  \vspace{-0.60cm}
  \[
    \text{LB}(3) =\pause 3 \pause \qquad \teal{(\text{LB}(3) \ge \log 3!)}
  \]
\end{frame}
%%%%%%%%%%%%%%%

%%%%%%%%%%%%%%%
\begin{frame}{}
  \begin{exampleblock}{$3$-element Median Seletion (Problem $1.2$)}
    \begin{enumerate}[(1)]
      \item Design an algorithm for \red{selecting the median} of $3$ distinct elements.
      \item Worst-case and average-case time complexity.
      \item Worst-case lower bound.
    \end{enumerate}
  \end{exampleblock}

  \pause
  \fignocaption{width = 0.45\textwidth}{figs/3-median}

  \pause
  \vspace{-0.30cm}
  \[
    W(3) = 3 \qquad B(3) = 2 \qquad A(3) = \frac{8}{3}
  \]

  \pause
  \vspace{-0.30cm}
  \[
    \text{LB}(3) =\pause 3 \pause \qquad \teal{(\text{LB}(3) \ge \frac{3n}{2} - \frac{3}{2})}
  \]
\end{frame}
%%%%%%%%%%%%%%%

%%%%%%%%%%%%%%%
\begin{frame}{}
  \begin{columns}
    \column{0.40\textwidth}
      \fignocaption{width = 0.60\textwidth}{figs/keep-calm-wait}
      \[
	\blue{\text{LB} = 2}
      \]
    \column{0.55\textwidth}
      \pause
      % file: algs/3-median-noncomp.tex

\begin{algorithm}[H]
  % \caption{A median selection algorithm for $3$ distinct elements which is not comparison-based.}
  \begin{algorithmic}[1]
    \Procedure{Median}{$a, b, c$}
      \If{$(a-b)(a-c) < 0$}
        \State \Return $a$
      \EndIf

      \If{$(b-a)(b-c) < 0$}
        \State \Return $b$
      \EndIf

      \State \Return $c$
    \EndProcedure
  \end{algorithmic}
\end{algorithm}

  \end{columns}

  \pause
  \vspace{0.60cm}
  \centerline{\red{\Large Not comparison-based!}}
\end{frame}
%%%%%%%%%%%%%%%

%%%%%%%%%%%%%%%
\begin{frame}{}
  \begin{exampleblock}{Exercise}
    \[
      n = 5
    \]
  \end{exampleblock}

  \pause
  \vspace{0.80cm}
  \begin{alertblock}{Reference}
    {\it ``The Art of Computer Programming, Vol 3: Sorting and Searching (Section 5.3.1)''} by Donald E. Knuth
  \end{alertblock}

  \[
    S(21) = 66
  \]
\end{frame}
%%%%%%%%%%%%%%%

%%%%%%%%%%%%%%%
\begin{frame}{}
  \begin{exampleblock}{Analysis of Bubblesort (Problem $3.2$)}
    \begin{columns}
      \column{0.45\textwidth}
	\begin{enumerate}[(a)]
	  \item \gray{Correctness}
	  \item \gray{$W(n)$} \& \red{$A(n)$}
	  \item Improved version \red{$A'(n):$}
	    \[
	      \!\!\!\! l \triangleq \max_{k} \Big\{\textsc{Swap}(A[k], A[k+1])\Big\}
	    \]
	\end{enumerate}
      \column{0.55\textwidth}
	% file: algs/bubblesort-largest.tex

\begin{algorithm}[H]
  % \caption{Bubblesort.}
  \begin{algorithmic}[1]
    \Procedure{Bubblesort}{$A[1 \cdots n]$}
      \For{$i \gets n \;\text{\bf downto } 2$}
	\For{$j \gets 1 \;\text{\bf to } i-1$}
	  \If{\red{$A[j] > A[j+1]$}}
	    \State $\Call{Swap}{A[j], A[j+1]}$
	  \EndIf
	\EndFor
      \EndFor
    \EndProcedure
  \end{algorithmic}
\end{algorithm}

    \end{columns}
  \end{exampleblock}

  \pause
  \[
    A(n) = \Theta(n^2)
  \]

  \pause
  \vspace{-0.30cm}
  \[
    A'(n) = \pause \frac{1}{2} \Big(n^2 - n \ln n - (\gamma + \ln 2 - 1) n \Big) + O(\sqrt{n}) = \pause \Theta(n^2)
  \]

  \pause
  \vspace{-0.50cm}
  \begin{alertblock}{Reference}
    {\it ``The Art of Computer Programming, Vol 3: Sorting and Searching (Section 5.2.2)''} by Donald E. Knuth
  \end{alertblock}
\end{frame}
%%%%%%%%%%%%%%%

%%%%%%%%%%%%%%%
\begin{frame}{}

  \begin{quote}
    {\large People who analyze algorithms have \red{double happiness}. \\[8pt]

    \uncover<3->{
      First of all they experience the sheer \teal{beauty of elegant mathematical patterns}
      that surround elegant computational procedures. \\[6pt]
    }

    \uncover<2->{
      Then they receive a \teal{practical payoff} when their theories 
      make it possible to get other jobs done more quickly and more economically.} \\[4pt]
    }
    
    \hfill --- Donald E. Knuth (1995)
  \end{quote}

  \vspace{-0.30cm}
  \begin{columns}[c]
    \column{0.50\textwidth}
      \fignocaption{width = 0.55\textwidth}{figs/taocp-box}
    \column{0.50\textwidth}
      \fignocaption{width = 0.35\textwidth}{figs/knuth}
  \end{columns}
\end{frame}
%%%%%%%%%%%%%%%
