% file: sections/unique-mst.tex

%%%%%%%%%%%%
\begin{frame}{}
  \centerline{\teal{\Large Uniqueness of MST}}
\end{frame}
%%%%%%%%%%%%%

%%%%%%%%%%%%
\begin{frame}{}
  \begin{exampleblock}{Uniqueness of MST \pno{$10.18\; (1)$}}
    \centerline{Distinct weights $\implies$ Unique MST.}
  \end{exampleblock}

  \pause
  \vspace{0.50cm}
  \centerline{\teal{By Contradiction.}}

  \pause
  \[
    \exists \text{ MSTs } T_1 \neq T_2
  \]

  \pause
  \[
    \Delta E = \set{e \mid e \in T_1 \setminus T_2 \lor e \in T_2 \setminus T_1}
  \]

  \pause
  \[
    e = \min \Delta E
  \]

  \pause
  \[
    e \in T_1 \setminus T_2 \;\text{\small \it (w.l.o.g)}
  \]
\end{frame}
%%%%%%%%%%%%%

%%%%%%%%%%%%
\begin{frame}{}
  \[
    e \in T_1 \setminus T_2
  \]

  \fignocaption{width = 0.40\textwidth}{figs/mst-unique.pdf}

  \pause
  \vspace{-0.30cm}
  \[
    T_2 + \set{e} \implies C
  \]

  \pause
  \vspace{-0.30cm}
  \[
    \exists (e' \in C) \red{\;\notin\; T_1} \implies e' \in T_{2} \setminus T_{1} \implies e' \in \Delta E \pause \implies w(e') > w(e)
  \]

  \pause
  \vspace{-0.50cm}
  \[
    T' = T_{2} + \set{e} - \set{e'} \implies w(T') < w(T_{2})
  \]
\end{frame}
%%%%%%%%%%%%%

%%%%%%%%%%%%%
\begin{frame}{}
  \begin{exampleblock}{Condition for Uniqueness of MST \pno{$10.18\; (2)$}}
    \centerline{Unique MST $\centernot\implies$ Equal weights.}
  \end{exampleblock}

  \pause
  \fignocaption{width = 0.30\textwidth}{figs/unique-mst-partition.pdf}
\end{frame}
%%%%%%%%%%%%%

%%%%%%%%%%%%%
\begin{frame}{}
  \begin{exampleblock}{Unique MST \pno{$10.21 (3)$}}
    \centerline{Unique MST $\centernot\implies$ Minimum-weight edge across any cut is unique.}
  \end{exampleblock}

  \pause
  \fignocaption{width = 0.30\textwidth}{figs/unique-mst-cut-counterexample.pdf}

  \pause
  \begin{theorem}
    Minimum-weight edge across any cut is unique $\implies$ Unique MST.
  \end{theorem}
\end{frame}
%%%%%%%%%%%%%

%%%%%%%%%%%%%
\begin{frame}{}
  \begin{exampleblock}{Unique MST \pno{$10.21\; (3)$}}
    Unique MST $\centernot\implies$ Maximum-weight edge in any cycle is unique.
  \end{exampleblock}

  \pause
  \fignocaption{width = 0.30\textwidth}{figs/unique-mst-cycle-counterexample.pdf}

  \pause
  \begin{theorem}[Conjecture]
    Maximum-weight edge in any cycle is unique $\implies$ Unique MST.
  \end{theorem}

  \pause
  \fignocaption{width = 0.20\textwidth}{figs/unique-mst-cycle-noncounterexample.pdf}
\end{frame}
%%%%%%%%%%%%

%%%%%%%%%%%%
\begin{frame}{}
  \begin{exampleblock}{Unique MST \pno{$10.21\; (4)$}}
    \centerline{To decide whether a graph has a unique MST.}
  \end{exampleblock}

  \pause
  \vspace{0.80cm}
  \centerline{Ties breaking in Prim's and Kruskal's algorithms}

  \pause
  \vspace{0.50cm}
  \begin{proof}
    \centerline{\teal{Cut property and Cycle property.}}
  \end{proof}
\end{frame}
%%%%%%%%%%%%
