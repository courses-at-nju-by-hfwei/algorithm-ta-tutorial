% file: parts/asymptotic.tex

%%%%%%%%%%%%%%%
\begin{frame}{}
  \centerline{\teal{\large Asymptotics}}
  \vspace{-0.30cm}
  \fignocaption{width = 0.60\textwidth}{figs/asymptotic-notations}

  \pause
  \vspace{-0.30cm}
  \[
    \red{Q: \text{\huge $\theta(f)$}?}
  \]
\end{frame}
%%%%%%%%%%%%%%%

%%%%%%%%%%%%%%%
\begin{frame}{}
  \[
    O(g(n)) = \set{f(n) \mid \red{\exists} c > 0, \exists n_0, \forall n \ge n_0: 0 \le f(n) \le c g(n)}
  \]
  \[
    \Omega(g(n)) = \set{f(n) \mid \red{\exists} c > 0, \exists n_0, \forall n \ge n_0: 0 \le c g(n) \le f(n)}
  \]

  \pause
  \begin{align*}
    \Theta(g(n)) &= \set{f(n) \mid \exists c_1 > 0, \exists c_2 >0, \exists n_0, \forall n \ge n_0: \\ 
		& 0 \le c_1 g(n) \le f(n) \le c_2 g(n)}
  \end{align*}

  \pause
  \[
    o(g(n)) = \set{f(n) \mid \textcolor{red}{\forall c > 0}, \exists n_0, \forall n \ge n_0: 0 \le f(n) \le c g(n)}
  \]
  \[
    \omega(g(n)) = \set{f(n) \mid \textcolor{red}{\forall c > 0}, \exists n_0, \forall n \ge n_0: 0 \le c g(n) \le f(n)}
  \]

  \pause
  \[
    \teal{f(n) \sim g(n) \iff \lim_{n \to \infty} \frac{f(n)}{g(n)} = 1}
  \]
\end{frame}
%%%%%%%%%%%%%%%

%%%%%%%%%%%%%%%
\begin{frame}{}
  \begin{exampleblock}{Asymptotics (Problem $2.6\; (4)$)}
    \[
      f(n) = \Theta(g(n)) \iff f(n) = O(g(n)) \land f(n) = \Omega(g(n))
    \]
  \end{exampleblock}

  % \pause
  % \begin{exampleblock}{Asymptotics (Problem $2.6 (5)$)}
  %   \begin{align*}
  %     f(n) = O(g(n)) &\iff g(n) = \Omega(f(n)) \\
  %     f(n) = o(g(n)) &\iff g(n) = \omega(f(n))
  %   \end{align*}
  % \end{exampleblock}

  \pause
  \vspace{0.50cm}
  \begin{exampleblock}{Asymptotics (Problem $2.6\; (6)$)}
    \[
      \Theta(g(n)) \cap o(g(n)) = \emptyset
    \]
  \end{exampleblock}

  \pause
  \vspace{0.50cm}
  \begin{alertblock}{$\red{Q:}\; f(n) = O(g(n)) \lor g(n) = \Omega(f(n)) ?$}
    \pause
    \[
      f(n) = n, \quad g(n) = n^{1 + \sin n}
    \]
  \end{alertblock}

  \pause
  \vspace{0.20cm}
  \begin{alertblock}{Reference:}
    {\it ``Big Omicron and Big Omega and Big Theta''} by Donald E. Knuth, 1976.
  \end{alertblock}
\end{frame}
%%%%%%%%%%%%%%%

%%%%%%%%%%%%%%%
\begin{frame}{}
  \begin{exampleblock}{Asymptotics (Problem $2.7\; (2)$)}
    \[
      (\log n)^{2} \;\;\text{\it vs. }\; \sqrt{n}
    \]
  \end{exampleblock}

  \pause
  \[
    (\log n)^{c_1} = O(n^{c_2}) \quad c_1, c_2 > 0
  \]
\end{frame}
%%%%%%%%%%%%%%%

%%%%%%%%%%%%%%%
\begin{frame}{}
  \begin{exampleblock}{Asymptotics (Problem $2.10$)}
    \[
      \log(n!) = \Theta(n \log n)
    \]
  \end{exampleblock}

  \pause
  \vspace{0.30cm}
  \begin{alertblock}{Stirling Formula (by {\it James Stirling}):}
    \begin{columns}
      \column{0.50\textwidth}
	\[
	  n! = \Theta\Big(\sqrt{2 \pi n} \Big(\frac{n}{e}\Big)^{n}\Big)
	\]
      \column{0.50\textwidth}
	\fignocaption{width = 0.30\textwidth}{figs/stirling-formula-wiki-qrcode}
    \end{columns}
  \end{alertblock}

  \pause
  \vspace{0.30cm}
  \[
    \log (n!) = \log 1 + \log 2 + \cdots + \log n
  \]

  \pause
  \[
    \log (n!) \le n \log n \pause \qquad \log (n!) \ge \frac{n}{2} \log \frac{n}{2}
  \]
\end{frame}
%%%%%%%%%%%%%%%
