%% LaTeX Beamer presentation template (requires beamer package)
%% see http://bitbucket.org/rivanvx/beamer/wiki/Home
%% idea contributed by H. Turgut Uyar
%% template based on a template by Till Tantau
%% this template is still evolving - it might differ in future releases!

\documentclass{beamer}
\mode<presentation>
{
	\usetheme{Singapore}
	\setbeamertemplate{footline}[frame number]
	\usefonttheme{serif}
	
	\setbeamertemplate{navigation symbols}{}
    \setbeamertemplate{caption}[numbered]
% 	\setbeamercovered{transparent}
}

\usepackage{graphics}

\usepackage{mathtools}
\DeclarePairedDelimiter\ceil{\lceil}{\rceil}
\DeclarePairedDelimiter\floor{\lfloor}{\rfloor}

\title{Tutorial}

%\subtitle{}

%\author{Hengfeng Wei}

% \institute[Universities of]
% {
% 	Institute of Computer Software, NJU
% }

\date{\today}


% If you have a file called "university-logo-filename.xxx", where xxx
% is a graphic format that can be processed by latex or pdflatex,
% resp., then you can add a logo as follows:

% \pgfdeclareimage[height=0.5cm]{university-logo}{university-logo-filename}
% \logo{\pgfuseimage{university-logo}}



% Delete this, if you do not want the table of contents to pop up at
% the beginning of each subsection:
\AtBeginSubsection[]
{
	\begin{frame}<beamer>
		\frametitle{Tutorial}
		\tableofcontents[currentsection]
	\end{frame}
}

\AtBeginSection[]
{
	\begin{frame}<beamer>
		\frametitle{Tutorial}
		\tableofcontents[currentsection]
	\end{frame}
}
% If you wish to uncover everything in a step-wise fashion, uncomment
% the following command:

%\beamerdefaultoverlayspecification{<+->}

\begin{document}

\begin{frame}
	\titlepage
\end{frame}

\begin{frame}
	\frametitle{Tutorial}
	\tableofcontents
% You might wish to add the option [pausesections]
\end{frame}

%%%%%%%%%%%%%%%
\section{Amortized Analysis}

\begin{frame}
  \frametitle{Basic}
  \begin{quote}
    Amortized analysis is a strategy for analyzing \\ 
    \textcolor{blue}{a sequence of operations} to show that \\
    \textcolor{blue}{the average cost per operation} is small, \\ 
    even though a \textcolor{blue}{single operation within the sequence might be
    expensive}.
  \end{quote}
  
  \vspace{0.30cm}
  Key points:
  \begin{itemize}
    \setlength{\itemsep}{3pt}
    \item $\neq$ average-case analysis; \textcolor{red}{no probability here}\\
    (\textsc{Quick-Sort}, \textsc{HashTable})
    \item worst-case analysis; upper-bound
    \item on operation sequence; not on separate operations
    \item cheap ops (often) vs. expensive ops (occasionally)
  \end{itemize}
\end{frame}

\begin{frame}
  \frametitle{Basic}
  
  Methods:
  \begin{itemize}
    \setlength{\itemsep}{3pt}
    \item summation method
    \begin{itemize}
      \item the op sequence is known and easy to analyze
      \item sum and then average
    \end{itemize}
    \item accounting method
    \begin{itemize}
      \item impose an extra charge on inexpensive ops and use it to pay
      for expensive ops later on
    \end{itemize}
  \end{itemize}
\end{frame}

\begin{frame}
  \frametitle{Example for Summation Method}
  
  \begin{example}[Binary Counter]
    \begin{itemize}
      \item start from 0
	  \item \textsc{Increment}
	  \item measure: \emph{bit flip}
      \item cost sequence: $1,2,1,3,1,2,1,4,1,2,1,3,1,2,1,5,\ldots$
      \item think globally about each bit: 
        \[
          \sum_{i=0}^{\floor{\log n}} \floor{\frac{n}{2^i}} < n
          \sum_{i=0}^{\infty} \floor{\frac{1}{2^i}} = 2n
        \]
      \item average cost per op: $2$ 
    \end{itemize}
  \end{example}
\end{frame}

\begin{frame}
  \frametitle{Accounting Method}
  
  Accounting method: \\ 
  \emph{amortized cost = actual cost + accounting cost}
  \[
    \hat{c_i} = c_i + a_i, a_i >=< 0. 
  \]
  \[ 
    \forall n, \sum_{i=1}^{n} \hat{c_i} = \sum_{i=1}^{n} c_i + \sum_{i=1}^{n}
    a_i
  \]
  \begin{itemize}
    \item upper bound: the \emph{total amortized cost} of a sequence of ops must
    be an upper bound on the \emph{total actual cost} of the sequence
        \[ \forall n, \sum_{i=1}^{n} \hat{c_i} \geq \sum_{i=1}^{n} c_i
          \Rightarrow \textcolor{red}{\forall n, \sum_{i=1}^{n} a_i \geq 0} 
        \]
	\item put the accounting cost on specific objects 
  \end{itemize}
\end{frame}

\begin{frame}
  \frametitle{Example for Accounting Method}
  
  \begin{example}[Binary Counter]
    To show that $\hat{c_i} = 2$:
    \begin{itemize}
      \item low-level bit: $0 \to 1 (c=2 = 1:set+1:accouting)$
      \item put the accounting (1) on the bit 1; number of 1s = sum of
      accounting $\geq 0$
      \item $1 \to 0 (c=0)$
      \item \textsc{Increment} $\hat{c_i} = 2: (0 \to 1) + (1 \to 0)$; \emph{at
      most $0 \to 1$}
    \end{itemize}
  \end{example}
  
  \begin{block}{Remarks}
    \begin{itemize}
      \item \emph{you cannot say:} cost per operation = 2
      \item \emph{you should say:} average cost per operation over any operation
      sequence $\leq$ 2
      \item constant vs. variable
      \item \textsc{Decrement}? how to support \textsc{Decrement} in $O(1)$?
    \end{itemize}
  \end{block}
\end{frame}

\begin{frame}
  \frametitle{Table Expansion}
  
  \begin{example}[Table Expansion]
    \begin{itemize}
      \item ``double when it is full''
      \item \textsc{Table-Insert}
      \item measure: \emph{elementary insert}
    \end{itemize}
    \begin{itemize}
      \item Summation Method: \\
        \emph{a sequence of $n$ \textsc{Table-Insert}}
        \begin{itemize}
          \item \begin{displaymath}
					c_i = \left\{ \begin{array}{ll}
					i & \textrm{if $i-1$ is an exact power of 2}\\
					l & \textrm{o.w.}
					\end{array} \right.
				\end{displaymath}
		  \item \[
				  \sum_{i=1}^{n} c_i \leq n + \sum_{j=0}^{\floor{\lg n}} 2^j < n + 2n = 3n.
		  		\]
        \end{itemize}
    \end{itemize}
  \end{example}
\end{frame}

\begin{frame}
  \frametitle{Table Expansion}
  
  \begin{example}[Table Expansion]
    \begin{itemize}
      \item ``double when it is full''
      \item \textsc{Table-Insert}
      \item measure: \emph{elementary insert}
    \end{itemize}
    \begin{itemize}
      \item Accounting Method: \\
        \emph{why $\hat{c_i} = 3$? why not 2?}
        \begin{itemize}
          \setlength{\itemsep}{3pt}
          \item check $\hat{c_i} = 2 = 1 (insert) + 1 (move)$
		  \item problem: $2 \Rightarrow 4 (\textrm{after expansion}, \sum a_i = 0)
		  \Rightarrow 8$
		  \item so, $\hat{c_i} = 3 = 1 (insert) + 1 (\textrm{move itself}) + 1
		  (\textrm{give a hand})$
		  \item Meta-method: \textcolor{red}{trial and error} 
        \end{itemize}
    \end{itemize}
  \end{example}
\end{frame}

\begin{frame}
  \frametitle{Insertion Cost (4.2.7)}
  
  \begin{example}[Insertion Cost (4.2.7)]
    \begin{itemize}
      \item \textsc{Insert}
      \item measure: create + merge ($2m$)
    \end{itemize}
    Summation Method: 

	  \[
		\sum_{i=1}^{n} c_i = \sum_{j=0}^{\lfloor \log n \rfloor} \lfloor \frac{n}{2^j} \rfloor 2^j \le n (\lfloor \log n \rfloor + 1)
	  \]
    
    Accounting Method: 
      \begin{itemize}
        \item $\hat{c_i} = 1 + \lfloor \log n \rfloor$
        \item consider each inserted element
        \item to check $\sum a_i = \sum_{i} (\lg n - D_{a_i}) \geq 0$
      \end{itemize}
  \end{example}
\end{frame}

\begin{frame}
  \frametitle{Union Find}
  
  \begin{example}[Union Find]
    \begin{itemize}
      \item \textsc{Make-Set}
      \item \textsc{wUnion}
      \item \textsc{cFind}: beautiful code ($P_{289}$, $P_{508}$)
      \item $\hat{c}$ \emph{are different}
      \item worst-case
    \end{itemize} 
  \end{example}
\end{frame}
%%%%%%%%%%%%%%%
\section{Adversary Argument}

\begin{frame}
  \frametitle{Basic}
  
  \begin{itemize}
    \item lower bound
      \[ T(A) = \max_{I} T(A,I) \]
      \[ T(P) = \min_{A} \max_{I} T(A,I) \]
    \item two directions: \uppercase{Sorting}
		\[ n! \Rightarrow n^2 (cards) \Rightarrow n \lg n (\textrm{John von
		Neumann}@1948) \Rightarrow ??? \] 
		\[ n \Rightarrow n \lg n \]
    \item adversary argument
      \begin{itemize}
        \setlength{\itemsep}{3pt}
        \item alg vs. adversary
        \item \emph{alg says:} I have a dream
        \item \emph{adversary says:} pay something \textcolor{red}{necessary}  
        \item adversary strategy: also an alg; \textcolor{red}{How to construct
        it?}
        \item finding min and max; finding the second largest; \\ 
        finding median; horse racing; matrix search
      \end{itemize}
  \end{itemize}
\end{frame}

\begin{frame}
  \frametitle{Finding the Median}
  
  \begin{example}[Finding the Median]
    \begin{itemize}
      \item decision tree
		\begin{itemize}
		  \item tree vs. alg
		  \item path vs. execution on input
		  \item internal node vs. comparison
		  \item leaf vs. output
		\end{itemize}
      \item $\forall$ leaf, $x^{\ast}$: every other keys are comparable to
      $x^{\ast}$. \textcolor{red}{why?}
      \item critical comparison (first): $\forall y, \exists y \textrm{ vs. } z:
      y > z \geq x^{\ast} \textrm{ or } y < z \leq x^{\ast}$
		\begin{itemize}
		  \item path, internal node
		  \item \emph{alg:} how do I know?
		  \item \emph{adversary:} no, you don't know. It is my \emph{private
		  classification}.
		  \item \emph{alg:} so you can cheat!
		  \item \emph{adversary:} no, look at this path; critical, critical,
		  non-critical $\ldots$ (example)
		\end{itemize}
    \end{itemize}
  \end{example}
\end{frame}

\begin{frame}
  \frametitle{Finding the Median}
  
  \begin{example}[Finding the Median]
    Adversary: \emph{to enfore critical comparisons + non-critical operations as
    many as possible}
    \begin{itemize}
      \item critical: $n-1$
      \item non-critical:
        \begin{itemize}
          \item $L;N;S$
          \item \textsc{Compare}$(x,y)$ until $|L| = \frac{n-1}{2}$ or $|R| =
          \frac{n-1}{2}$ 
          $$N,N;N,S; S,N;N,L; L,N;$$
          $$L,S; S,L;$$ 
          $$L,L; S,S \textrm{(\emph{maybe critical}; however no new } L, S)$$
        \end{itemize}
      \item \emph{alg:} do $|L| = \frac{n-1}{2}$ at least; via \textsc{Compare}
      \item \emph{adversary:} $1 L$ per compare at most; all non-critical
      \item \emph{alg:} at least $\frac{n-1}{2}$ non-critical comparisons
    \end{itemize}
  \end{example}
\end{frame}

\begin{frame}
  \frametitle{Horse Racing}
  
  \begin{example}[Horse Racing]
    \begin{itemize}
      \item 25;5;3
      \item 7 rounds
      \item adversary argument
        \begin{itemize}
          \item $< 5$: \textcolor{red}{why?}
          \item $= 5$: which is the fastest?
          \item $= 6$: to know the fastest, you must run the five first ranked
          \textcolor{red}{why?}
          \item $= 6$: which is the second? ($a_2, b_1$)
        \end{itemize}
    \end{itemize}
  \end{example}
\end{frame}

\begin{frame}
  \frametitle{Matrix Search}
  
  \begin{example}[Matrix Search ($P_{246}$, 5.24)]
    \begin{itemize}
      \item $M[n][n]$; row; column
      \item $n^2 \Rightarrow T(n) = 3T(\frac{n}{2}) + O(1) = n^{\lg 3}
      \Rightarrow 2n-1$
      \item $2n-1:$ check the below-left corner element
      \item worst case: check the two diagonals $i+j=n-1,i+j=n$
      \item adversary \textsc{Compare$(x, M[i,j])$}:
      \[
        i + j \leq n-1 : x > M[i,j]; i + j > n-1 : x < M[i,j]
      \] 
      \item \emph{alg:} at least check every element in two diagonals 
      \item \emph{adversary:} eliminates 1 at most per comparison
      \item \emph{alg:} at least $2n-1$ comparisons 
    \end{itemize}
  \end{example}
\end{frame}

%%%%%%%%%%%%%%%%%%
\section{Other Problems}

\begin{frame}
  \frametitle{Other Problems}
  
  \begin{itemize}
    \setlength{\itemsep}{5pt}
    \item 4.2.3: sorted, distinct; $O(\lg n)$; $a_i = i$; binary search; $T(n)
    = T(\frac{n}{2}) + O(1)$
    \item 4.2.5: search for two numbers; $x = a+b$; 
    
    \textcolor{red}{extension: sorted}; 
    
    Ex: 1,2,3,5,8,9;2,3,4,6,10,12 \textcolor{red}{[proof or counterexample]}
    \item 4.2.6: closed address hashing
    	\[
    	  Q_k = (\frac{1}{n})^k (1-\frac{1}{n})^{n-k} \binom{n}{k}
    	\]
    \item 4.1.4,4.1.5,4.1.6: binomial tree
    \item 2.1.2: $H \leq N -1$
  \end{itemize}
\end{frame}
%%%%%%%%%%
% \section*{Summary}
% \begin{frame}
%   \frametitle{}
%   
% %   \begin{figure}[htp]
% % 	\centering
% % 	  \includegraphics[scale = 0.40]{figure/thankyou.png}
% %   \end{figure}
% \end{frame}
%%%%%%%%%%
\end{document}
