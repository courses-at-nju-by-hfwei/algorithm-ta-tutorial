%% LaTeX Beamer presentation template (requires beamer package)
%% see http://bitbucket.org/rivanvx/beamer/wiki/Home
%% idea contributed by H. Turgut Uyar
%% template based on a template by Till Tantau
%% this template is still evolving - it might differ in future releases!

\documentclass{beamer}
\mode<presentation>
{
	\usetheme{Singapore}
	\setbeamertemplate{footline}[frame number]
	\usefonttheme{serif}
	
	\setbeamertemplate{navigation symbols}{}
    \setbeamertemplate{caption}[numbered]
% 	\setbeamercovered{transparent}
}

\usepackage{cancel}
\usepackage{amsmath}
\usepackage{graphics}

\title{Tutorial for Mid-term Exam}

% \author{Hengfeng Wei}

\date{November 12, 2013}

% Delete this, if you do not want the table of contents to pop up at
% the beginning of each subsection:
\AtBeginSubsection[]
{
	\begin{frame}<beamer>
		\frametitle{Outline}
		\tableofcontents[currentsection]
	\end{frame}
}

\AtBeginSection[]
{
	\begin{frame}<beamer>
		\frametitle{Outline}
		\tableofcontents[currentsection]
	\end{frame}
}
% If you wish to uncover everything in a step-wise fashion, uncomment
% the following command:

%\beamerdefaultoverlayspecification{<+->}

\begin{document}

\begin{frame}
	\titlepage
\end{frame}

\begin{frame}
	\frametitle{Outline}
	\tableofcontents
% You might wish to add the option [pausesections]
\end{frame}

%%%%%%%%%%%%%%%%%%%
\section{A Warm-up Function}

\begin{frame}
  \frametitle{Growth of Function}
  
  \begin{Problem}[Growth of Function $\lbrack \textrm{\textbf{Test 1}}
  \rbrack$] 
    $S(n) = 1^c + 2^c + 3^c + \ldots + n^c, c \in \mathbb{Z^{+}}.$
    \begin{itemize}
      \item $S(n) \in O(n^{c+1})$;
      \item $S(n) \in \Omega(n^{c+1})$.
    \end{itemize}
  \end{Problem}
  
  \vspace{0.30cm}
  To prove: $\exists$ constants $a > 0, n_0 > 0$ such that $S(n) \geq a
  n^{c+1}, \forall n \geq n_0$.
  
  \[
   S(n) \geq \underbrace{(\frac{n}{2})^{c} + (\frac{n}{2})^{c} + \ldots +
    (\frac{n}{2})^{c}}_{\# = \frac{n}{2}}
    = (\frac{n}{2})^{c} \cdot \frac{n}{2}
    = \underbrace{\frac{1}{2^{c+1}}}_{a}n^{c+1}.
  \]
  
  \begin{block}{Remark:}
    \begin{itemize}
      \item inductive on \emph{constant} $c$
%       \item $\int_{0}^{1} (1-x)^3 dx; (-x^3 + 3x^2 - 3x + 1); (x^4 - 4x^3 +
%       6x^2 - 4x + 1)$
    \end{itemize}
  \end{block}  
\end{frame}
%%%%%%%%%%%%%%%%%%%
\section{Recurrences}

\begin{frame}
  \frametitle{List of Recurrences}
  
  \begin{itemize}
    \setlength{\itemsep}{8pt}
    \item $T(n) = 2T(n-1) + O(1)$ \hfill $T(n) = O(2^n)$ Hanio tower
    \item $T(n) = 7T(\frac{n}{2}) + O(n^2)$ \\ \hfill $T(n) = O(n^{2.81})$
    Strassen matrix multiplication 
    \item $T(n) = 3T(\frac{n}{2}) + O(n)$ \\ \hfill $T(n) = O(n^{1.59})$ Gauss
    integer multiplication
    \item $T(n) = 2T(\frac{n}{2}) + O(n)$ \\ \hfill $T(n) = O(n \lg n)$ merge
    sort, median-quicksort
    \item $T(k) = 2T(\frac{k}{2}) + O(nk)$, 
    $T(n,k) = 2T(\frac{n}{2}, \frac{k}{2}) + O(n)$ \hfill in exam
    \item $T(n) = T(\frac{n}{5}) + T(\frac{7n}{10}) + O(n)$ \hfill linear-time
    selection
    \item $T(n) = T(\frac{n}{2}) + O(1)$ \hfill $T(n) = O(\lg n)$ binary
    search
    \item $T(n) = 2T(\frac{n}{4}) + O(1)$ \hfill $T(n) = O(\sqrt{n})$ VLSI
    layout
  \end{itemize} 
\end{frame}

\begin{frame}
  \frametitle{Solving Recurrences}
  
  \begin{Problem}[Solving Recurrences $\lbrack \textrm{\textbf{Test 2}}
  \rbrack$]
    \begin{itemize}
      \item $T(n) = 5T(\frac{n}{2}) + \Theta(n)$ \hfill
        \textcolor{blue}{$T(n) = \Theta(n^{\lg 5})$}
      \item $T(n) = 9T(\frac{n}{3}) + \Theta(n^2)$ \hfill
        \textcolor{blue}{$T(n) = \Theta(n^2 \lg n)$}
      \item $T(n) = 2T(n-1) + O(1)$ \hfill
        \textcolor{blue}{$T(n) = O(2^n)$}
    \end{itemize}
  \end{Problem}
  
  \begin{block}{Remark:}
    \begin{itemize}
      \item $T(n) = 2T(n-1) + O(1)$: unfolding; recursion tree; \\
      $(1 + 2 + \ldots + 2^{n-1}) O(1) = (2^{n}-1) O(1)$; \\ 
      The Tower of Hanio.
      \item which to choose?
    \end{itemize}
  \end{block}
\end{frame}

\begin{frame}
  \frametitle{Merge Sort}
  
  \begin{Problem}[Merge Sort $\lbrack \textrm{\textbf{Test 3}} \rbrack$]
    $k$ sorted arrays, each with $n$ elements; merge-sort them.
  \end{Problem}
  
  \begin{itemize}
    \item one by one: $2n + 3n + \ldots + kn = \frac{k^2 + k - 2}{2}n$
    \item divide and conquer: $T(k) = 2T(\frac{k}{2}) + O(nk) = O(n \cdot k
    \lg k)$; \\
    unfolding the recursion; dfs (post-order); \\
    layer by layer; recursion tree. 
  \end{itemize}
\end{frame}

\begin{frame}
  \frametitle{$k$-Sort}
  
  \begin{Problem}[$k$-Sort $\lbrack \textrm{\textbf{Test 6}} \rbrack$]
    $A[1 \ldots n]$, $k$ blocks, each of size $\frac{n}{k}$:
    \begin{itemize}
      \item sorted within each block
      \item \emph{fuzzy} sorted among blocks
    \end{itemize}
    E.g., $1\; 2\; 4\; 3; 7\; 6\; 8\; 5; 10\; 11\; 9\; 12; 15\; 13\; 16\; 14$
  \end{Problem}
  
  \vspace{0.40cm}
  It is a \emph{unfinished} median-quicksort.
  \begin{itemize}
    \item recursion of ``median + partition''
    \item $T(n,k) = 2T(\frac{n}{2}, \frac{k}{2}) + O(n)$
    \item recursion tree; $\lg k$ layers
  \end{itemize}
\end{frame}

\begin{frame}
  \frametitle{VLSI Layout (1)}
  
  \begin{Problem}[VLSI Layout]
    Embed a complete binary tree with $n$ leaves into a grid with minimum area.
    \begin{itemize}
      \item VLSI: Very Large Scale Integration
      \item complete binary tree circuit: $\#layer = 3,5,7,\ldots$
      \item $n$ leaves (\textcolor{red}{why only leaves?})
      \item grid; vertex + edge (no crossing)
      \item area
      \item take 5-layer complete binary tree as an example
    \end{itemize}
  \end{Problem}
\end{frame}

\begin{frame}
  \frametitle{VLSI Layout (2)}
  
  \begin{itemize}
    \setlength{\itemsep}{0.50cm}
    \item Na\"{\i}ve embedding
     \[ H(n) = H(\frac{n}{2}) + \Theta(1) = \Theta(\lg n) \]
     \[ W(n) = 2W(\frac{n}{2}) + \Theta(1) = \Theta(n) \]
     \[ A(n) = \Theta(n \lg n) \]
    \item Smart (H-Layout) embedding
    \[ \Box \times \Box = n? \; 1 \times n;\; \frac{n}{\lg n} \times \lg n;\;
    \textcolor{blue}{\sqrt{n} \times \sqrt{n}}
    \]
    Goal: $H(n) = \Theta(\sqrt{n}); W(n) = \Theta(\sqrt{n}); A(n) = \Theta(n).$
    \[ 
      H(n) = \Box H(\frac{n}{\Box}) + O(\Box); H(n) = 2 H(\frac{n}{4}) +
      O(n^{\frac{1}{2} - \epsilon})
    \]
    \[ H(n) = 2H(\frac{n}{4}) + \Theta(1) \]
    \centering \textcolor{blue}{Here it is: H-Layout}
  \end{itemize}
\end{frame}

\begin{frame}
  \frametitle{Median of Two Sorted Arrays (1)}
  
  \begin{Problem}[Median of Two Sorted Arrays $\lbrack {\bf P_{245}, 5.18}
  \rbrack$]
    $A[1 \ldots n], B[1 \ldots n]$; sorted, ascending order; distinct.
    
    find the median of the combined set of $A$ and $B$ ($O(\lg n)$).
  \end{Problem}
  
  \[ T(n) = T(\frac{n}{2}) + \Theta(1) \]
  \[A = 2,4,6,8,10 = \fbox{$A_1: (n-1)/2$}\; \fbox{$M_A: 1$}\; \fbox{$A_2:
  (n-1)/2$} \] 
  \[B = 1,3,5,7,9 = \fbox{$B_1: (n-1)/2$}\; \fbox{$M_B: 1$}\; \fbox{$B_2:
  (n-1)/2$} \] 
  \[
    \fbox{$n$ \textrm{ is odd}} C = \fbox{$C_1: (n-1)$}\; \fbox{$M_C: 1$}\;
    \fbox{$C_2: (n-1)$}
  \]
  \[
    M_A > M_B \Rightarrow \cancel{M_A + A_2; B_1} \Rightarrow (A_1 + M_1, M_2 +
    B_2)
  \]
  \textcolor{red}{why not $(A_1, M_2 + B_2)$?}
  and not finished yet $\ldots$ (\textcolor{red}{why?})
\end{frame}

\begin{frame}
  \frametitle{Median of Two Sorted Arrays (2)}
  
  \textcolor{red}{To prove:} The median $M_C'$ of the subproblem $(A_1 + M_1,
  M_2 + B_2)$ is also the median $M_C$ of the original problem $(A,B)$.
  
  \vspace{0.50cm}  
  \textcolor{blue}{Proof: \textcolor{red}{why?}} 
  \[ M_C \in [M_B, M_A) \]
  \[ 
    M_C' \in [M_B, M_A)\; \fbox{$\frac{n+1}{2}-1$}\; \fbox{$M_C': 1$}\;
    \fbox{$\frac{n+1}{2}-1$}
  \]
  
  \begin{block}{Extensions:}
    \begin{itemize}
      \item $A[1 \ldots n], B[1 \ldots n]$; not sorted;
      \item $A[1 \ldots n], B[1 \ldots n], C[1 \ldots n]$; sorted; median
      \item $A[1 \ldots n], B[1 \ldots n], \ldots$; sorted; median
      \item $A[1 \ldots n], B[1 \ldots n], \ldots$; sorted; $k$-th smallest 
    \end{itemize}
  \end{block}
\end{frame}

\begin{frame}
  \frametitle{Two Stacks, One Queue (1)}
  
  \begin{Problem}[Two Stacks, One Queue $\lbrack \textrm{\textbf{Test 4}}
  \rbrack$]
    two stacks $\Rightarrow$ one queue; correctness proof; amortized analysis
    \begin{description}
      \item[\textsc{Enqueue}$(x)$:] \textsc{Push}$(S_1, x)$; 
      \item[\textsc{Dequeue}$()$:] while $S_2 = \emptyset$,
      \textsc{Push}$(S_2, \textsc{Pop}(S_1))$; \textsc{Pop}$(S_2)$; 
      \item[Ex:] \textsc{Enqueue}$(1,2,3)$, \textsc{Dequeue}$()$
    \end{description}
  \end{Problem}
  
  \vspace{0.30cm}
  Simple observation: \fbox{$S_1$ to push; $S_2$ to pop.}
  \begin{align*}
    FIFO &\Leftrightarrow \\
    &\forall \textsc{De}_1 \prec \textsc{De}_2: a =
    \textsc{De}_1, b = \textsc{De}_2 \Rightarrow \exists \textsc{En}_1(a) \prec
    \textsc{En}_2(b).
  \end{align*}
  \begin{itemize}
    \item Summation method: the sequence is (\uppercase{not}) known
    \item Accounting method: $\hat{c_i} = c_i + a_i$; $\sum_i^n \hat{c_i} \geq
    \sum_i^n c_i$; \fbox{$\sum_i^n a_i \geq 0, \forall n$}
  \end{itemize}
\end{frame}

\begin{frame}
  \frametitle{Two Stacks, One Queue (2)}
  
  \begin{table}
    \begin{tabular}{ccccc}
	  item: & \textsc{Push} into $S_1$ & \textsc{Pop} from $S_1$ & \textsc{Push}
	  into $S_2$ & \textsc{Pop} from $S_2$ \\
	  & 1 & 1 & 1 & 1
    \end{tabular}
  \end{table}

  \begin{table}
    \begin{tabular}{cccc}
	  & $\hat{c_i}$ & $c_i$ & $a_i$ \\
	  \textsc{Enqueue} & 3 & 2 & 1 \\
	  \textsc{Dequeue} & 2 & 2 & 0
    \end{tabular}
  \end{table} 
  Ex: \textsc{Enqueue}$(1,2,3)$, \textsc{Dequeue}$()$
  
  \[ \sum a_i = \#S_1 \times 2 \geq 0.\] 
\end{frame}

\begin{frame}
  \frametitle{Two Queues, One Stack (1)}
  
  \begin{Problem}[Two Queues, One Stack]
    implement/simulate a stack using two queues
  \end{Problem}
  
  Let's try [left: example; right: code]:
  \begin{itemize}
    \item push (1,2,3,4)
    \item pop (4) $\Rightarrow Q_1 \xrightarrow{(1,2,3)} Q_2$
    [\textcolor{blue}{transfer}]
    \item pop (3) $\Rightarrow Q_2 \xrightarrow{(1,2)} Q_1$
    \item push (5,6,7)
    \item pop 7 $\Rightarrow Q_1 \xrightarrow{(1,2,5,6)} Q_2$
  \end{itemize}
  
  \vspace{0.50cm}
  $Q_1$ for push; $Q_2$ for pop:
  \begin{description}
    \item[Push$(x)$:] Enqueue$(Q_1,x)$
    \item[Pop$()$:] $Q_1 \xrightarrow{transfer} Q_2$; swap $Q_1 \leftrightarrow
    Q_2$
  \end{description}
\end{frame}

\begin{frame}
  \frametitle{Two Queues, One Stack (2)}

  Analysis:
  \[ Push^{n}(Push^1 Pop^1)^{n/2} \]
  \[ 
    \sum c_i = n + (1 + (n+1)) \times \frac{n}{2} = n + (n+2) \times
    \frac{n}{2} = (n^2 + 4n)/2
  \]
  \[ (\sum c_i)/n = (n+4)/2 = \Theta(n) \]
  
  \begin{block}{Remark: \textcolor{red}{Why so bad?}}
    \begin{itemize}
      \item only use \emph{one} queue: push
      (1,2,3,4); pop (4); [\textcolor{blue}{circulate}]
      \item one queue + \textcolor{blue}{circulate} 
        \begin{description}
          \item [\textsc{Push$(x)$}:] \textsc{Enqueue$(Q_1,x)$};
          circulate$(Q_1)$
          \item [\textsc{Pop$()$}:] \textsc{Dequeue$(Q_1)$}
        \end{description}
      \item review of ``array doubling'': expensive, cheap, cheap, $\ldots$,
      expensive
      \item \textsc{Push}: \textcolor{red}{expensive, cheap, cheap, $\ldots$,
      expensive?}
    \end{itemize}
  \end{block}
\end{frame}

\begin{frame}
  \frametitle{Two Queues, One Stack (3)}
  
  Hey, $Q_2$:
  \begin{itemize}
    \item split the elements into $Q_2$ (cache vs. memory); \\
    do not change \textsc{Pop} $\Rightarrow$ ``stack order''
    \item $I_1$ [stack order]: $Q_1$ for top of stack; $Q_2$ for bottom of stack
        \begin{description}
          \item [\textsc{Push$(x)$}:] \textsc{Enqueue$(Q_1,x)$};
          circulate$(Q_1)$
          \item [\textsc{Pop$()$}:] if $Q_1 \neq \emptyset$
          \textsc{Dequeue$(Q_1)$}; else \textsc{Dequeue$(Q_2)$} 
        \end{description}    
    \item $I_2$ [$\#Q_1 < \sqrt{\#Q_2}$]: keep $\#Q_1$ small
  \end{itemize}
  
  \vspace{0.50cm}
  Ex: 
  \begin{itemize}
    \item \textsc{Push$(1,2,3,4,5,6)$} :$Q_1: 6$; $Q_2: 5,4,3,2,1$
    (\textcolor{red}{why?})
    \item \textsc{Push$(7,8)$} \textcolor{red}{how to [transfer] now?}
    \item $Q_2 \xrightarrow{transfer} Q_1$; swap $Q_1 \leftrightarrow Q_2$
    \item \textsc{Push$(9,10,11)$}
  \end{itemize}
\end{frame}

\begin{frame}
  \frametitle{Two Queues, One Stack (4)}
  
  \begin{itemize}
    \setlength{\itemsep}{10pt}
    \item ultimate code:
        \begin{description}
          \item [\textsc{Push$(x)$}:] \textsc{Enqueue$(Q_1,x)$};
          circulate$(Q_1)$; \\
          if $\#Q_1 \geq \sqrt{\#Q_2}$ \\ 
          $Q_2 \xrightarrow{transfer} Q_1$; swap $Q_1 \leftrightarrow Q_2$;
          \item [\textsc{Pop$()$}:] if $Q_1 \neq \emptyset$
          \textsc{Dequeue$(Q_1)$}; else \textsc{Dequeue$(Q_2)$} 
        \end{description}  
	\item analysis: \textsc{Pop} ($O(1)$), \textsc{Push}:
	  \begin{description}
	    \item[cheap:] $\#Q_1 < \sqrt{\#Q_2} \Rightarrow O(\sqrt{n})$
	    \item[expensive:] $\#Q_1 \neq \sqrt{\#Q_2} \Rightarrow O(n)$
	    \item[amortized:] $E, \underbrace{C, C, \ldots, C}_{\# = O(\sqrt{n})}, E
	    \Rightarrow O(\sqrt{n})$
	  \end{description}
  \end{itemize}
\end{frame}
%%%%%%%%%%%%%%%%%%%
% \section{A Spare Adversary}
% 
% \begin{frame}
%   \frametitle{01 Adversary}
%   
%   \begin{Problem}[01 Adversary $\lbrack {\bf P_{245}, 5.20} \rbrack$]
%     bit array $A[1 \ldots n]$; if it contains the substring $01$? \\
%     can we answer this question without looking at every bit?
%   \end{Problem}
%   
%   \begin{itemize}
%     \item $n$ is odd ($n=9$): check each \emph{even} bit $A[2], A[4], \ldots
%     A[8]$
%       \begin{itemize}
%         \item all 1: $A[n]$
%         \item all 0: $A[1]$
%         \item all 1 come before 0 (1\;1\;0\;0): $A[4], A[6]$ 
%         \item $\exists i < j, A[i] = 0, A[j] = 1\; (A[2], A[6])$
%       \end{itemize}
%     \item $n$ is even
%       \begin{itemize}
%         \item $1\;1\;1\;1\;1\;1\; \Box \Box \Box \Box \Box \Box\; 0\;0\;0\;0$
%         \item Init: $l = 0; r = n + 1$
%         \item Invariant: $r - l - 1$ is even
%         \item \textsc{Look($i$):} $i \leq l \Rightarrow 1$; $i \geq r
%         \Rightarrow 0$; \\ 
%         $i-r$ is even $\Rightarrow (r \gets i; 0)$; \\
%         $i-l$ is even $\Rightarrow (l \gets i; 1)$
%       \end{itemize}
%   \end{itemize}
% \end{frame}
% %%%%%%%%%%%%%%%%%%%
% \section*{Summary}

%%%%%%%%%%
\end{document}
