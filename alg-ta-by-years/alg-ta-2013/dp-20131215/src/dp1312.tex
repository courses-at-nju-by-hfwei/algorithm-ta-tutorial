%% LaTeX Beamer presentation template (requires beamer package)
%% see http://bitbucket.org/rivanvx/beamer/wiki/Home
%% idea contributed by H. Turgut Uyar
%% template based on a template by Till Tantau
%% this template is still evolving - it might differ in future releases!

\documentclass{beamer}
\mode<presentation>
{
	\usetheme{Dresden} % Dresden, Singapore, Amsterdam
	\usecolortheme{beaver}
	\setbeamercolor{itemize item}{fg=darkred!80!black}
	\setbeamertemplate{footline}[frame number]
	\setbeamerfont{subsection in toc}{size = \small}
% 	\setbeamercovered{transparent}
}

\usepackage{tikz, adjustbox}
\usetikzlibrary{calc, shapes.callouts, decorations.text}
\usetikzlibrary{positioning, arrows, decorations.pathreplacing,
decorations.pathmorphing, backgrounds, fit}

\newcommand{\incitefull}[3]{\textcolor{blue}{\scriptsize [#1@#2'#3]}}

\newcommand{\inputtikz}[2] % #1: size; #2: tikz file (without prefix: tex(tikz))
{
      \begin{center}
      \adjustbox{max width = #1}
      {
        \input{tex(tikz)/#2}
      }
      \end{center}
}

% new command for figure:
% #1: size; #2: figure file (without prefix figure/)
\newcommand{\fig}[2]
{
  \begin{figure}[htp]
	  \centering
	  \includegraphics[#1]{figure/#2}
  \end{figure}
}

\usepackage{graphicx}
\usepackage{mdframed}
\usepackage{amsmath}
\usepackage{wrapfig}
\usepackage{algpseudocode}
\usepackage{multicol}

\title{Dynamic Programming}

\author{Hengfeng Wei}

\institute[Universities of]
{
	Institute of Computer Software, NJU
}

\date{\today}


% If you have a file called "university-logo-filename.xxx", where xxx
% is a graphic format that can be processed by latex or pdflatex,
% resp., then you can add a logo as follows:

% \pgfdeclareimage[height=0.5cm]{university-logo}{university-logo-filename}
% \logo{\pgfuseimage{university-logo}}



% Delete this, if you do not want the table of contents to pop up at
% the beginning of each subsection:
\AtBeginSubsection[]
{
	\begin{frame}<beamer>
		\frametitle{Outline}
		\begin{multicols}{2}
		  \tableofcontents[currentsection]
		\end{multicols}
	\end{frame}
}

\AtBeginSection[]
{
	\begin{frame}<beamer>
		\frametitle{Outline}
		\begin{multicols}{2}
		  \tableofcontents[currentsection]
		\end{multicols}
	\end{frame}
}
% If you wish to uncover everything in a step-wise fashion, uncomment
% the following command:

%\beamerdefaultoverlayspecification{<+->}

\begin{document}

\begin{frame}
	\titlepage
\end{frame}

\begin{frame}
	\frametitle{Outline}
		\begin{multicols}{2}
		  \tableofcontents[currentsection]
		\end{multicols}
% You might wish to add the option [pausesections]
\end{frame}
%%%%%%%%%%%%%%%%%%%
\section{Overview}

%%%%%%%%%
\begin{frame}{Overview}
  \begin{mdframed}
    \begin{center}
    Dynamic programming is \emph{not} about filling in tables!
    
    It is about smart recursion.
    
    Smart recursion is about the \textcolor{blue}{\emph{structure of
    subproblems}}.
    \end{center}
  \end{mdframed}
\end{frame}
%%%%%%%%%
\begin{frame}{Common Subproblems}
  \begin{itemize}
    \setlength{\itemsep}{8pt}
    \item 1-dimension subproblems
      \begin{itemize}
        \item input: string (or array) $x[1 \ldots n]$;  \\
        subproblem: prefix ($x[1 \ldots i]$), suffix ($x[i \ldots n]$);  \\
        examples: Maximum Subarray Sum, Typesetting Problem,
        \textcolor{blue}{Longest Increasing Subsequence, Hotel Placement}
      \end{itemize}
    \item 2-dimension subproblems
      \begin{itemize}
        \setlength{\itemsep}{5pt}
        \item input: two strings $x[1 \ldots m], y[1 \ldots n]$; \\ 
        subproblem: prefixes $x[1 \ldots i], y[1 \ldots j]$;  \\
        examples: \textcolor{blue}{Edit Distance, Longest Common Subsequence}
        \item input: a string $x[1 \ldots n]$; \\
        subproblem: interval $x[x \ldots j]$;  \\
        examples: Chain Matrix Multiplication, Optimal BST,
        \textcolor{blue}{Longest Palindrome}
      \end{itemize}
  \end{itemize}
\end{frame}
%%%%%%%%%
\begin{frame}{Common Subproblems}
  \begin{itemize}
    \setlength{\itemsep}{8pt}
    \item 3-dimension subproblems
      \begin{itemize}
        \item Floyd-Warshall algorithm
      \end{itemize}
    \item tree-like subproblems
      \begin{itemize}
        \item input: tree;
          subproblem: rooted subtree;  \\
          example: \textcolor{blue}{Independent Set in Tree}
      \end{itemize}
    \item two mind games
      \begin{itemize}
        \item \textcolor{blue}{Google Egg Game, Hungry-Lion Game}
      \end{itemize}
  \end{itemize}
\end{frame}
%%%%%%%%%
\begin{frame}{Common Subproblems}
  How to identify subproblems?
  \vspace{0.50cm}
  \begin{mdframed}
    \centering
    Make your choice:
    \begin{itemize}
      \item binary choice (whether)
        \begin{itemize}
          \item coin-changing, $\ldots$
        \end{itemize}
      \item multi-way choices (where, which $\ldots$)
        \begin{itemize}
          \item optimal BST, $\ldots$
        \end{itemize}
    \end{itemize}
  \end{mdframed}
\end{frame}
%%%%%%%%%%%%%%%%%%%
\section{1-Dimension Subproblems}

%%%%%%%%%%%%%%
\subsection{Longest Increasing Subsequence}

%%%%%%%%%
\begin{frame}{Longest Increasing Subsequence}
  \begin{problem}[Longest Increasing Subsequence (LIS)]
    \begin{itemize}
      \item given an integer array $A[1 \ldots n]$
      \item to find (the length of) a longest increasing subseq.
    \end{itemize}
    
    \vspace{0.50cm}
    \begin{mdframed}[leftmargin = 2.5cm, rightmargin = 2.5cm]
      \[ A: 5,2,8,6,3,6,7,9 \]
      \[ IS: 5,8,9; \; 2, 3, 6, 9 \]
    \end{mdframed}
  \end{problem}
\end{frame}
%%%%%%%%%
\begin{frame}{Longest Increasing Subsequence}

  \begin{mdframed}[rightmargin = 3.0cm]
    $L(i):$ the length of the LIS of $A[1 \ldots i]$.
  \end{mdframed}
  
  \vspace{0.50cm}
  \begin{theorem}[Recurrence]
    \fbox{Make choice:} whether $A[i] \in LIS[1 \ldots i]$.
    \begin{displaymath}
      L(i) = \max \left\{
        \begin{array}{ll}
          L(i - 1) & \textrm{if } A[i] \notin LIS[1 \ldots i]  \\
          1 + \max \{ L(j): j < i \land A[j] < A[i] \} & \textrm{o.w.}    
        \end{array}
      \right.
    \end{displaymath}
  \end{theorem}
  \vspace{0.50cm}
  \fbox{Base cases:} $L(0) = 0$.
\end{frame}
%%%%%%%%%
\begin{frame}{Longest Increasing Subsequence}
  Filling the table:
  \fig{width = 0.70\textwidth}{LISorder}
  
  \vspace{0.50cm}
  Time complexity: $\Theta(n^2)$
  
  Space complexity: $\Theta(n)$
\end{frame}
%%%%%%%%%
\subsection{Hotel Placement}

%%%%%%%%%
\begin{frame}{Hotel Placement}
  \begin{problem}[Hotel Placement]
    \begin{itemize}
      \item open hotels along a highway
      \item possible locations: $X[1 \ldots n] = x_1, \ldots, x_n$
      \item profit: $P[1 \ldots n] = p_1, \ldots, p_n$
      \item any two hotels should be at least $k$ miles apart
      \item \fbox{Goal:} to maximize the total profit
    \end{itemize}
  \end{problem}
\end{frame}
%%%%%%%%%
\begin{frame}{Hotel Placement}
  \begin{mdframed}
    $M(i):$ max profit when considering only the first $i$ locations.
  \end{mdframed}
  
  \vspace{0.50cm}
  \begin{theorem}[Recurrence]
    \fbox{Make choice:} whether to build a hotel at location $i$.
    \begin{displaymath}
      M(i) = \max \left\{
        \begin{array}{ll}
          M(i-1) & \textrm{do not build at } i  \\
          p_i + \max \{ M(j): j < i \land x_i - x_j \ge k \} & \textrm{o.w.}
        \end{array}
      \right.
    \end{displaymath}
    
    \fbox{Base cases:} $M(0) = 0$

    \vspace{0.30cm}
    Time complexity: $\Theta(n^2)$
    
    Space complexity: $\Theta(n)$
  \end{theorem}
\end{frame}
%%%%%%%%%
\subsection{Typesetting Problem}

%%%%%%%%
\begin{frame}{Typesetting Problem}
  \begin{problem}[Typesetting Problem]
    \begin{itemize}
      \item text: $n$ words of widths $W: w_1, \ldots, w_n$
      \item line width $L$
      \item penalty: $f(w_i \ldots w_j)$
      \item \fbox{Goal:} to minimize the typesetting penalty
    \end{itemize}
  \end{problem}
\end{frame}
%%%%%%%%%
\begin{frame}{Typesetting Problem}
  \begin{mdframed}[rightmargin = 2cm]
    $P(i):$ the minimum penalty to typeset $W[i \ldots n]$.
  \end{mdframed}
  
  \vspace{0.50cm}
  \begin{theorem}[Recurrence]
    \fbox{Make choice:} how many words to put in the first line.
    \begin{displaymath}
      P(i) = \left\{
        \begin{array}{ll}
          0 & \textrm{if } \sum (w_i, \ldots, w_n) \le L  \\
          \min_{w_i + \ldots w_{i+k-1} \le L} f(w_i \ldots w_k) + P(i+k) &
          \textrm{o.w.}
          \\
        \end{array}
      \right.
    \end{displaymath}
  \end{theorem}
\end{frame}
%%%%%%%%%
\begin{frame}{Typesetting Problem}
  Filling the table:
  \fig{width = 0.60\textwidth}{typesettingorder}
  \vspace{0.50cm}
  
  Time complexity: $\Theta(nW)$
  
  Space complexity: $\Theta(n)$
\end{frame}
% \begin{frame}{Rod-Cutting Problem}
%   \begin{problem}[Rod-Cutting Problem]
%     \begin{itemize}
%       \item given a rod of length $n$
%       \item I want to cut it into pieces
%       \item profit table $P[1 \ldots n]$
%       \item \fbox{Goal:} to maximize the profit
%     \end{itemize}
%   \end{problem}
% \end{frame}
%%%%%%%%%%%%%%%%%%%
\section{2-Dimension Subproblems}

%%%%%%%%%%%%%%
\subsection{Edit Distance}

%%%%%%%%%
\begin{frame}{Edit Distance}
  \begin{problem}[Edit Distance]
    \begin{itemize}
      \item transform string $A[1 \ldots m]$ to another one $B[1 \ldots n]$
      \item allowed edits: insertion, deletion, substitution
        \fig{width = 0.90\textwidth}{editdistanceex}
      \item \fbox{Goal:} to find the edit distance (similarity) --- the minimum
      number of edits
    \end{itemize}
  \end{problem}
\end{frame}
%%%%%%%%%
\begin{frame}{Edit Distance}
  \begin{mdframed}[rightmargin = 1cm]
    $E(i,j):$ the edit distance of $A[1 \ldots i]$ and $B[1 \ldots j]$.
  \end{mdframed}
  
  \vspace{0.50cm}
  \begin{theorem}[Recurrence]
    \fbox{Make choices:} three cases for the rightmost column
    \begin{displaymath}
      E(i,j) = \min \left\{
        \begin{array}{ll}
          E(i-1,j) + 1 & (x[i], \_) \\
          E(i, j-1) + 1 & (\_, y[j]) \\
          E(i-1, j-1) & (x[i], y[j]) \textrm{ and $A[i] = B[j]$}  \\
          E(i-1, j-1) + 1 & (x[i], y[j]) \textrm{ and $A[i] \neq B[j]$}  
        \end{array}
      \right.
    \end{displaymath}
    \fbox{Base cases:}
    $E(i,0) = i \; (deletion); E(0,j) = j \; (insertion).$ 
  \end{theorem}
\end{frame}
%%%%%%%%%
\begin{frame}{Edit Distance}
  Filling the table:
  \begin{columns}
    \column{0.40\textwidth}
      \fig{width = 0.80\textwidth}{EDorder}
    \column{0.30\textwidth}
      \fig{width = 0.80\textwidth}{row-major}
    \column{0.30\textwidth}
      \fig{width = 0.80\textwidth}{col-major}
  \end{columns}
  
  \vspace{0.80cm}
  Time complexity: $\Theta(nm)$
  
  Space complexity: $\Theta(nm)$
\end{frame}
%%%%%%%%%%%%%%
\subsection{Longest Common Subsequence}
  
%%%%%%%%%
\begin{frame}{Longest Common Subsequence}
  \begin{problem}[Longest Common Subsequence (LCS)]
    \begin{itemize}
      \item another similarity measurement
      \item given two sequences $X[1 \ldots m]$ and $Y[1 \ldots n]$.
      \item to find (the length of) a longest subseq. common to both.
    \end{itemize} 
    \vspace{0.50cm}
    \begin{mdframed}[leftmargin = 1.5cm, rightmargin = 1.5cm]
      \begin{align*}
        X &: A \; B \; C \; B \; D \; A \; B \\
        Y &: B \; D \; C \; A \; B \; A
      \end{align*}
      \[
        LCS: B \; D \; A\; B; \; B\; C\; A\; B; \; B\; C\; B\; A; \;
      \]
    \end{mdframed}
  \end{problem}
\end{frame}
%%%%%%%%%
\begin{frame}{Longest Common Subsequence}
  \begin{mdframed}[rightmargin = 0.5cm]
    $L(i,j)$: the length of the LCS of $X[1 \ldots i]$ and $Y[1 \ldots j]$.
  \end{mdframed}
  
  \vspace{0.50cm}
  \fbox{Make choice:} Let $Z[1 \ldots k] = LCS(X[1 \ldots i], Y[1 \ldots j]$, \\
    whether $(X[i], Y[j]) \in Z[1 \ldots k]$?
  \begin{lemma}[Make Choice]
    \fbox{If} $(X[i],Y[j]) \notin Z[1 \ldots k]$, then either $X[i] \notin Z$ or
    $X[j] \notin Z$.
  \end{lemma}
  
  \pause
  \begin{proof}
    \fbox{By contradiction. \textcolor{blue}{Or, case by case.}} 
    \fig{width = 0.50\textwidth}{lcschoice}
  \end{proof}
\end{frame}
%%%%%%%%%
\begin{frame}{Longest Common Subsequence}
  \begin{theorem}[Recurrence]
    \begin{displaymath}
      L(i,j) = \left\{
        \begin{array}{ll}
      	  L(i-1, j-1) + 1 \textrm{ (\textcolor{red}{why?})} & \textrm{if } X[i] =
      	  Y[j]
      	  \\
      	  \max \{ L(i,j-1), L(i-1,j) \} & \textrm{o.w.}  
        \end{array}
      \right.
    \end{displaymath}
    
    \vspace{0.50cm}
    \fbox{Base cases:} 
    \[
      L(i,0) = 0; L(0,j) = 0;
    \]  
  \end{theorem}
\end{frame}
%%%%%%%%%
\begin{frame}{Longest Common Subsequence}
  \begin{columns}
    \column{0.40\textwidth}
      \fig{width = 0.80\textwidth}{EDorder}
    \column{0.30\textwidth}
      \fig{width = 0.80\textwidth}{row-major}
    \column{0.30\textwidth}
      \fig{width = 0.80\textwidth}{col-major}
  \end{columns}
  
  \vspace{0.80cm}
  Time complexity: $\Theta(nm)$
  
  Space complexity: $\Theta(nm)$
\end{frame}
%%%%%%%%%%%%%%
\subsection{Longest Palindrome}

%%%%%%%%%
\begin{frame}{Palindrome}
  \begin{problem}[Palindrome]
    \begin{itemize}
      \item given a seq. $X[1 \ldots n]$
      \item to find (the length of) a longest palindrome subseq.
    \end{itemize}
    
    \vspace{0.50cm}
    \begin{mdframed}[leftmargin = 2.0cm, rightmargin = 2.0cm]
      \begin{align*}
        X &: A\; C\; G\; T\; G\; T\; C\; A\; A\; A\; T\; C\; G\;  \\
        P &: A\; C\; G\; C\; A; \;\; A\; C\; G\; T\; G\; C\; A
      \end{align*}
    \end{mdframed}
  \end{problem}
\end{frame}
%%%%%%%%%
\begin{frame}{Palindrome}
  \begin{mdframed}[rightmargin = 1.5cm]
    $L[i \ldots j]:$ the length of the LP subseq. of $X[i \ldots
    j]$.
  \end{mdframed}
  
  \vspace{0.50cm}
  \fbox{Make choice:} Let $Z[1 \ldots k] = LP(X)$, whether $X[i] = Z[1] = X[j]
  = Z[k]$. 
  \begin{theorem}[Recurrence]
    \fbox{If} $X[i] \neq X[j]$, then either $X[i] \notin Z$ or $X[j] \notin Z$.
  \end{theorem}
  
  \begin{proof}
    \fbox{By contradiction.}
%    \fig{width = 0.50\textwidth}{palindrome}
  \end{proof}
\end{frame}
%%%%%%%%%
\begin{frame}{Palindrome}
  \begin{theorem}[Recurrence]
    \begin{displaymath}
      L(i,j) = \left\{
        \begin{array}{ll}
          L(i+1,j-1) + 2 \textrm{ (\textcolor{red}{why?})} & \textrm{if } X[i]
          = Y[j]
          \\
          \max \{ L(i+1,j), L(i, j-1) \} & \textrm{o.w.}
        \end{array}
      \right.
    \end{displaymath}
    
    \vspace{0.50cm}
    \fbox{Base cases:} 
    \[ L(i,i) = 1, \forall i = 1 \ldots n \]
  \end{theorem}
\end{frame}
%%%%%%%%%
\begin{frame}{Palindrome}
  Filling the table:
  \begin{columns}
    \column{0.30\textwidth}
    \fig{width = 0.95\textwidth}{palindromeorder}
    \column{0.70\textwidth}
    \fig{width = 0.90\textwidth}{threeorders}
    \begin{center}
      diagonal-major, row-major, column-major
    \end{center}
  \end{columns}
\end{frame}
%%%%%%%%%
\begin{frame}{Palindrome}
  \begin{columns}
    \column{0.60\textwidth}
	  Diagonal-major:
	  \vspace{0.20cm}
	  \begin{mdframed}
	  \begin{algorithmic}[1]
	    \For{$i = 1 \to n$}
	      \State $L(i,i) = 1$
	    \EndFor
	    \Statex
	    \For{$l = 1 \to n$}
	      \For{$i = 1 \to n-l$}
	        \State $j = i + l$
	        \State $L(i,j) = \ldots$
	      \EndFor
	    \EndFor
	  \end{algorithmic}
	  \end{mdframed}
	\column{0.40\textwidth}
	  \begin{mdframed}
	    \textcolor{red}{Questions:} 
	    \begin{itemize}
	      \item ED vs. LCS
	      \item LCS vs. LP
	    \end{itemize}
	  \end{mdframed}
  \end{columns}
\end{frame}
%%%%%%%%%
%%%%%%%%%%%%%%%%%%%
\section{3-Dimension Subproblems}

%%%%%%%%%%%%%%%%%%%
\subsection{Floyd-Warshall Algorithm}

%%%%%%%%%
\begin{frame}{Floyd-Warshall Algorithm Revisited}
  \fbox{$D^{k}(i,j) \equiv D(i,j,k)$:} the length of the shortest path from $i$
  to $j$ where intermediate vertex is numbered at most $k$.
  
  \begin{theorem}[Recurrence]
    \fig{width = 0.70\textwidth}{floyd-warshall-recursion}
    \[
      D^{k}(i,j) = \min \{ D^{k-1}(i,k) + D^{k-1}(k,j), D^{k-1}(i,j) \}
    \]
    
    \vspace{0.20cm}
    \fbox{Base cases:} 
    \[ D^{0}(i,j) = w(i \to j). \]
  \end{theorem}
\end{frame}
%%%%%%%%%%%%%%%%%%%
\section{Tree-Like Subproblems}

%%%%%%%%%%%%%%
\subsection{Independent Set in Tree}

%%%%%%%%%
\begin{frame}{Independent Set in Tree}
  \begin{problem}{Independent Set in Tree}
    \begin{itemize}
      \item \fbox{Goal:} to find the largest independent set in \emph{tree}
    \end{itemize}
  \end{problem}
\end{frame}
%%%%%%%%%
\begin{frame}{Independent Set in Tree}
  \begin{mdframed}
    $I(u):$ the size of the LIS of the subtree rooted at $u$.
  \end{mdframed}
  
  \vspace{0.30cm}
  \begin{theorem}[Recurrence]
    \fbox{Make choice:} whether $u$ is in LIS.
      \[
        I(u) = \max \big\{
        \sum_{\textrm{children } w \textrm{ of } u} I(w),
		1 + \sum_{\textrm{grandchildren } w \textrm{ of } u} I(w) \big\}.
	  \]
  \end{theorem}
  
  \vspace{0.30cm}
  \fbox{Base cases:} $I(v) = 1, \forall \textrm{ leave in tree.}$
\end{frame}
%%%%%%%%%%%%%%%%%%%
\section{Two Mind Games}

%%%%%%%%%%%%%%
\subsection{Google Eggs Game}

%%%%%%%%%
\begin{frame}{Google Eggs Game}
  \begin{problem}[Google Eggs Game]
    \begin{itemize}
      \item $n$ floors, $m$ eggs
      \item test the ``quality'' of egg: critical floor $c$
      \item cases of $c=0, c=n$
      \item \fbox{Goal:} to determine $c$ while minimizing the number of throws 
    \end{itemize}
  \end{problem}
\end{frame}
%%%%%%%%%
\begin{frame}{Google Eggs Game}
  \begin{mdframed}
    $T(i,j):$ minimum number of throws with $i$ floors and $j$ eggs
  \end{mdframed}
  
  \vspace{0.30cm}
  \begin{theorem}[Recurrence]
    \fbox{Make choice:} which floor to throw; broken or not.
    \begin{align*}
      T(i,j) &= \min_{1 \le k \le i} \big( T(i,j \mid k) \big)  \\
      T(i,j \mid k) &= \max \{ T(i,j \mid k,Y), T(i,j \mid k,N) \} \\
      T(i,j \mid k,Y) &= 1 + T(k-1, j-1)  \\
      T(i,j \mid k,N) &= 1 + T(i-k, j)
    \end{align*}
    \[
	  T(i,j) = \min_{1 \le k \le i} \{ 1 + \max \{ T(k-1,j-1), T(i-k,j) \} \}.      
    \]
    \vspace{0.30cm}
    \fbox{Base cases:} $T(n,0) = 0, T(0,m) = 0; T(i,1) = i, T(1,j) = 1$
  \end{theorem}
\end{frame}
%%%%%%%%%
\begin{frame}{Google Eggs Game}
  Filling the table:
  \begin{columns}
    \column{0.40\textwidth}
      \fig{width = 0.90\textwidth}{googleeggorder}
    \column{0.30\textwidth}
      \fig{width = 0.75\textwidth}{row-major}
    \column{0.30\textwidth}
      \fig{width = 0.75\textwidth}{col-major}
  \end{columns}

  \vspace{1.00cm}
  \begin{mdframed}
    \textcolor{blue}{Strongly recommend}: numerical experiments
  \end{mdframed}
\end{frame}
%%%%%%%%%%%%%%
\subsection{Hungry-Lion Game}

%%%%%%%%%
\begin{frame}{Hungry-Lion Game}
  \begin{problem}[Hungry-Lion Game]
    \begin{itemize}
      \item a sheep in danger
      \item hungry lions
    \end{itemize}
    \begin{columns}
      \column{0.20\textwidth}
        \fig{width = 0.90\textwidth}{sheep}
       \column{0.20\textwidth}
        \fig{width = 0.90\textwidth}{lionking}
      \column{0.15\textwidth}
        \fig{width = 0.90\textwidth}{lion}
      \column{0.15\textwidth}
        \fig{width = 0.90\textwidth}{lion}
      \column{0.15\textwidth}
        \fig{width = 0.90\textwidth}{lion}  
      \column{0.15\textwidth}
        \fig{width = 0.90\textwidth}{lion}
    \end{columns}
  \end{problem}
\end{frame}
%%%%%%%%%%
\end{document}
