% file: parts/amortized-analysis.tex

%%%%%%%%%%%%%%%
\begin{frame}{}
  \centerline{\teal{\Large Amortized Analysis}}

  \fig{width = 0.45\textwidth}{figs/amortized-analysis-pig}
\end{frame}
%%%%%%%%%%%%%%%

%%%%%%%%%%%%%%%
\begin{frame}{}
  \begin{quote}
    \centering
    {\large
      \red{\Large Amortized analysis} is \\[6pt]
      \blue{an algorithm analysis technique} for \\[6pt]
      \purple{analyzing a sequence of operations} \\[6pt]
      \blue{irrespective of the input} to show that \\[6pt]
      \textcolor{cyan}{the average cost per operation} is small, even though \\[6pt]
      \teal{a single operation within the sequence might be expensive}.
    }
  \end{quote}
\end{frame}
%%%%%%%%%%%%%%%

%%%%%%%%%%%%%%%
\begin{frame}{}
  \fig{width = 0.75\textwidth}{figs/amortization-three-methods}
\end{frame}
%%%%%%%%%%%%%%%

%%%%%%%%%%%%%%%
\begin{frame}{}
  \centerline{\teal{\Large The Summation Method}}

  \vspace{0.50cm}
  \fig{width = 0.20\textwidth}{figs/sigma}
\end{frame}
%%%%%%%%%%%%%%%

%%%%%%%%%%%%%%%
\begin{frame}{}
  \[
    o_1, o_2, \ldots, o_n
  \]

  \[
    c_1, c_2, \ldots, c_n
  \]

  \pause
  \[
    \boxed{\red{\forall i,\; \hat{c_i} = \frac{\left(\sum\limits_{i = 1}^{n} c_i\right)}{n}}}
  \]
\end{frame}
%%%%%%%%%%%%%%%

%%%%%%%%%%%%%%%
\begin{frame}{}
  \centerline{\teal{\large The Summation Method for Array Doubling}}

  \pause
  \vspace{0.30cm}
  \centerline{On \red{any sequence} of $n$ \textsc{Insert}s on an \blue{initially empty} array.}

  \pause
  \vspace{0.30cm}
  \[
    \begin{array}{ccccccccccc}
      o_i:  & o_1 & \red{o_2} & \red{o_3} & o_4 & \red{o_5} & o_6 & o_7 & o_8 & \red{o_9} & o_{10} \\[3pt]
      c_i:  & 1 & \red{2} & \red{3} & 1 & \red{5} & 1 & 1 & 1 & \red{9} & 1  \\
    \end{array}
  \]

  \pause
  \vspace{0.30cm}
  \begin{displaymath}
    c_i = \left\{ \begin{array}{ll}
      (i-1)+1 = i & \textrm{if $i - 1$ is an exact power of 2}\\
      1 & \textrm{o.w.}
    \end{array} \right.
  \end{displaymath}

  \pause
  \vspace{0.30cm}
  \[
    \sum_{i=1}^{n} c_i \le n + \sum_{j=0}^{\lceil \lg n \rceil - 1} 2^{j} = n +
    (2^{\lceil \lg n \rceil} - 1) \le n + 2n = 3n
  \]

  \pause
  \[
    \red{\boxed{\forall i,\; \hat{c_i} = 3}}
  \]
\end{frame}
%%%%%%%%%%%%%%%

%%%%%%%%%%%%%%%
\begin{frame}{}
  \centerline{\teal{\Large The Accounting Method}}

  \vspace{0.50cm}
  \fig{width = 0.40\textwidth}{figs/accounting}
\end{frame}
%%%%%%%%%%%%%%%

%%%%%%%%%%%%%%%
\begin{frame}{}
  \begin{gather*}
    o_1, o_2, \ldots, o_n \\[5pt]
    c_1, c_2, \ldots, c_n \\[5pt]
    \red{a_1, a_2, \ldots, a_n}
  \end{gather*}

  \pause
  \vspace{-0.30cm}
  \[
    \red{\boxed{\hat{c_i} = c_i + a_i, \; a_i >=< 0}}
  \]

  \[
    \teal{\text{Amortized Cost } = \text{ Actual Cost } + \text{ Accounting Cost }}
  \]

  \pause
  \[
	\blue{\forall n,\; \sum_{i=1}^{n} c_i \le \sum_{i=1}^{n} \hat{c_i}} \pause \implies \red{\boxed{\forall n,\; \sum_{i=1}^{n} a_i \geq 0}}
  \]

  \pause
  \begin{center}
    {\large \red{Key Point: Put the accounting cost on specific objects.}}
  \end{center}
\end{frame}
%%%%%%%%%%%%%%%

%%%%%%%%%%%%%%%
\begin{frame}{}
  \centerline{\teal{\large The Accounting Method for Array Doubling}}

  \[
    \red{Q:}\; \hat{c_i} = 3 \text{\emph{ vs. }} \hat{c_i} = 2
  \]

  \pause
  \[
    \hat{c_i} = 3 = \underbrace{1}_{\textrm{insert}} +
    \underbrace{1}_{\textrm{move itself}} + \underbrace{1}_{\textrm{\red{help move another}}}
  \]

  \pause
  \vspace{0.30cm}
  \begin{table}
    \begin{tabular}{c|ccc}
      & $\hat{c_i}$ & $c_i$ {\it (actual cost)} & $a_i$ {\it (accounting cost)}
      \\ \hline
      \textsc{Insert} {\it (normal)} & $3$ & $1$ & $2$\\
      \textsc{Insert} {\it (expansion)} & $3$ & $1 + t$ & $-t + 2$
    \end{tabular}
  \end{table}
\end{frame}
%%%%%%%%%%%%%%%

%%%%%%%%%%%%%%%
\begin{frame}{}
  \centerline{\teal{\large Simulating a queue $Q$ using two stacks $S_1, S_2$ {\small (Problem $\mathbb{E}3$)}}}

  \begin{columns}
	\column{0.20\textwidth}
	\column{0.60\textwidth}
	  \begin{algorithm}[H]
		\begin{algorithmic}[]
		  \Procedure{Enq}{$x$}
		\State \textsl{Push}($S_1, x$)
		  \EndProcedure

		  \Statex
		  \Procedure{Deq}{\null}
			\If{$S_2 = \emptyset$}
			  \While{$S_1 \neq \emptyset$}
				\State \textsl{Push}($S_2$, \textsl{Pop}($S_1$))
			  \EndWhile
			\EndIf
			\textsl{Pop}($S_2$)
		  \EndProcedure
		\end{algorithmic}
	  \end{algorithm}
	\column{0.20\textwidth}
  \end{columns}
\end{frame}
%%%%%%%%%%%%%

%%%%%%%%%%%%%
\begin{frame}{}
  \centerline{\teal{\large The Summation Method for Queue Simulation}}

  \vspace{0.30cm}
  \[
    \frac{\left(\sum\limits_{i = 1}^{n} c_i\right)}{n}
  \]

  \pause
  \vspace{0.50cm}
  \centerline{The operation sequence is \red{\it NOT} known.}
\end{frame}
%%%%%%%%%%%%%

%%%%%%%%%%%%%
\begin{frame}{}
  \centerline{\teal{\large The Accounting Method for Queue Simulation}}

  \begin{table}
    \begin{tabular}{ccccc}
      {\it item}: & Push into $S_1$ & Pop from $S_1$ & Push into $S_2$ & Pop from $S_2$ \\
      & $1$ & $1$ & $1$ & $1$
    \end{tabular}
  \end{table}

  \pause
  \vspace{-0.60cm}
  \begin{align*}
    \hat{c}_{\textsc{Enq}} &= 3 \\
    \hat{c}_{\textsc{Deq}} &= 1 \\
  \end{align*}

  \pause
  \vspace{-0.60cm}
  \[ 
	\red{\sum_{i=1}^{n} a_i \ge 0} \pause \Longleftarrow \sum_{i=1}^{n} a_i = \#S_1 \times 2
  \]
\end{frame}
%%%%%%%%%%%%%

%%%%%%%%%%%%%
\begin{frame}{}
  \centerline{\teal{\large The Accounting Method for Queue Simulation}}

  \begin{align*}
    \hat{c}_{\textsc{Enq}} &= 3 \\
    \hat{c}_{\textsc{Deq}} &= 1 \\
  \end{align*}

  \vspace{-0.60cm}
  \[
	\blue{\#S_1 = t}
  \]

  \begin{table}
    \begin{tabular}{c|ccc}
      & $\hat{c_i}$ & $c_i$ {\it (actual cost)} & $a_i$ {\it (accounting cost)} \\ \hline
      \textsc{Enqueue} & $3$ & $1$ & $2$ \\
      \textsc{Dequeue} ($S_2 \neq \emptyset$) & $1$ & $1$ & $0$ \\
      \textsc{Dequeue} ($S_2 = \emptyset$) & $1$ & $1+2t$ & $-2t$
    \end{tabular}
  \end{table}
\end{frame}
%%%%%%%%%%%%%

%%%%%%%%%%%%%
\begin{frame}{}
  \begin{exampleblock}{Array Merging Dictionary (Problem $\mathbb{E}\; 2$)}
    \begin{columns}
      \column{0.50\textwidth}
        \pause
		\begin{align*}
		  \red{i} & \quad \red{s_i = 2^i} \\
		  A_{0} & \quad 1 \\
		  A_{1} & \quad 2 \\
		  A_{2} & \quad 4 \\
		  A_{3} & \quad 8 \\
		  \vdots & \quad \cdots \\
		  A_{i} & \quad 2^{i}
		\end{align*}
      \column{0.50\textwidth}
        \pause
        \[
		  11 = 2^{0} + 2^{1} + 2^{3} 
		\]
	\begin{align*}
	  \red{i} & \quad \red{e_i} \\
	  A_{0} & \quad [5] \\
	  A_{1} & \quad [4,8] \\
	  A_{2} & \quad [\;] \\
	  A_{3} & \quad [2, 6, 9, 12, 13, 16, 20, 25]
	\end{align*}
  \end{columns}
    
  \pause
  \vspace{0.30cm}
  \[
	\blue{\textsc{Create}: 1 \quad \textsc{Merge$(A_i, A_i)$}: 2 \cdot 2^i}
  \]
  \end{exampleblock}

  \pause
  \[
    \textsc{\teal{Insert()}}: 1 + 2 + 4; \quad \pause \textsc{\teal{Insert()}}: 1; \quad \pause \textsc{\teal{Insert()}}: 1 + 2
  \]
\end{frame}
%%%%%%%%%%%%%

%%%%%%%%%%%%%
\begin{frame}{}
  \begin{exampleblock}{The Summation Method for ``Array Merging Dictionary''}
    \[
	  \blue{\textsc{Create}: 1 \quad \textsc{Merge$(A_i, A_i)$}: 2 \cdot 2^i}
    \]
  \end{exampleblock}

  \begin{columns}
    \column{0.40\textwidth}
      \pause
      \begin{align*}
	i & \quad c_i \\
	1 & \quad 1  \\
	2 & \quad 1 + 2 \\
	3 & \quad 1 \\
	4 & \quad 1 + 2 + 4 \\
	5 & \quad 1 \\
	6 & \quad 1 + 2 \\
	7 & \quad 1 \\
	8 & \quad 1 + 2 + 4 + 8 \\
	\vdots & \quad \cdots
      \end{align*}
    \column{0.55\textwidth}
      \pause
      \[
	\sum_{i=1}^{n} c_i = \sum_{j=0}^{\lfloor \log n \rfloor} \lfloor \frac{n}{2^j} \rfloor 2^j \le n (\lfloor \log n \rfloor + 1)
      \]

      \pause
      \[
	\forall i,\; \hat{c_i} = 1 + \lfloor \log n \rfloor
      \]
  \end{columns}
\end{frame}
%%%%%%%%%%%%%%%%%%

%%%%%%%%%%%%%%%%%%
\begin{frame}{}
  \begin{exampleblock}{The Accounting Method for ``Array Merging Dictionary''}
    \[
	  \blue{\textsc{Create}: 1 \quad \textsc{Merge$(A_i, A_i)$}: 2 \cdot 2^i}
    \]
  \end{exampleblock}

  \pause
  \[
	\red{\hat{c_i} = 1 + \lfloor \log n \rfloor}
  \]

  \pause
  \fig{width = 0.40\textwidth}{figs/why-why-why}

  \pause
  \[
    \forall n,\; \sum_{i=1}^{n} a_i \geq 0
  \]
\end{frame}
%%%%%%%%%%%%%%%%%%
