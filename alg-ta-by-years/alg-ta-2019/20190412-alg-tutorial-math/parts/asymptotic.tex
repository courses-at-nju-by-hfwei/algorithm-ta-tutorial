% file: parts/asymptotic.tex

%%%%%%%%%%%%%%%
\begin{frame}{}
  \centerline{\teal{\large Asymptotics}}

  \fig{width = 0.45\textwidth}{figs/growth-order}

  \pause
  \vspace{-0.30cm}
  \[
    \red{\text{\Large $Q: \theta(f)\;?$}}
  \]
\end{frame}
%%%%%%%%%%%%%%%

%%%%%%%%%%%%%%%
\begin{frame}{}
  \[
	O(g(n)) = \Big\{f(n) \mid \red{\exists c > 0}, \blue{\exists n_0 > 0, \forall n \ge n_0}: 0 \le f(n) \le c g(n)\Big\}
  \]

  \pause
  \[
	\purple{\Big\{ \quad \Big\}}
  \]

  \pause
  \[
	\blue{\exists n_0 > 0, \forall n \ge n_0}
  \]

  \pause
  \[
	\red{\exists c > 0}
  \]
\end{frame}
%%%%%%%%%%%%%%%

%%%%%%%%%%%%%%%
\begin{frame}{}
  \[
	O(g(n)) = \Big\{f(n) \mid \red{\exists} c > 0, \exists n_0 > 0, \forall n \ge n_0: 0 \le f(n) \le c g(n)\Big\}
  \]
  \[
	\Omega(g(n)) = \Big\{f(n) \mid \red{\exists} c > 0, \exists n_0 > 0, \forall n \ge n_0: 0 \le \red{c} g(n) \le f(n) \Big\}
  \]

  \pause
  \begin{align*}
    \Theta(g(n)) = \Big\{f(n) \mid &\exists c_1 > 0, \exists c_2 >0, \exists n_0 > 0, \forall n \ge n_0: \\ 
								   & 0 \le c_1 g(n) \le f(n) \le c_2 g(n)\Big\}
  \end{align*}

  \pause
  \[
	o(g(n)) = \Big\{f(n) \mid \textcolor{red}{\forall c > 0}, \exists n_0 > 0, \forall n \ge n_0: 0 \le f(n) \;\red{<}\; c g(n)\Big\}
  \]
  \[
	\omega(g(n)) = \Big\{f(n) \mid \textcolor{red}{\forall c > 0}, \exists n_0 > 0, \forall n \ge n_0: 0 \le c g(n) \;\red{<}\; f(n)\Big\}
  \]

  \pause
  \[
	\teal{f(n) \;\red{\sim}\; g(n) \iff \lim_{n \to \infty} \frac{f(n)}{g(n)} = 1}
  \]
\end{frame}
%%%%%%%%%%%%%%%

%%%%%%%%%%%%%%%
\begin{frame}{}
  \begin{exampleblock}{Asymptotics (Problem $2.6\; (4)$)}
    \[
      f(n) = \Theta(g(n)) \iff f(n) = O(g(n)) \land f(n) = \Omega(g(n))
    \]
  \end{exampleblock}

  % \pause
  % \begin{exampleblock}{Asymptotics (Problem $2.6 (5)$)}
  %   \begin{align*}
  %     f(n) = O(g(n)) &\iff g(n) = \Omega(f(n)) \\
  %     f(n) = o(g(n)) &\iff g(n) = \omega(f(n))
  %   \end{align*}
  % \end{exampleblock}

  \pause
  \vspace{0.50cm}
  \begin{exampleblock}{Asymptotics (Problem $2.6\; (6)$)}
    \[
      \Theta(g(n)) \cap o(g(n)) = \emptyset
    \]
  \end{exampleblock}

  \pause
  \vspace{0.30cm}
  \[
    \red{Q:}\; f(n) = O(g(n)) \;\teal{\lor}\; f(n) = \Omega(g(n)) \;?
  \]

  \pause
  \[
    f(n) = n, \quad g(n) = n^{1 + \sin n}
  \]
\end{frame}
%%%%%%%%%%%%%%%

%%%%%%%%%%%%%%%
\begin{frame}{}
  \begin{exampleblock}{Asymptotics (Problem $2.7\; (2)$)}
    \[
      (\log n)^{2} \;\;\text{\it vs. }\; \sqrt{n}
    \]
  \end{exampleblock}

  \pause
  \[
    (\log n)^{c_1} = O(n^{c_2}) \quad c_1, c_2 > 0
  \]
\end{frame}
%%%%%%%%%%%%%%%

%%%%%%%%%%%%%%%
% \begin{frame}{}
%   \begin{exampleblock}{Asymptotics (Problem $2.10$)}
%     \[
%       \log(n!) = \Theta(n \log n)
%     \]
%   \end{exampleblock}
% 
%   \pause
%   \vspace{0.30cm}
%   \begin{alertblock}{Stirling Formula (by {\it James Stirling}):}
%     \begin{columns}
%       \column{0.50\textwidth}
% 	\[
% 	  n! \;\red{\sim}\; \sqrt{2 \pi n} \Big(\frac{n}{e}\Big)^{n}
% 	\]
%       \column{0.50\textwidth}
% 	\fig{width = 0.30\textwidth}{figs/stirling-formula-wiki-qrcode}
%     \end{columns}
%   \end{alertblock}
% 
%   \pause
%   \vspace{0.30cm}
%   \[
%     \log (n!) = \log 1 + \log 2 + \cdots + \log n
%   \]
% 
%   \pause
%   \[
%     \log (n!) \le n \log n \pause \qquad \log (n!) \ge \frac{n}{2} \log \frac{n}{2}
%   \]
% \end{frame}
%%%%%%%%%%%%%%%

%%%%%%%%%%%%%%%
\begin{frame}{}
  \begin{exampleblock}{Summation (Problem $2.20$)}
    \vspace{-0.30cm}
    % file: algs/conundrum.tex

\begin{algorithm}[H]
  % \caption{Conundrum.}
  \begin{algorithmic}[1]
    \Procedure{Conundrum}{$n$}
      \State $r \gets 0$

      \hStatex
      \For{$i \gets 1 \;\text{\bf to } n$}
	\For{$j \gets i+1 \;\text{\bf to } n$}
	  \For{$k \gets i+j-1 \;\text{\bf to } n$}
	    \State $r \gets r + 1$
	  \EndFor
	\EndFor
      \EndFor

      \hStatex
      \State \Return $r$
    \EndProcedure
  \end{algorithmic}
\end{algorithm}

  \end{exampleblock}

  \pause
  \[
    \teal{\sum_{i=1}^{n}}\; \blue{\sum_{j=i+1}^{n}}\; \red{\sum_{k=i+j-1}^{n}}\; 1 = \pause \purple{\frac{n^2-n}{2}} = \Theta(n^2)
  \]
  \pause
  \fig{width = 0.15\textwidth}{figs/wrong}
\end{frame}
%%%%%%%%%%%%%%%

%%%%%%%%%%%%%%%
\begin{frame}{}
  \[
    \teal{\sum_{i=1}^{n}}\; \blue{\sum_{j=i+1}^{n}}\; \red{\sum_{k=i+j-1}^{n}}\; 1 = \purple{\frac{n^2-n}{2}} = \Theta(n^2)
  \]

  \pause
  \fig{width = 0.40\textwidth}{figs/go-ok}

  \pause
  \[
    \teal{\sum_{i=1}^{n}}\; \blue{\sum_{j=i+1}^{n}}\; \red{\sum_{k=i+j-1}^{n}}\; 1 = \frac{1}{48} \Big(3 \red{\big(-1 + (-1)^n \big)} + 2n (n+2) (2n-1)\Big) = \Theta(n^3)
  \]
\end{frame}
%%%%%%%%%%%%%%%

%%%%%%%%%%%%%%%
\begin{frame}{}
  \begin{align*}
	\teal{\sum_{i=1}^{n}}\; \blue{\sum_{j=i+1}^{n}}\; &\red{\sum_{k=i+j-1}^{n}}\; 1 \\[6pt]
	\onslide<2->{&= \teal{\sum_{i=1}^{n}}\; \blue{\sum_{j=i+1}^{n}}\; (n-i-j+2)} \onslide<3->{\;\red{[j \le n-i+1, i \le \frac{n}{2}]} \\[6pt]}
	\onslide<4->{&= \teal{\sum_{i=1}^{\lfloor \frac{n}{2} \rfloor}}\; \blue{\sum_{j=i+1}^{n-i+1}}\; (n-i-j+2)}
  \end{align*}
\end{frame}
%%%%%%%%%%%%%%%

%%%%%%%%%%%%%%%
\begin{frame}{}
  \fig{width = 0.60\textwidth}{figs/sum}{\centerline{From Zheng (171860658)}}
\end{frame}
%%%%%%%%%%%%%%%

%%%%%%%%%%%%%%%
\begin{frame}{}
  \fig{width = 0.25\textwidth}{figs/knuth}

  \begin{alertblock}{Reference:}
    {\it ``Big Omicron and Big Omega and Big Theta''} by Donald E. Knuth, 1976.
  \end{alertblock}
\end{frame}
%%%%%%%%%%%%%%%
