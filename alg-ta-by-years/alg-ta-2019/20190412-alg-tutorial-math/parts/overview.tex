% overview.tex

%%%%%%%%%%%%%%%
\begin{frame}
  \fig{width = 0.50\textwidth}{figs/little-math}
\end{frame}
%%%%%%%%%%%%%%%

%%%%%%%%%%%%%%%
\begin{frame}
  \begin{figure}
	\centering
	\begin{subfigure}[b]{0.45\textwidth}
	  \centering
	  \includegraphics[width = 0.50\textwidth]{figs/AoA} \vspace{-0.30cm}
	  \[
		A(n)
	  \]
	\end{subfigure}
	\hfill
	\begin{subfigure}[b]{0.45\textwidth}  
	  \centering 
	  \includegraphics[width = 0.70\textwidth]{figs/domino} \vspace{-0.30cm}
	  \[
		\text{Mathematical Induction}
	  \]
	\end{subfigure}
	\vskip\baselineskip
	\begin{subfigure}[b]{0.45\textwidth}   
	  \centering 
	  \includegraphics[width = 0.60\textwidth]{figs/growth-order} \vspace{-0.30cm}
	  \[
		\Omega, \Theta, O
	  \]
	\end{subfigure}
	\hfill
	\begin{subfigure}[b]{0.45\textwidth}   
	  \centering 
	  \includegraphics[width = 0.70\textwidth]{figs/fractal-tree} \vspace{-0.30cm}
	  \[
		T(n) = a T(n/b) + f(n)
	  \]
	\end{subfigure}
  \end{figure}
\end{frame}
%%%%%%%%%%%%%%%
