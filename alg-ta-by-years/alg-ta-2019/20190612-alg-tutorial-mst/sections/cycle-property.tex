% file: sections/cycle-property.tex

%%%%%%%%%%%%
\begin{frame}{}
  \centerline{\teal{\Large Cycle Property}}
\end{frame}
%%%%%%%%%%%%%

%%%%%%%%%%%%%
\begin{frame}{}
  \begin{block}{Cycle Property \pno{$10.19 (b)$}}
    \begin{itemize}
      \item Let $C$ be any cycle in $G$
      \item Let $e = (u,v)$ be \red{\it a} maximum-weight edge in $C$
    \end{itemize}

    \centerline{Then $\red{\exists} \text{ MST } T \text{ of } G: e \notin T$.}
  \end{block}

  \fig{width = 0.50\textwidth}{figs/cycle-property.pdf}

  \pause
  \vspace{-0.30cm}
  \[
    T' = \underbrace{T - \set{e}}_{\red{\text{if } e \;\in\; T}} + \set{e'}
  \]

  \pause
  \centerline{``a'' $\to$ ``the'' \red{$\implies$} ``$\exists$'' $\to$ ``$\forall$''}
\end{frame}
%%%%%%%%%%%%%

%%%%%%%%%%%%%
\begin{frame}<presentation:0>[noframenumbering]
  \begin{proof}
    Basic idea: pick any MST $T$ of $G$
    \begin{itemize}
      \item $e \notin T$
      \item $e \in T \Rightarrow e \notin T'$
	\begin{itemize}
	  \item $T - \set{e}$ $\Rightarrow$ $(S, V \setminus S)$
	  \item $\exists e' = (u', v') \in C$ ($e' \in P_{u,v}$) across the cut
	  \item \textcolor{red}{$T' = T - \set{e} + \set{e'}$}: spanning tree
	  \item $w(e') \leq w(e) \Rightarrow w(T') \leq w(T) \Rightarrow w(T') = w(T)$
	\end{itemize}
    \end{itemize}
  \end{proof}
\end{frame}
%%%%%%%%%%%%%

%%%%%%%%%%%%%
\begin{frame}{}
  \begin{exampleblock}{Anti-Kruskal algorithm \pno{$10.19\; (c)$}}
	\begin{center}
	  {\href{https://en.wikipedia.org/wiki/Reverse-delete\_algorithm}{Reverse-delete algorithm \teal{\small (wiki; clickable)}}} \\[3pt]
	  \purple{Delete an edge if this does not disconnect the graph.}
	\end{center}
  \end{exampleblock}

  \pause
  \[
    O\Big(m \log n\; (\log \log n)^3\Big)
  \]

  \pause
  \begin{proof}
    \centerline{\purple{Cycle Property}}
    \pause
    \[
      \red{T \subseteq F \implies \exists\; T': T' \subseteq F - \set{e}}
    \]
  \end{proof}
  % \begin{theorem}[Invariant for Correctness]
  %   \begin{center}
  %     If $F$ is the set of edges remained at the end of the while loop, \\
  %     then there is some \text{MST} that are a subset of $F$.
  %   \end{center}
  % \end{theorem}

  \pause
  \vspace{0.20cm}
  \begin{center}
    {\it ``On the Shortest Spanning Subtree of a Graph \\
    and the Traveling Salesman Problem''} \\
    \hfill --- \red{Kruskal}, $1956$.
  \end{center}
\end{frame}
%%%%%%%%%%%%%

%%%%%%%%%%%%%
\begin{frame}{}
  \begin{exampleblock}{Application of Cycle Property \pno{$10.15\; (1)$}}
    \[
      G = (V, E), \quad \red{|E| > |V| - 1}
    \]
    \[
      e: \text{the unique maximum-weighted edge of } G
    \]
    \[
      \red{\implies^{?}} 
    \]
    \[
      e \notin \text{ any MST}
    \]
  \end{exampleblock}

  \pause
  \vspace{0.60cm}
  \centerline{\large Bridge}
\end{frame}
%%%%%%%%%%%%%

%%%%%%%%%%%%%
\begin{frame}{}
  \begin{exampleblock}{Application of Cycle Property \pno{$10.15\; (2)$}}
    \[
      C \subseteq G, \quad \red{e \in C}
    \]
    \[
      e: \text{ the unique maximum-weighted edge of } G
    \]
    \[
      \implies
    \]
    \[
      e \notin \text{ any MST}
    \]
  \end{exampleblock}

  \pause
  \vspace{0.50cm}
  \centerline{\teal{Cycle Property}}
\end{frame}
%%%%%%%%%%%%%

%%%%%%%%%%%%%
\begin{frame}{}
  \begin{exampleblock}{Application of Cycle Property \pno{$10.15\; (5)$}}
    \[
      C \subseteq G, \quad e \in C
    \]
    \[
      e: \text{ the unique lightest edge of } C
    \]
    \[
      \red{\implies^{?}}
    \]
    \[
      e \in \forall \text{ MST}
    \]
  \end{exampleblock}

  \pause
  \vspace{0.50cm}
  \fig{width = 0.30\textwidth}{figs/cycle-property-inverse.pdf}
\end{frame}
%%%%%%%%%%%%%
