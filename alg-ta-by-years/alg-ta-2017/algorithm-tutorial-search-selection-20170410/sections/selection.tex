\section{Selection}

%%%%%%%%%%%%%%%%%%%%
\begin{frame}{The $3$rd largest element (Problem 3.1)}
  \[
	V_k(n): \text{$\min$ \#comparisons to find the $k$-th largest element of $n$ elements.}
  \]

  \pause
  \begin{align*}
	V_1(n) &= n - 1 \\
	V_2(n) &= (n - 1) + (\lceil \log n \rceil - 1) \\
  \end{align*}

  \pause
  \[
	V_3(n) = \;?
  \]

  \pause
  \[
	V_3(n) \le (n - 1) + (\lceil \log n \rceil - 1) + \textcolor{red}{(n - 3)}
  \]

  \pause
  \[
	V_3(n) \le (n - 1) + (\lceil \log n \rceil - 1) + \textcolor{red}{(\lceil \log n \rceil - 1)} 
  \]
\end{frame}
%%%%%%%%%%%%%%%%%%%%
\begin{frame}{The $3$rd largest element (Problem 3.1)}
  \centerline{``$Q_1$: What is the exact value of $V_3(n)$?''}

  \begin{theorem}[$V_3(n)$]
	$n \ge 6, n = 2^k + r (0 \le r < 2^k)$:
	\begin{equation*}
	  V_3(n) = \begin{cases}
		(n - 3) + 2 k 	  & r = 0, 1 \\
		(n - 3) + 2 k + 1 & 2 \le r \le 2^{k -2} + 1 \\
		(n - 3) + 2 k + 2 & \text{o.w.} \\
	  \end{cases}
	\end{equation*}
  \end{theorem}

  \begin{alertblock}{References}
	``Selecting the Top Three Elements'' by Aigner, 1982.
  \end{alertblock}

  \pause
  \begin{alertblock}{Reference}
	``The Art of Computer Programming, Vol 3: Sorting and Searching (Section 5.3.3)'' by Donald E. Knuth.
  \end{alertblock}
\end{frame}
%%%%%%%%%%%%%%%%%%%%
\begin{frame}{The $3$rd largest element (Problem 3.1)}
  \begin{center}
	``$Q_2$: Does your algorithm need to find the $1$st and the $2$nd elements?'' \\[0.30cm] \pause

	``YES!''
  \end{center}

  \pause
  \begin{center}
	``$Q_3$: Do all algorithms have to find the $1$st and the $2$nd elements?'' \\[0.30cm] \pause

	``NO!''
  \end{center}

  \pause
  \begin{alertblock}{References}
	``Selecting the Top Three Elements'' by Aigner, 1982.
  \end{alertblock}

  % \pause
  % \begin{description}
  %   \item[$V_t(n)$:] 
  %   \item[$W_t(n)$:] 
  %   \item[$U_t(n)$:] 
  % \end{description}

  % \pause
  % \begin{align*}
  %   U_1(n) &= V_1(n) = W_1(n) = n - 1 \\
  %   W_2(n) &= V_2(n) = n + \lceil \log n \rceil - 2
  % \end{align*}
\end{frame}
%%%%%%%%%%%%%%%%%%%%
\begin{frame}{Selection with minimum \#comparisons (Problem 3.2)}
  Selecting the median of $5$ elements using $6$ comparisons.

  \fignocaption{width = 0.60\textwidth}{figs/median5.jpg}
\end{frame}
%%%%%%%%%%%%%%%%%%%%
\begin{frame}{Sorting with minimum \#comparisons (Problem 2.4)}
  Sorting $5$ elements using $7$ comparisons.

  \[
    S(5) = 7
  \]

  \pause
  \begin{alertblock}{Reference}
	``The Art of Computer Programming, Vol 3: Sorting and Searching (Section 5.3.1)'' by Donald E. Knuth.
  \end{alertblock}

  \[
	S(21) = 66
  \]
\end{frame}
%%%%%%%%%%%%%%%%%%%%
\begin{frame}{Sorting with minimum \#comparisons (Problem 2.4)}
  Sorting $5$ elements using $7$ comparisons.
  \fignocaption{width = 0.60\textwidth}{figs/median5.jpg}
\end{frame}
%%%%%%%%%%%%%%%%%%%%
\begin{frame}{Medians of sorted arrays (Problem 3.7)}
  \centerline{\url{http://cs.stackexchange.com/a/33129/4911}}
\end{frame}
%%%%%%%%%%%%%%%%%%%%
% \begin{frame}{The largest $k$ elements (Problem 3.5)}
% \end{frame}
%%%%%%%%%%%%%%%%%%%%
% \begin{frame}{Close to median (Problem 3.6)}
% \end{frame}
% %%%%%%%%%%%%%%%%%%%%
% \begin{frame}{Dynamic median (Problem 3.8)}
% \end{frame}
% %%%%%%%%%%%%%%%%%%%%
% \begin{frame}{Weighted median (Problem 3.9)}
% \end{frame}
%%%%%%%%%%%%%%%%%%%%

