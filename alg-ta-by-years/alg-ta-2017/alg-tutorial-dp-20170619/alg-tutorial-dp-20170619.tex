\documentclass[]{beamer}
\usepackage{lmodern}

\usetheme{CambridgeUS} % try Madrid, Pittsburgh
\usecolortheme{beaver}
\usefonttheme[onlymath]{serif} % try "professionalfonts"

\setbeamertemplate{itemize items}[default]
\setbeamertemplate{enumerate items}[default]

\usepackage{amsmath, amsfonts, latexsym, mathtools, extarrows}
\usepackage{centernot} % for ``not implies'' symbol
\DeclareMathOperator*{\argmin}{arg\,min}
\DeclareMathOperator*{\argmax}{arg\,max}

\usepackage{pifont}
\usepackage{hyperref}
\usepackage{comment}

\usepackage[normalem]{ulem}
\newcommand{\middlewave}[1]{\raisebox{0.5em}{\uwave{\hspace{#1}}}}

\usepackage{graphicx, subcaption}

\usepackage{algorithm}
\usepackage[noend]{algpseudocode}

\newcommand{\pno}[1]{\textcolor{blue}{\scriptsize [Problem: #1]}}
\newcommand{\set}[1]{\{#1\}}

\newcommand{\cmark}{\textcolor{red}{\ding{51}}}
\newcommand{\xmark}{\textcolor{red}{\ding{55}}}
%%%%%%%%%%%%%%%%%%%%%%%%%%%%%%%%%%%%%%%%%%%%%%%%%%%%%%%%%%%%%%
% for fig without caption: #1: width/size; #2: fig file
\newcommand{\fignocaption}[2]{
  \begin{figure}[htp]
    \centering
      \includegraphics[#1]{#2}
  \end{figure}
}

% for fig with caption: #1: width/size; #2: fig file; #3: fig caption
\newcommand{\fig}[3]{
  \begin{figure}[htp]
    \centering
      \includegraphics[#1]{#2}
      \caption[labelInTOC]{#3}
  \end{figure}
}

\newcommand{\titletext}{Dynamic Programming}

\newcommand{\tx}{%
  \section*{}
  \begin{frame}[noframenumbering]
	\fignocaption{width = 0.50\textwidth}{figs/thankyou.jpg}
  \end{frame}
}
%%%%%%%%%%%%%%%%%%%%
\title[\titletext]{\titletext}
\subtitle{}

\author[Hengfeng Wei]{Hengfeng Wei}
% \titlegraphic{\includegraphics[height = 1.5cm]{figs/qrcode-alg-ta-dp-20170612.png}}
\institute{hfwei@nju.edu.cn}
\date{June 11 -- June 19, 2017}

\AtBeginSection[]{
  \begin{frame}[noframenumbering, plain]
    \frametitle{\titletext}
    \tableofcontents[currentsection, sectionstyle=show/shaded, subsectionstyle=show/show/hide]
  \end{frame}
}

%%%%%%%%%%
\begin{document}
\maketitle

\section{Overview}

%%%%%%%%%%%%%%%%%%%%
\begin{frame}{What is DP?}
  \begin{center}
	DP $\approx$ ``\only<2->{smarter} brute force'' \\[8pt]
	\uncover<3->{DP $\approx$ ``smart scheduling of subproblems''} \\[8pt]
    \uncover<4->{DP $\approx$ ``shortest/longest paths in some DAG''}
  \end{center}
\end{frame}
%%%%%%%%%%%%%%%%%%%%
\begin{frame}{What is not DP?}
  \begin{center}
	Programming $=$ Planning \\[15pt] \pause
	Programming $\neq$ Coding \\[5pt]
	(Richard Bellman, 1940s)
  \end{center}
\end{frame}
%%%%%%%%%%%%%%%%%%%%
\begin{frame}{Steps for applying DP}
  \begin{enumerate}
	\item Define subproblems
	  \begin{itemize}
		\item \# of subproblems
	  \end{itemize}
	\item Set the goal
	\item Define the recurrence
	  \begin{itemize}
		\item larger subproblem $\gets$ \# smaller subproblems
		\item init. conditions
	  \end{itemize}
	  \pause
	\item Write pseudo-code: fill ``table'' in topo. order
	\item Analyze the time complexity
	\item Extract the optimal sulution
  \end{enumerate}
\end{frame}
%%%%%%%%%%%%%%%%%%%%
\begin{frame}{Common subproblems in DP}
  1D subproblems:
  \begin{description}
	\item[Input:] $x_1, x_2, \dots, x_n$ (array, sequence, string)
	\item[Subproblems:] $x_1, x_2, \dots, x_i$ (prefix/suffix)
	\item[\#:] $\Theta(n)$
	\item[Examples:] Maximum-sum subarray, Longest increasing subsequence, Text justification (\LaTeX)
  \end{description}
\end{frame}
%%%%%%%%%%%%%%%%%%%%
\begin{frame}{Common subproblems in DP}
  2D subproblems:
  \begin{enumerate}
	\item Input: $x_{1}, x_{2}, \dots, x_{m}; \quad y_{1}, y_{2}, \dots, y_{n}$
	  \begin{description}
		\item[Subproblems:] $x_{1}, x_{2}, \dots, x_{i}; \quad y_{1}, y_{2}, \dots, y_{j}$
		\item[\#:] $\Theta(mn)$
		\item[Examples:] Edit distance, Longest common subsequence
	  \end{description}
	  \pause
	  \vspace{0.30cm}
	\item Input: $x_{1}, x_{2}, \dots, x_{n}$
	  \begin{description}
		\item[Subproblems:] $x_{i}, \dots, x_{j}$
		\item[\#:] $\Theta(n^{2})$
		\item[Examples:] Matrix chain multiplication, Optimal BST
	  \end{description}
  \end{enumerate}
\end{frame}
%%%%%%%%%%%%%%%%%%%%
\begin{frame}{Common subproblems in DP}
  3D subproblems:
  \begin{itemize}
	\item Floyd-Warshall algorithm
	  \[
		\text{d}(i,j,k) = \min \set{\text{d}(i,j,k-1), \text{d}(i,k,k-1) + \text{d}(k,j,k-1)}
	  \]
  \end{itemize}

  \pause
  DP on graphs:
  \begin{enumerate}
	\item On rooted tree
	  \begin{description}
		\item[Subproblems:] rooted subtrees
	  \end{description}
	\item On DAG
	  \begin{description}
		\item[Subproblems:] nodes after/before in the topo. order
	  \end{description}
  \end{enumerate}

  \pause
  \vspace{0.30cm}
  Knapsack problem:
  \begin{itemize}
	\item Subset sum problem, change-making problem
  \end{itemize}
\end{frame}
%%%%%%%%%%%%%%%%%%%%
\begin{frame}{Common subproblems in DP}
  \centerline{\huge And Others $\dots$}
\end{frame}
%%%%%%%%%%%%%%%%%%%%
\begin{frame}{Recurrences in DP}
  Make choices by asking yourself the right question:
  \begin{enumerate}
	\item Binary choice
	  \begin{itemize}
		\item whether $\dots$
	  \end{itemize}
	\item Multi-way choices
	  \begin{itemize}
		\item where to $\dots$ 
		\item which one $\dots$
	  \end{itemize}
  \end{enumerate}
\end{frame}
%%%%%%%%%%%%%%%%%%%%

\section{1D DP}

%%%%%%%%%%%%%%%%%%%%
\begin{frame}{$f^{(S(n))} = 1$}
  \begin{exampleblock}{$f^{(S(n))} = 1$ (Problem 7.2)}
	\begin{displaymath}
	  f(n) = \begin{cases}
		n - 1 & \text{if } n \in \mathbb{Z}^{+} \\
		n / 2 & \text{if } n \% 2 = 0 \\
		n / 3 & \text{if } n \% 3 = 0
	  \end{cases}
	\end{displaymath}

	\centerline{$S(n):$ minimum number of steps taking $n$ to 1.}
  \end{exampleblock}

  \pause
  \[
	S(i): \text{minimum number of steps taking } i \text{ to } 1
  \]

  \pause
  \[
	S(i) = 1 + \min \set{N(i-1), N(i/2) (\text{if } n\%2 = 0), N(i/3) (\text{if } n\%3 = 0)}
  \]

  \[
	S(1) = 0
  \]
\end{frame}
%%%%%%%%%%%%%%%%%%%%
\begin{frame}{$f^{(S(n))} = 1$}
  Collatz ($3n+1$) conjecture:
  \begin{displaymath}
	f(n) = \begin{cases}
	  n / 2 & \text{if } n \% 2 = 0 \\
	  3n + 1 & \text{if } n \% 2 = 1
	\end{cases}
  \end{displaymath}

  \[
	f^{\ast}(n) = 1 ?
  \]

  \pause
  \vspace{0.50cm}
  \begin{quote}
	``Mathematics may not be ready for such problems.''\\
	\hfill --- Paul Erdős
  \end{quote}
\end{frame}
%%%%%%%%%%%%%%%%%%%%
\begin{frame}{Longest increasing subsequence}
  \begin{exampleblock}{Longest increasing subsequence (Problem 7.3)}
    \begin{itemize}
      \item Given an integer array $A[1 \ldots n]$
      \item To find (the length of) a longest increasing subseqence.
    \end{itemize}
  \end{exampleblock}

  \[
    5,2,8,6,3,6,9,7 \implies 2, 3, 6, 9
  \]
\end{frame}
%%%%%%%%%%%%%%%%%%%%
\begin{frame}{Longest increasing subsequence}
  \begin{description}
	\item[Subproblem:] $L(i):$ the length of the LIS of $A[1 \ldots i]$
	\item[Goal:] $L(n)$
	  \pause
	\item[Make choice:] whether $A[i] \in LIS[1 \ldots i]$?
	\item[Recurrence:] 
	  \[
		L(i) = \max \set{L(i-1), 1 + \max_{j < i \land A[j] \le A[i]} L(j)}
	  \]
	  \pause
	\item[Init:]
	  \[
		L(0) = 0
	  \]
	\item[Time:] 
	  \[
		O(n^2) = \Theta(n) \cdot O(n)
	  \]
  \end{description}
\end{frame}
%%%%%%%%%%%%%%%%%%%%
\begin{frame}{Longest increasing subsequence}
  \fignocaption{width = 0.40\textwidth}{figs/LIS-example.png}

  \centerline{Longest path distance in the DAG!}
\end{frame}
%%%%%%%%%%%%%%%%%%%%
\begin{frame}{Maximum-sum subarray}
  \begin{exampleblock}{Maximum-sum subarray (Google Interview)}
    \begin{itemize}
      \item Array $A[1 \cdots n], a_{i} >=< 0$
      \item To find (the sum of) a maximum-sum subarray of $A$
		\begin{itemize}
		  \item $\text{mss} = 0$ if all negative
		\end{itemize}
    \end{itemize}
	
    \[
      A[-2,1 ,-3,4,-1,2,1,-5,4] \implies [4,-1,2,1]
    \]
  \end{exampleblock}

  \pause
  \begin{description}
	\item[Subproblem:] $\text{MSS}[i]$: the sum of the $\text{MS}[i]$ of $A[1 \cdots i]$
	\item[Goal:] $\text{mss} = \text{MSS}[n]$
	\pause
	\item[Make choice:] Is $a_{i} \in \text{MS}[i]$?
	\item[Recurrence:]
	  \[ 
		\text{MSS}[i] = \max \set{\text{MSS}[i-1], \textcolor{red}{???}}
	  \]
  \end{description}
\end{frame}
%%%%%%%%%%%%%%%%%%%%
\begin{frame}{Maximum-sum subarray}
  \begin{description}
	\item[Subproblem:] $\text{MSS}[i]$: the sum of the $\text{MS}[i]$ \textcolor{red}{\it ending with} $a_{i}$ or 0
	\item[Goal:] $\text{mss} = \max_{1 \le i \le n} \text{MSS}[i]$
	\pause
	\item[Make choice:] where does the $\text{MS}[i]$ start?
	\item[Recurrence:]
	  \[ 
		\text{MSS}[i] = \max \set{\text{MSS}[i-1] + a_{i}, 0} \text{ \textcolor{red}{(prove it!)}}
	  \]
	\pause
	\item[Init:]
	  \[
		\text{MSS}[0] = 0
	  \]
	  \pause
	\item[Time:] $\Theta(n)$
  \end{description}
\end{frame}
%%%%%%%%%%%%%%%%%%%%
\begin{frame}{Maximum-sum subarray}
%  \begin{block}{Code.}
%    \begin{verbatim}
%      MSS[0] = 0
%      For i = 1 to n
%        MSS[i] = max{MSS[i-1] + A[i], 0}
%      return max_{i = 1 to n} MSS[i]
%    \end{verbatim}
%  \end{block}
%
%  \begin{block}{Simpler code.}
%    \begin{verbatim}
%      mss = 0
%      MSS = 0
%      For i = 1 to n
%        MSS = max{MSS + A[i], 0}
%        mss = max{mss, MSS}
%      return mss
%    \end{verbatim}
%  \end{block}
\end{frame}
%%%%%%%%%%%%%%%%%%%%
\begin{frame}{Maximum-product subarray}
  \begin{exampleblock}{Maximum-product subarray (Problem 7.4)}
  \end{exampleblock}
\end{frame}
%%%%%%%%%%%%%%%%%%%%
\begin{frame}{Reconstructing string}
  \begin{exampleblock}{Reconstructing string (Problem 7.9)}
    \begin{itemize}
      \item String $S[1 \cdots n]$
      \item Dict for \emph{lookup}:
		\begin{displaymath}
		  \text{dict}(w) = \left\{ \begin{array}{ll}
			\text{true} & \textrm{if } w \textrm{ is a valid word}\\
			\text{false} & \textrm{o.w.}
		  \end{array} \right.
		\end{displaymath}
	  \item Is $S[1 \cdots n]$ valid (reconstructed as a sequence of valid words)?
    \end{itemize}
  \end{exampleblock}
\end{frame}
%%%%%%%%%%%%%%%%%%%%
\begin{frame}{Reconstructing string}
  \begin{description}
	\item[Subproblem:] $V[i]$: is $S[1 \cdots i]$ valid?
	\item[Goal:] $V[n]$
	\pause
	\item[Make choice:] where does the last word start?
	\item[Recurrence:] 
	  \[ 
		V[i] = \bigvee_{j = 1 \ldots i} (V[j-1] \land \text{dict}(S[j \cdots i]))
	  \]
	\pause
	\item[Init:]
	  \[
		V[0] = \text{true}
	  \]
	\item[Time:] $O(n^2)$
  \end{description}
\end{frame}
%%%%%%%%%%%%%%%%%%%%
\begin{frame}{Hotel along a trip}
  \begin{exampleblock}{Hotel along a trip (Problem 7.15)}
    \begin{itemize}
      \item Hotel sequence (distance): $a_{0} = 0, a_{1}, \cdots, a_{n}$
      \item $a_0 \leadsto a_n$ 
	  \item Stop at only hotels
      \item Cost: $(200 - x)^{2}$ 
      \item To minimize overall cost
    \end{itemize}
  \end{exampleblock}
\end{frame}
%%%%%%%%%%%%%%%%%%%%
\begin{frame}{Hotel along a trip}
  \begin{description}
	\item[Subproblem:] $C[i]$: minimum cost when the destination is $a_{i}$
	\item[Goal:] $C[n]$
	\pause
	\item[Make choice:] what is the last but one hotel $a_{j}$ to stop?
	\item[Recurrence:] 
	  \[
		C[i] = \min_{0 \le j < i} \set{C[j] + (200 - (a_{i} - a_{j}))^{2}}
	  \]
	\pause
	\item[Init:]
	  \[
		C[0] = 0
	  \]
	\item[Time:] 
	  \[
		O(n^2) = \Theta(n) \cdot O(n)
	  \]
  \end{description}
\end{frame}
%%%%%%%%%%%%%%%%%%%%
\begin{frame}{Highway restaurants}
  \begin{exampleblock}{Highway restaurants (Problem 7.16)}
	\begin{itemize}
	  \item Locations: $L[1 \dots n]$
	  \item Profits: $P[1 \dots n]$
	  \item Any two hotels should be $\ge k$ miles apart
	  \item To maximize the total profit
	\end{itemize}
  \end{exampleblock}

  \begin{description}
	\item[Subproblem:]
	\item[Goal:] 
	\item[Make choice:] 
	\item[Recurrence:] 
	\item[Time:] 
  \end{description}
\end{frame}
%%%%%%%%%%%%%%%%%%%%
\begin{frame}{Weighted interval/class scheduling}
  \begin{exampleblock}{Weighted interval/class scheduling (Problem 7.14)}
    \begin{itemize}
      \item Classes: $\mathcal{C} = \set{c_{1}, c_{2}, \cdots, c_{n}} \quad c_{i} \triangleq \langle g_{i}, s_{i}, f_{i} \rangle$
      \item Choosing non-conflicting classes to maximize your grades
    \end{itemize}
  \end{exampleblock}

  \fignocaption{width = 0.40\textwidth}{figs/weighted-interval.png}
  \centerline{\textcolor{red}{sort $\mathcal{C}$ by finishing time.}}
\end{frame}
%%%%%%%%%%%%%%%%%%%%
\begin{frame}{Weighted interval/class scheduling}
  \centerline{Greedy algorithms by finishing time or weights fail.}
\end{frame}
%%%%%%%%%%%%%%%%%%%%
\begin{frame}{Weighted interval/class scheduling}
  \begin{description}
	\item[Subproblem:] $G[i]$: the maximal grades obtained from $\set{c_{1}, c_{2}, \cdots, c_{i}}$
	\item[Goal:] $G[n]$
	  \pause
	\item[Make choice:] choose $c_{i}$ or not in $G[i]$?  
	\item[Recurrence:] 
	  \[
		G[i] = \max \set{G[i-1], G[p(i)] + g_{i}}
	  \]

	  $p(i)$: the largest index $j < i$ \emph{s.t.} $c_{i}$ and $c_{j}$ are disjoint
	  \pause
	\item[Init:]
	  \[
		G[0] = 0
	  \]
	  \pause
	\item[Time:] $O(n \log n) + T(p(i)) + O(n) \cdot O(1)$
  \end{description}
\end{frame}
%%%%%%%%%%%%%%%%%%%%
\begin{frame}{Weighted interval/class scheduling}
  Why is ordering necessary?
  \fignocaption{width = 0.50\textwidth}{figs/weighted-interval-unordered.pdf}

  \[
	G[7] = \max \set{G[6], G[\set{1,3,5}] + g_{7}}
  \]

  \begin{center}
	subproblems changed: all $O(2^{n})$ subsets
  \end{center}
\end{frame}
%%%%%%%%%%%%%%%%%%%%
\begin{frame}{Weighted interval/class scheduling}
  What about sorting by starting time?
  \fignocaption{width = 0.50\textwidth}{figs/weighted-interval-starting-time-ordered.pdf}

  \[
	G[6] = \max \set{G[5], G[\set{2,3}] + g_{6}}
  \]

  \begin{center}
	subproblems changed: all $O(2^{n})$ subsets
  \end{center}
\end{frame}
%%%%%%%%%%%%%%%%%%%%
\begin{frame}{}
  \begin{description}
	\item[Subproblem:]
	\item[Goal:] 
	  \pause
	\item[Make choice:] 
	\item[Recurrence:] 
	  \pause
	\item[Init:]
	\item[Time:] 
  \end{description}
\end{frame}
%%%%%%%%%%%%%%%%%%%%

\section{2-D DP}

\subsection{2-D DP (part 1)}
%%%%%%%%%%
\begin{frame}{2-D DP (part 1)}
  \begin{exampleblock}{LCS: longest common subsequence \pno{2.2.7}}
    \begin{itemize}
      \item $X = X_{1} \cdots X_{m}; Y = Y_{1} \cdots Y_{n}$
      \item find (the length of) an LCS of $X$ and $Y$
    \end{itemize}
    \begin{align*}
      X &= \langle A,\textcolor{blue}{B},\textcolor{blue}{C},\textcolor{blue}{B},D,\textcolor{blue}{A},B \rangle  \\
      Y &= \langle \textcolor{blue}{B},D,\textcolor{blue}{C},A,\textcolor{blue}{B},\textcolor{blue}{A} \rangle \\
      Z &= \langle B,C,B,A \rangle
    \end{align*}
  \end{exampleblock}

  \begin{block}{Solution.}
    \begin{itemize}
      \item subproblem: $L[i,j]$: the length of an LCS of $X[1 \cdots i]$ and $Y[1 \cdots j]$
      \item goal: $L[m,n]$
    \end{itemize}
  \end{block}
\end{frame}
%%%%%%%%%%
\begin{frame}{2-D DP (part 1)}
  \begin{block}{Solution.}
    \begin{itemize}
      \item question: Is $X_{i} = Y_{j}$?
      \item recurrence:
	\begin{displaymath}
	  L[i,j] = \left\{ \begin{array}{ll}
	    L[i-1, j-1] + 1 & \textrm{if $X_{i} = Y_{j}$}\\
	    \max \set{L[i-1,j], L[i,j-1]} & \textrm{if $X_{i} \neq Y_{j}$}
	  \end{array} \right.
	\end{displaymath}
      \item<2-> initialization:
	\begin{align*}
	  L[i,0] &= 0, 0 \le i \le m \\
	  L[0,j] &= 0, 0 \le j \le n
	\end{align*}
    \end{itemize}
  \end{block}

  \uncover<3->{
  \begin{center}
    \textcolor{red}{It {\it may be} correct. But I feel quite uncomfortable without a proof.}
  \end{center}
  }
\end{frame}
%%%%%%%%%%
\begin{frame}{2-D DP (part 1)}
  \begin{alertblock}{Counterexample?}
    \[
      L[i,j] = L[i-1,j-1] + 1 \text{ if } X_{i} = Y_{j}
    \]

    \begin{columns}
      \column{0.50\textwidth}
	\begin{align*}
	  X &= \textcolor{blue}{a},\textcolor{blue}{b},c,c,\textcolor{red}{c} \\
	  Y &= \textcolor{blue}{a},\textcolor{blue}{b},\textcolor{red}{c}  \\
	  Z &= \textcolor{blue}{a},\textcolor{blue}{b},\textcolor{red}{c}
	\end{align*}
      \column{0.50\textwidth}
	\begin{align*}
	  X &= \textcolor{blue}{a},\textcolor{blue}{b},\textcolor{blue}{c},c,c \\
	  Y &= \textcolor{blue}{a},\textcolor{blue}{b},\textcolor{blue}{c}  \\
	  Z &= \textcolor{blue}{a},\textcolor{blue}{b},\textcolor{blue}{c}
	\end{align*}
    \end{columns}
  \end{alertblock}
\end{frame}
%%%%%%%%%%
\begin{frame}{2-D DP (part 1)}
  \begin{block}{Correctness proof (I).}
    \begin{theorem}{}
      $L[i,j] = L[i-1,j-1] + 1$ if $X_{i} = Y_{j}$.
    \end{theorem}

    \begin{theorem}{}
      $Z[1 \cdots k]$ with $\textcolor{blue}{Z_{k} \equiv X_{i} \land Z_{k} \equiv Y_{j}}$ \emph{is} an LCS of $X[1 \cdots i]$ and $Y[1 \cdots j]$.
    \end{theorem}
    \begin{proof}
      \begin{enumerate}
	\item $Z_{k} = X_{i} = Y_{j}$ (by contradiction)
	% \item But, $Z_{k} = X_{i} \nRightarrow Z_{k} \equiv X_{i}; Z_{k} = Y_{i} \nRightarrow Z_{k} \equiv Y_{i}$ 
	\item $Z_{k} = X_{i} = Y_{j} \Rightarrow \text{ either } Z_{k} \equiv X_{i} \text{ or } Z_{k} \equiv Y_{j}$ (by contradiction)
	  \begin{enumerate}
	    \item $Z_{k} \equiv X_{i} \land Z_{k} \equiv Y_{j}$
	    \item $Z_{k} \not\equiv X_{i} \land Z_{k} \equiv Y_{j}$
	    \item $Z_{k} \equiv X_{i} \land Z_{k} \not\equiv Y_{j}$
	  \end{enumerate}
      \end{enumerate}
    \end{proof}
  \end{block}
\end{frame}
%%%%%%%%%%
\begin{frame}{2-D DP (part 1)}
  \begin{block}{Correctness proof (II).}
    \begin{theorem}{}
      $L[i,j] = \max \set{L[i-1,j], L[i,j-1]} \text{ if } X_{i} \neq Y_{j}$
    \end{theorem}

    \begin{theorem}{}
      If $X_{i} \neq Y_{j}$, then either $X_{i} \notin \text{LCS}[i,j]$ or $Y_{j} \notin \text{LCS}[i,j]$.
    \end{theorem}
    \begin{proof}
      By contradiction.
    \end{proof}
  \end{block}
\end{frame}
%%%%%%%%%%
\begin{frame}<beamer:0>{2-D DP (part 1)}
  \begin{exampleblock}{LCS with repetition of $X_{i}$ \pno{2.2.8}}
    \begin{enumerate}
      \item repetition of $X_{i}$
      \item $k$-bounded repetition of $X_{i}$
    \end{enumerate}
  \end{exampleblock}

  \begin{block}{Solution.}
    \begin{enumerate}
      \item repetition of $x_{i}$:
	\begin{displaymath}
	  L[i,j] = \left\{ \begin{array}{ll}
	    L[\textcolor{red}{i}, j-1] + 1 & \textrm{if $X_{i} = Y_{j}$}\\
	    \max \set{L[i-1,j], L[i,j-1]} & \textrm{if $X_{i} \neq Y_{j}$}
	  \end{array} \right.
	\end{displaymath}
      \item $k$-bounded repetition of $X_{i}$:

	$X^{(k)} = X_{1}^{(k)} \cdots X_{m}^{(k)}$
    \end{enumerate}
  \end{block}
\end{frame}
%%%%%%%%%%
\begin{frame}<beamer:0>{2-D DP (part 1)}
  \begin{exampleblock}{Shortest common supersequence \pno{2.2.10}}
    \begin{itemize}
      \item $X = \set{x_{1} \cdots x_{m}}; Y = \set{y_{1} \cdots y_{n}}$
      \item to find (the length of) a SCS of $X$ and $Y$
    \end{itemize}
  \end{exampleblock}

  \begin{block}{Solution.}
    \begin{itemize}
      \item subproblem $L[i,j]$: the length of an SCS of $X[1 \cdots i]$ and $Y[1 \cdots j]$
      \item goal: $L[m,n]$
      \item question: is $X_{i} = Y_{j}$
      \item recurrence:
	\begin{displaymath}
	  L[i,j] = \left\{ \begin{array}{ll}
	    L[i-1, j-1] + 1 & \textrm{if $X_{i} = Y_{j}$}\\
	    \max \set{L[i-1,j] + 1, L[i,j-1] + 1} & \textrm{if $X_{i} \neq Y_{j}$}
	  \end{array} \right.
	\end{displaymath}
    \end{itemize}
  \end{block}

  \begin{alertblock}{Remark.}
    $\max(m,n) \le L(m,n) \le m+n$
  \end{alertblock}
\end{frame}
%%%%%%%%%%
\begin{frame}{2-D DP (part 1)}
  \begin{exampleblock}{Edit distance revisited}
    \begin{displaymath}
      \text{ED}[i,j] = \min \left\{ \begin{array}{ll}
	\text{ED}[i-1,j] + 1 &  \\
	\text{ED}[i,j-1] + 1 & \\
	\text{ED}[i-1,j-1] + \text{I}\set{X_{i} = Y_{j}} &
      \end{array} \right.
    \end{displaymath}
  \end{exampleblock}
    
  \uncover<2->{
  \begin{exampleblock}{}
    \begin{displaymath}
      \text{ED}[i,j] = \left\{ \begin{array}{ll}
        \text{ED}[i-1,j-1] & \text{if } X_{i} = Y_{j}  \\
        \min \left\{ \begin{array}{ll}
          \text{ED}[i-1,j] + 1 &  \\
          \text{ED}[i,j-1] + 1 &  \\
          \text{ED}[i-1,j-1] + 1 \\
	\end{array} \right. & \text{if } X_{i} \neq Y_{j}
      \end{array} \right.
    \end{displaymath}
  \end{exampleblock}
  }

  \uncover<3->{
  \begin{theorem}
    If $X_{i} = Y_{j}$, then $\text{ED}[i-1,j-1] \le \text{ED}[i-1,j] + 1$.
  \end{theorem}
  }
\end{frame}
%%%%%%%%%%
\begin{frame}<beamer:0>{2-D DP (part 1)}
  \begin{exampleblock}{Shuffle of strings \pno{2.2.12}}
    \begin{itemize}
      \item $X[1 \cdots m]; Y[1 \cdots n], Z[1 \cdots m+n]$
      \item is $Z$ a shuffle of $X$ and $Y$?
    \end{itemize}
  \end{exampleblock}

  \begin{block}{Solution.}
    \begin{itemize}
      \item $S[i,j]$: Is $Z[1 \cdots i+j]$ a shuffle of $X[1 \cdots i]$ and $Y[1 \cdots j]$?
      \item goal: $S[m,n]$
      \item question: what is the relation among $X_{i}, Y_{j}, \text{ and } Z_{i+j}$?
    \end{itemize}
  \end{block}
\end{frame}
%%%%%%%%%%
\begin{frame}<beamer:0>{2-D DP (part 1)}
  \begin{block}{Solution.}
    \begin{itemize}
      \item recurrence:
	\begin{displaymath}
	  S[i,j] = \left\{ \begin{array}{ll}
	    \text{false} & \textrm{if $Z_{i+j} \neq X_{i} \land Z_{i+j} \neq Y_{j}$}\\
	    S[i-1,j] & \textrm{if $Z_{i+j} = X_{i} \land Z_{i+j} \neq Y_{j}$}\\
	    S[i,j-1] & \textrm{if $Z_{i+j} \neq X_{i} \land Z_{i+j} = Y_{j}$}\\
	    S[i-1,j] \lor S[i,j-1] & \textrm{if $Z_{i+j} = X_{i} = Y_{j}$}
	  \end{array} \right.
	\end{displaymath}
      \item initialization: 
	\begin{align*}
	  S[0,0] &= \text{true} \\
	  S[0,j] &= \left\{ \begin{array}{ll}
	    \text{true} & \text{if } Y = Z  \\
	    \text{false} & \text{if } Y \neq Z
	  \end{array} \right. \\
	  S[i,0] &= \left\{ \begin{array}{ll}
	    \text{true} & \text{if } X = Z  \\
	    \text{false} & \text{if } X \neq Z
	  \end{array} \right.
	\end{align*}
    \end{itemize}
  \end{block}
\end{frame}
%%%%%%%%%%
\subsection{2-D DP (part 2)}

%%%%%%%%%%
\begin{frame}{2-D DP (part 2)}
  \begin{exampleblock}{Longest contiguous substring both forward and backward \pno{2.2.9}}
    \begin{itemize}
      \item string $T[1 \cdots n]$
      \item to find LCS both forward and backward
    \end{itemize}

    \begin{center}
      d\textcolor{blue}{ynam}icprogramming\textcolor{blue}{many}times
    \end{center}
  \end{exampleblock}

  \begin{alertblock}{Trial and error.}
    \begin{itemize}
      \item try subproblem $L[i]$: the length of an LCS in $T[1 \cdots i]$
      \item try subproblem $L[i,j]$: the length of an LCS in $T[i \cdots j]$
    \end{itemize}
  \end{alertblock}
\end{frame}
%%%%%%%%%%
\begin{frame}{2-D DP (part 2)}
  \begin{block}{Solution.}
    \begin{itemize}
      \item $L[i,j]$: the length of an LCS \textcolor{blue}{starting with $T_{i}$ and ending with $T_{j}$}
      \item goal: $\max_{1 \le i \le j \le n} L[i,j]$ (simply $O(n^3)$)
      \item<2-> question: Is $T_{i} = T_{j}$?
      \item<2-> recurrence: 
	\begin{displaymath}
	  L[i,j] = \left\{ \begin{array}{ll}
	    0 & \textrm{if $T_{i} \neq T_{j}$}  \\
	    L[i+1,j-1] + 1 & \textrm{if $T_{i} = T_{j}$}
	  \end{array} \right.
	\end{displaymath}
      \item<3-> initialization: 
	\begin{align*}
	  L[i,i] &= 0, 0 \le i \le n  \\
	  L[i,i+1] &= \left\{ \begin{array}{ll}
	    1 & \text{if } T_{i} = T_{i+1}  \\
	    0 & \text{if } T_{i} \neq T_{i+1}
	    \end{array} \right.
	\end{align*}
    \end{itemize}
  \end{block}
\end{frame}
%%%%%%%%%%
\begin{frame}[fragile]{2-D DP (part 2)}
  \begin{block}{Code: three ways of filling the table.}
    \fignocaption{width = 0.50\textwidth}{fig/three-ways-filling-table.png}

    \begin{center}
        \begin{verbatim}
          for d = 2 to n-1
            for i = 1 to n-d
              j = i + d
              ...
          return max_{1 <= i <= j <= n} L[i,j]
       \end{verbatim}
    \end{center}
  \end{block}
\end{frame}
%%%%%%%%%%
\begin{frame}[fragile]{2-D DP (part 2)}
  \begin{block}{Code: three ways of filling the table.}
    \begin{columns}[t]
      \column{0.50\textwidth}
        \begin{verbatim}
          for i = n-2 to 1
            for j = i+2 to n
              ...
          return ...
        \end{verbatim}
      \column{0.50\textwidth}
        \begin{verbatim}
          for j = 3 to n
            for i = j-2 to 1
              ...
          return ...
       \end{verbatim}
    \end{columns}
  \end{block}
\end{frame}
%%%%%%%%%%
\begin{frame}<beamer:0>{2-D DP (part 2)}
  \begin{exampleblock}{Longest subsequence palindrome \pno{2.2.15 (a)}}
    \begin{itemize}
      \item string $S[1 \cdots n]$
      \item to find (the length of) a longest subsequence palindrome
    \end{itemize}
  \end{exampleblock}

  \pause
  \begin{block}{Solution.}
    \begin{itemize}
      \item subproblem $L[i,j]$: the length of the LSP in $S[i \cdots j]$
      \item goal: $L[1,n]$
      \item question: is $S[i] = S[j]$?
      \item recurrence
    \end{itemize}
  \end{block}
\end{frame}
%%%%%%%%%%
\begin{frame}<beamer:0>{2-D DP (part 2)}
  \begin{exampleblock}{Longest subsequence palindrome \pno{2.2.15 (b)}}
    \begin{itemize}
      \item string $S[1 \cdots n]$
      \item decompose into a sequence of palindromes
    \end{itemize}
  \end{exampleblock}

  \begin{block}{Solution.}
    \begin{itemize}
      \item $\text{Num}[i,j]$: the minimum number of palindromes obtained from $S[i \cdots j]$
      \item subproblem $\text{MinPals}[i]$: the minimum number of palindromes obtained from $S[1 \cdots i]$
      \item goal: $\text{MinPals}[n]$
      \item question: what is the start index of the last palindrome?
      \item recurrence:
    \end{itemize}
  \end{block}
\end{frame}
%%%%%%%%%%
\begin{frame}{2-D DP (part 2)}
  \begin{exampleblock}{String split problem \pno{2.2.16}}
    \begin{itemize}
      \item split a string $S$ into many pieces
      \item cost $|S| = n \Rightarrow n$
      \item given locations of $m$ cuts: $\textcolor{gray}{C_{0}}, C_{1}, \cdots, C_{m}, \textcolor{gray}{C_{m+1}}$
      \item to find the MinCost of splitting $S$ into $m+1$ pieces $S_{0} \cdots S_{m}$
    \end{itemize}
  \end{exampleblock}

  \begin{block}{Solution.}
    \begin{itemize}
      \item subproblem: $\text{MinCost}[i,j]$: the minimum cost of splitting substring $S_{i} \cdots S_{j-1}$ using cuts $C_{i+1} \cdots C_{j-1}$
      \item goal: $\text{MinCost}[0,m+1]$
    \end{itemize}
  \end{block}
\end{frame}
%%%%%%%%%%
\begin{frame}{2-D DP (part 2)}
  \begin{block}{Solution.}
    \begin{itemize}
      \item question: what is the first cut in $C_{i+1} \cdots C_{j-1}$?
      \item recurrence:
	\[
	  \text{MinCost}[i,j] = \min_{i < k < j} \left( \text{MinCost}[i,k] + \text{MinCost}[k,j] + l(S_{i} \cdots S_{j-1}) \right)
	\]
      \item initialization:
	\[
	  \text{MinCost}[i, i+1] = 0
	\]
    \end{itemize}
  \end{block}
\end{frame}
%%%%%%%%%%

\section{DP on Graphs}

%%%%%%%%%%%%%%%%%%%%
\begin{frame}{Minimum vertex cover on trees}
  \begin{exampleblock}{Minimum vertex cover on trees \pno{2.2.18}}
    \begin{itemize}
	  \item Undirected tree $T = (V, E)$; \textcolor{red}{No designated root!}
      \item Compute (the size of) a minimum vertex cover of $T$
    \end{itemize}
  \end{exampleblock}

  \fignocaption{width = 0.30\textwidth}{figs/vertex-cover.png}
\end{frame}
%%%%%%%%%%%%%%%%%%%%
\begin{frame}{Minimum vertex cover on trees}
  \centerline{Rooted $T$ at any node $r$.}
  \pause
  \vspace{0.30cm}

  \begin{description}
	\item[Subproblem:] $I(u)$: the size of an MVC of subtree $T_{u}$ rooted at $u$
	\item[Goal:] $I(r)$
	  \pause
	\item[Make choice:] Is $u$ in $\text{MVC}[u]$?
	\item[Recurrence:] 
	  \begin{align*}
		I(u) = \min \{\text{\# children of } u &+ \sum_{v: \text{ grandchildren of } u} I(v), \\
			1 &+ \sum_{v: \text{ children of } u} I(v)\}
	  \end{align*}
	  \pause
	\item[Init:] $I(u) = 0$, if $u$ is a leave
  \end{description}
\end{frame}
%%%%%%%%%%%%%%%%%%%%
\begin{frame}{Minimum vertex cover on trees}
  \begin{columns}
	\column{0.35\textwidth}
	  DFS on $T$ from root $r$:

	  \vspace{0.50cm}
	  \begin{algorithmic}
		\State when $u$ is ``finished'':
		\If{$u$ is a leave}
		  \State $I(u) \gets 0$
		\Else
		  \State $I(u) \gets \dots$ 
		\EndIf
	  \end{algorithmic}
	  \pause
	\column{0.60\textwidth}
	  Greedy algorithm:

	  \vspace{0.50cm}
	  \begin{theorem}
		There is an MVC which contains no leaves.
	  \end{theorem}
  \end{columns}
\end{frame}
%%%%%%%%%%%%%%%%%%%%
\begin{frame}{DP on DAG}
  \begin{exampleblock}{Longest path in DAG (Problem 7.17)}
	\begin{itemize}
	  \item Direction: $\downarrow$ OR $\rightarrow$
	  \item Score: $>=< 0$
	\end{itemize}
  \end{exampleblock}

  \pause
  \begin{enumerate}
	\item digraph $G$
	\item node weight $\to$ edge weight
	\item adding an extra sink $s$
	\item $G \to G^{T}$
  \end{enumerate}

  \pause
  \centerline{Compute a longest path from $s$ in DAG}
\end{frame}
%%%%%%%%%%%%%%%%%%%%
\begin{frame}{DP on DAG}
  \begin{description}
	\item[Subproblem:] $\text{dist}[v]$: longest distance from $s$ to $v$ 
	\item[Goal:] $\text{dist}[v], \forall v \in V$
	  \pause
	\item[Make choice:] 
	\item[Recurrence:] 
	  \[
		\text{dist}[v] = \max_{u \to v} \left(\text{dist}[u] + w(u \to v)\right) 
	  \]
	  \pause
	\item[Init:] $\text{dist}[s] = 0$
  \end{description}

  \pause
  \vspace{0.60cm}
  \centerline{Compute $\text{dist}[v]$ in topo. order}
\end{frame}
%%%%%%%%%%%%%%%%%%%%
\begin{frame}{Bitonic tour}
  \begin{exampleblock}{Bitonic tour (Problem 7.18)}
  \end{exampleblock}
\end{frame}
%%%%%%%%%%%%%%%%%%%%
\begin{frame}{Bitonic tour}
\end{frame}
%%%%%%%%%%%%%%%%%%%%

% file: sections/knapsack-dp.tex

%%%%%%%%%%%%%%%%%%%%
\begin{frame}{}
  \centerline{\teal{\Large The Knapsack Problem}}
\end{frame}
%%%%%%%%%%%%%%%%%%%%

%%%%%%%%%%%%%%%%%%%%
\begin{frame}{}
  \begin{exampleblock}{The Change-making Problem (Problem $14.13$)}
    \begin{itemize}
      \item Coins values: $x_{1} \dots x_{n}$
      \item Amount: $v$
      \item Is it possible to make change for $v$?
    \end{itemize}
  \end{exampleblock}
\end{frame}
%%%%%%%%%%%%%%%%%%%%
\begin{frame}{}
  \begin{exampleblock}{The Change-making Problem (Problem $14.13\; (2)$, Problem $14.2$ (Subset sum))}
    \begin{enumerate}[(1)]
      \setcounter{enumi}{1}
      \item Without repetition ($0/1$)
    \end{enumerate}
  \end{exampleblock}

  \pause
  \begin{description}
    \item[Subproblem:] $C[i, w]$: Make change for $w$ using only values of $x_{1} \dots x_{i}$?
    \item[Goal:] $C[n,v]$
      \pause
    \item[Make choice:] Using value $x_{i}$ or not?
    \item[Recurrence:] 
      \[
	C[i,w] = C[i-1, w] \lor (C[i-1, w-x_{i}] \textcolor{red}{\;\land\; w \ge x_{i}})
      \]
      \pause
    \item[Init:]
      \begin{align*}
	C[i,0] &= \text{true}  \; \forall i = 0 \dots n  \\
	C[0,w] &= \text{false}, \;\text{if } w > 0 \\
	C[0,0] &= \text{true}
      \end{align*}
    \item[Time:] $O(nv)$
  \end{description}
\end{frame}
%%%%%%%%%%%%%%%%%%%%
\begin{frame}{}
  \begin{exampleblock}{The Change-making Problem (Problem $14.13\; (1)$)}
    \begin{enumerate}[(1)]
      \item Unbounded repetition ($\infty$)
    \end{enumerate}
  \end{exampleblock}

  \pause
  \begin{description}
    \item[Subproblem:] $C[i, w]$: Make change for $w$ using only values of $x_{1} \dots x_{i}$?
    \item[Goal:] $C[n,v]$
      \pause
    \item[Make choice:] Using value $x_{i}$ or not?
    \item[Recurrence:] 
      \[
	C[i,w] = C[i-1, w] \lor (C[\textcolor{red}{i}, w-x_{i}] \land w \ge x_{i})
      \]
      \pause
    \item[Init:]
      \begin{align*}
	C[i,0] &= \text{true}, \; \forall i = 0 \dots n  \\
	C[0,w] &= \text{false}, \; \text{if } w > 0 \\
	C[0,0] &= \text{true}
      \end{align*}
    \item[Time:] $O(nv)$
  \end{description}
\end{frame}
%%%%%%%%%%%%%%%%%%%%
%%%%%%%%%%%%%%%%%%%%
\begin{frame}{}
  \begin{exampleblock}{The Change-making Problem (Problem $14.13\; (3)$)}
    \begin{enumerate}[(1)]
      \setcounter{enumi}{2}
      \item Unbounded repetition with $\le k$ coins
    \end{enumerate}
  \end{exampleblock}

  \pause
  \begin{description}
	\item[Subproblem:] $C[i,w,l]$: Possible to make change for $w$ with $\le l$ coins of values of $x_{1} \dots x_{i}$?
	\item[Goal:] $C[n,v,k]$
	  \pause
	\item[Make choice:] Using value $x_{i}$ or not? 
	\item[Recurrence:] 
	  \[
		C[i,w,l] = C[i-1,w,l] \lor \left( C[\textcolor{red}{i}, w-x_{i}, \textcolor{red}{l-1}] \land w \ge x_{i} \right)
	  \]
	  \pause
	\item[Init:]
	  \begin{align*}
		C[0,0,l] &= \text{true}, \quad C[0,w,l] = \text{false}, \text{if } w > 0 \\
		C[i,0,l] &= \text{true}, \quad C[i,w,0] = \text{false}, \text{if } w > 0
	  \end{align*}
  \end{description}
\end{frame}
%%%%%%%%%%%%%%%%%%%%


\tx{}

\end{document}
%%%%%%%%%%
