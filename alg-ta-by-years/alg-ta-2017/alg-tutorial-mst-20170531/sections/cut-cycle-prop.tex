\section{Cut Property and Cycle Property}

%%%%%%%%%%%%%%%%%%%%
\begin{frame}{A generic MST algorithm}
\end{frame}
%%%%%%%%%%%%%%%%%%%%
\begin{frame}{Cut property (strong)}
  \begin{exampleblock}{Cut property (strong)}
	\begin{itemize}
	  \item Graph $G = (V, E)$
	  \item $X$ is some part of an MST $T$ of $G$
	  \item Any cut $(S, V \setminus S)$ \emph{s.t.} $X$ does not cross $(S, V \setminus S)$
	 ­\item Let $e$ be a lightest edge across $(S, V \setminus S)$
	\end{itemize}
	Then $X \cup \set{e}$ is some part of an MST $T'$ of $G$.
  \end{exampleblock}

  \begin{proof}
	\centerline{Exchange argument}
  \end{proof}
\end{frame}
%%%%%%%%%%%%%%%%%%%%
\begin{frame}{Cut property (strong)}
  Correctness of Prim's and Kruskal's algorithms.
\end{frame}
%%%%%%%%%%%%%%%%%%%%
\begin{frame}{Cut property (weak)}
  \begin{exampleblock}{Cut Property \pno{3.6.18 (a)}}
    \begin{itemize}
	  \item Graph $G = (V, E)$
	  \item Any cut $(S, V \setminus S)$ where $S, V-S \neq \emptyset$
	  \item Let $e = (u,v)$ be \emph{a} minimum-weight edge across $(S, V \setminus S)$
	\end{itemize}
	Then $e$ must be in \emph{some} MST of $G$.
  \end{exampleblock}

  \begin{center}
	``a'' $\to$ ``the'' $\implies$ ``some'' $\to$ ``any''
  \end{center}
\end{frame}
%%%%%%%%%%%%%%%%%%%%
\begin{frame}{Applications of cut property}
  \begin{exampleblock}{Application of cut property (Problem 6.10)}
	\begin{enumerate}[(1)]
	  \setcounter{enumi}{2}
	  \item (Problem 6.10--3) $e \in G$ is a lightest edge $\implies$ $e \in \exists$ MST of $G$
	  \item $e \in G$ is the unique lightest edge $\implies$ $e \in \forall$ MST of $G$
	\end{enumerate}
  \end{exampleblock}
\end{frame}
%%%%%%%%%%%%%%%%%%%%
\begin{frame}{Applications of cut property}
  \begin{exampleblock}{Wrong divide\&conquer algorithm for MST (Problem 6.14)}
    \begin{itemize}
      \item $G = (V, E, w)$
      \item $(V_{1}, V_{2}): ||V_{1}| - |V_{2}|| \le 1$
      \item $T_{1} + T_{2} + \set{e}$: $e$ is a lightest edge across $(V_{1}, V_{2})$
    \end{itemize}
  \end{exampleblock}

  \fignocaption{width = 0.30\textwidth}{figs/divide-conquer-mst-counterexample.pdf}
\end{frame}
%%%%%%%%%%%%%%%%%%%%
\begin{frame}{Cycle property (weak)}
  \begin{exampleblock}{Cycle property (Problem 6.13--2)}
	\begin{itemize}
	  \item $G = (V,E,w)$
	  \item Let $C$ be any cycle in $G$
	  \item $e = (u,v)$ is \emph{a} maximum-weight edge in $C$
	\end{itemize}
	Then $\exists \textrm{ MST } T \text{ of } G: e \notin T$.
  \end{exampleblock}

  \begin{center}
	``a'' $\to$ ``the'' $\implies$ ``some'' $\to$ ``any''
  \end{center}
\end{frame}
%%%%%%%%%%%%%%%%%%%%
\begin{frame}{Applications of cycle property}
  \begin{exampleblock}{Anti-Kruskal algorithm (Problem 6.13--3)}
	\centerline{\href{https://en.wikipedia.org/wiki/Reverse-delete_algorithm}{Reverse-delete algorithm (wiki)}}

	\[
	  O(m \log n (\log \log n)^3)
	\]
  \end{exampleblock}

  \begin{proof}
	\textbf{Invariant: } If $F$ is the set of edges remained at the end of the while loop, 
	then there is some MST that are a subset of $F$.
  \end{proof}

  \begin{alertblock}{Reference}
	\begin{itemize}
	  \item ``On the Shortest Spanning Subtree of a Graph and the Traveling Salesman Problem'' by Kruskal, 1956.
	\end{itemize}
  \end{alertblock}
\end{frame}
%%%%%%%%%%%%%%%%%%%%
\begin{frame}{Application of cycle property}
  \begin{exampleblock}{(Problem 6.13--1)}
	\begin{enumerate}[(1)]
	  \item $e \notin$ any cycle of $G$ $\implies$ $e \in \forall$ MST
	\end{enumerate}
  \end{exampleblock}

  \centerline{By contradiction.}
\end{frame}
%%%%%%%%%%%%%%%%%%%%
\begin{frame}{Application of cycle property}
  \begin{exampleblock}{(Problem 6.10)}
	\begin{enumerate}[(1)]
	  \item $|E| > |V| - 1$, $e$ is the unique max-weight edge of $G$ $\implies$ $e \notin \forall$ MST
	  \item $\exists C \subseteq G$, $e$ is the unique max-weight edge of $G$ $\implies$ $e \notin \forall$ MST
	  \setcounter{enumi}{4}
	  \item Cycle $C \subseteq G$, $e \in C$ is the unique lightest edge of $G$ $\implies$ $e \in \forall$ MST
	\end{enumerate}
  \end{exampleblock}
\end{frame}
%%%%%%%%%%%%%%%%%%%%
\begin{frame}{Unique MST}
  \begin{exampleblock}{Unique MST (Problem 6.12--1)}
    Distinct weights $\implies$ unique MST.
  \end{exampleblock}
\end{frame}
%%%%%%%%%%%%%%%%%%%%
\begin{frame}{Unique MST}
  \begin{exampleblock}{Unique MST (Problem 6.12--2)}
	Unique MST $\centernot\implies$ Equal weights.
  \end{exampleblock}

  \fignocaption{width = 0.20\textwidth}{figs/unique-mst-unique-weight-counterexample.pdf}
\end{frame}
%%%%%%%%%%%%%%%%%%%%
\begin{frame}{Unique MST}
  \begin{exampleblock}{Unique MST (Problem 6.12--3)}
	Unique MST $\centernot\implies$ Minimum-weight edge across any cut is unique.
  \end{exampleblock}

  \fignocaption{width = 0.20\textwidth}{figs/unique-mst-cut-counterexample.pdf}

  \begin{theorem}
    Minimum-weight edge across any cut is unique $\implies$ Unique MST.
  \end{theorem}
\end{frame}
%%%%%%%%%%%%%%%%%%%%
\begin{frame}{Unique MST}
  \begin{exampleblock}{Unique MST (Problem 6.12--3)}
    Unique MST $\centernot\implies$ Maximum-weight edge in any cycle is unique.
  \end{exampleblock}

  \fignocaption{width = 0.20\textwidth}{figs/unique-mst-cycle-noncounterexample.pdf}

  \begin{theorem}[Conjecture]
	Maximum-weight edge in any cycle is unique $\implies$ Unique MST.
  \end{theorem}
\end{frame}
%%%%%%%%%%%%%%%%%%%%
\begin{frame}{Unique MST}
  \begin{exampleblock}{Unique MST (Problem 6.12--4)}
	Decide whether a graph has a unique MST?
  \end{exampleblock}

  \vspace{0.60cm}
  \centerline{Modify an MST by exchange argument?}
\end{frame}
%%%%%%%%%%%%%%%%%%%%
