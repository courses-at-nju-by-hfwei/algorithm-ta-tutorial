\section{Asymptotics}

%%%%%%%%%%%%%%%%%%%%
\begin{frame}{$\Omega\;(\omega), \Theta, O\;(o)$}
  \[
    O(g(n)) = \set{f(n) \mid \exists c > 0, \exists n_0, \forall n \ge n_0: 0 \le f(n) \le c g(n)}
  \]

  \pause

  \[
    \Omega(g(n)) = \set{f(n) \mid \exists c > 0, \exists n_0, \forall n \ge n_0: 0 \le c g(n) \le f(n)}
  \]

  \pause

  \begin{align*}
    \Theta(g(n)) &= \set{f(n) \mid \exists c_1 > 0, \exists c_2 >0, \exists n_0, \forall n \ge n_0: \\ 
		& 0 \le c_1 g(n) \le f(n) \le c_2 g(n)}
  \end{align*}

  \pause

  \[
	  o(g(n)) = \set{f(n) \mid \textcolor{red}{\forall c > 0}, \exists n_0, \forall n \ge n_0: 0 \le f(n) \le c g(n)}
  \]

  \pause

  \[
	  \omega(g(n)) = \set{f(n) \mid \textcolor{red}{\forall c > 0}, \exists n_0, \forall n \ge n_0: 0 \le c g(n) \le f(n)}
  \]
\end{frame}
%%%%%%%%%%%%%%%%%%%%
\begin{frame}{(Problem 1.2.6)}
  \begin{exampleblock}{Problem 1.2.6 (4)}
	\[
		f(n) = \Theta(g(n)) \iff f(n) = O(g(n)) \land f(n) = \Omega(g(n))
	\]
  \end{exampleblock}

  \pause

  \begin{exampleblock}{Problem 1.2.6 (5)}
	\begin{align*}
	  f(n) = O(g(n)) &\iff g(n) = \Omega(f(n)) \\
	  f(n) = o(g(n)) &\iff g(n) = \omega(f(n))
	\end{align*}
  \end{exampleblock}

  \pause
  
  \begin{alertblock}{$f(n) = O(g(n)) \lor g(n) = \Omega(f(n)) ?$}
	\pause
	\[
		f(n) = n, \quad g(n) = n^{1 + \sin n}
	\]
  \end{alertblock}

  \pause

  \begin{exampleblock}{Problem 1.2.6 (6)}
	\[
		\Theta(g(n)) \cap o(g(n)) = \emptyset
	\]
  \end{exampleblock}
\end{frame}
%%%%%%%%%%%%%%%%%%%%
\begin{frame}{$\Omega\;(\omega), \Theta, O\;(o)$}
  \begin{alertblock}{Reference}
	``Big Omicron and Big Omega and Big Theta'' by Donald E. Knuth, 1976.
  \end{alertblock}
\end{frame}
%%%%%%%%%%%%%%%%%%%%
\begin{frame}{(Problem 1.2.10)}
  \[
	\log(n!) = \Theta(n \log n)
  \]

  \pause
  \vspace{0.30cm}
  \centerline{Prove by definition.}

  \pause
  \vspace{0.50cm}

  \centerline{Exercise: Prove it by Mathematical Induction.}
\end{frame}
%%%%%%%%%%%%%%%%%%%%
\begin{frame}{Horner's rule (Problem 1.1.6)}
  \[
	P(x) = a_0 + a_1 x + a_2 x^2 + \cdots + a_{n-1} x^{n-1} + a_n x^n
  \]

  \pause
  \vspace{0.50cm}
  
  \centerline{Loop invariant (after the $k$-th loop):}

  \[
	 \sum_{i = n}^{i = n-k} a_i x^{k-(n-i)}
  \]
\end{frame}
%%%%%%%%%%%%%%%%%%%%
%%%%%%%%%%%%%%%%%%%%
